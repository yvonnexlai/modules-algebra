%%%%% META DATA %%%%%%%%%%%%
\newcommand\Course{Algebra for Secondary Mathematics Teaching} % e.g., Algebra, Geometry, Modeling, Statistics
\newcommand\Location{University of Nebraska-Lincoln} % affiliation and course
\newcommand\Term{Spring 2018} % term taught

 \newcommand\MODULES{$\textnormal{MODULE}(\textnormal{S}^2)$}
 \title{{\normalsize{Mathematics Of Doing, Understand, Learning, and Educating Secondary Schools} }\\  $\;$ \\ \MODULES: \\  \Course}
 \author{Adapted for \Location} 
 \date{Version \Term} 
 
 % Input a course graphic or leave the { } empty:
\newcommand\coursegraphic{DoubleRectangles.png} 
 
 %%%%%%%DOCUMENT FORMATTING %%%%%%%%%%
\documentclass[11pt]{article}
\linespread{1.03}% 6 lpi http://tex.stackexchange.com/questions/23824/6-lines-in-one-inch
\parskip6pt
\usepackage{amsmath, amsthm, amsfonts, amssymb, mathpazo, url, graphicx, stackrel, mdwlist, enumitem, mdframed, ifthen}
	% usual suspects, palatino, hyperlink capability, PDF graphics, symbol stacking, list customizations, boxes, ifthenelse macros
\usepackage[top=1in,bottom=1in,left=1in,right=1in]{geometry} % 8.5" x 11" pages with 1 inch margins
\usepackage[pdftex, bookmarks, colorlinks, breaklinks]{hyperref} % prettier hyperlinks
\usepackage[usenames,dvipsnames,svgnames,table]{xcolor} % defines colors for text and tikz graphics
\definecolor{darkred}{rgb}{0.8,0.1,0.2} % for hyperlinks
\definecolor{darkblue}{rgb}{0.2,0.1,0.7} % for hyperlinks
\hypersetup{linkcolor=darkred,citecolor=blue,filecolor=dullmagenta,urlcolor=darkblue} % colors for links
\usepackage[none]{hyphenat} % prettier hyphenating
\raggedright \parskip4pt  \parindent0pt 
\usepackage{array} % tables with paragraphs of set widths
\renewcommand{\arraystretch}{1.3} % makes tables more legible
\newcolumntype{L}[1]{>{\raggedright\let\newline\\\arraybackslash\hspace{0pt}}p{#1}}
\newcolumntype{C}[1]{>{\centering\let\newline\\\arraybackslash\hspace{0pt}}p{#1}}
\newcolumntype{R}[1]{>{\raggedleft\let\newline\\\arraybackslash\hspace{0pt}}p{#1}}
\usepackage{rotating} % rotating figures and tables, provides sidewaystable and sidewaysfigure
\usepackage{lipsum} % text testing


%%%%%%% DOCUMENT MANAGEMENT %%%%%%%%%%%%
% To do notes and commenting 
\usepackage{comment}

% View instructor notes 
\newif\ifinstructor 
% \instructortrue  % view as instructor 
 \instructorfalse  % view as student
  
%%%%%%% SECTION FORMATTING %%%%%%%%%%%%
% sections
\usepackage{titlesec}
\titleformat{\subsection}[block]{\Large \bfseries \filcenter}{}{0em}{}
\titleformat{\subsubsection}[block]{\large \scshape\filcenter}{}{0em}{}
\newcommand{\handout}{\subsubsection}
\newcommand\header[1]{\vspace*{4pt}\par {\large {\bf #1}}\par}
\newcommand\about{\textasciitilde}

% itemize - second layer is an open circle instead of dash
\def\labelitemii{$\circ$}

% table colors - mostly for fun
\definecolor{yellow}{RGB}{255, 255, 0}
\definecolor{red}{RGB}{226, 30, 60}
\definecolor{orange}{RGB}{255, 159, 12}
\definecolor{green}{RGB}{16, 168, 112}
\definecolor{blue}{RGB}{1,200,255}
\definecolor{periwinkle}{RGB}{200,200,255}
\definecolor{lightteal}{RGB}{200,250,250}
\definecolor{purple}{RGB}{113, 1, 232}
%\definecolor{pink}{RGB}{255,70,192}
\definecolor{pink}{RGB}{232,1,193}
\definecolor{gray}{RGB}{100, 100, 100}

% instructor notes
\ifinstructor 
\newenvironment{bignote}[1][Instructor note]% default note is an "Instructor Note"
	{\begin{mdframed}\raggedright{\bf #1.~}}
	{\end{mdframed}}  
\else \excludecomment{bignote}
\fi

\ifinstructor
\newcommand\smallnote[1]
	{\begin{mdframed}\raggedright  {\bf Instructor note.} {#1} \end{mdframed}}
\else \newcommand\smallnote[1]{}
\fi

\ifinstructor  \usepackage{todonotes} 
\else \usepackage[disable]{todonotes}
\fi

% in-class task
\newenvironment{task}
	{\begin{mdframed}[linecolor=lightgray, linewidth=3pt]\raggedright}
	{\end{mdframed}}

%%% GRAPHICS / TIKZ %%%%%%%%%%%
\graphicspath{{Images/}}

\usepackage{tikz}
\usepackage{tkz-euclide} % tikz package for Euclidean geometry
\usepackage{siunitx} % typesetting quantities
\usepackage{pgfplots} \pgfplotsset{compat=1.13} % plotting graphs
\usetikzlibrary{calc} % calculations within tikz
\usetkzobj{all} % needed for tkz-euclide package
 
%%%%% MATH NOTATION %%%%%%%%

\newcommand\tn{\textnormal}

% systems
\newcommand{\R}{\mathbb{R}}
\newcommand{\C}{\mathbb{C}}
\newcommand{\Q}{\mathbb{Q}}
\newcommand{\N}{\mathbb{N}}
\newcommand{\Z}{\mathbb{Z}}

% notation tweaking
\renewcommand\phi\varphi  % normal \phi looks too much like the empty set.
\renewcommand\subset\subseteq 
\renewcommand\supset\supseteq  % to be careful about strict subsets and nonstrict subsets
\newcommand\st{:}

% divisibility
\newcommand\divides{\;|\;}
\newcommand\notdivides{\hspace*{-2pt}\not |\;}

% trig and geometry
\newcommand\degrees{^\circ}

%%%%%%%%%% THEOREMS AND RELATED STRUCTURES %%%%%
\renewcommand\emph[1]{\underline{\bf{#1}}} % terminology

\newtheorem{theorem}{Theorem}[section]
\newtheorem{proposition}[theorem]{Proposition}
\newtheorem{lemma}[theorem]{Lemma}
\newtheorem{corollary}[theorem]{Corollary}
\newtheorem{claim}{Claim}

\theoremstyle{definition}
\newtheorem{definition}[theorem]{Definition}
\newtheorem{example}[theorem]{Example}
\newtheorem{problem}[theorem]{Problem}
\newtheorem{conjecture}[theorem]{Conjecture}
\newtheorem{question}[theorem]{Question}
\newtheorem{remark}[theorem]{Remark}
\newtheorem{case}{Case}

\newtheorem*{theorem*}{Theorem}
\newtheorem*{example*}{Example}
\newtheorem*{question*}{Question}
\newtheorem*{claim*}{Claim}
\newtheorem*{definition*}{Definition}

\newenvironment{solution}{{\it Solution.} }{\hfill {\color{lightgray}$\blacksquare$}}

\newcommand\qedpart[1]{ \hfill \framebox(6,6){\tiny #1}}
\renewcommand\qed{\hfill \framebox(6,6){}}
%%%%%%%%%%%%%%%%%%%%%%%%%%%%%%%% 	
%%%%%%%% DOCUMENT BEGINS %%%%%%%%%%%%
%%%%%%%%%%%%%%%%%%%%%%%%%%%%%%%% 

\begin{document}

%%%%%% COVER PAGE %%%%%%%%%%%%%%% 
\pagenumbering{gobble} % no page number
\maketitle
\ifthenelse{\equal{\coursegraphic}{}} % insert course graphic if one exists
	{}
	{\begin{center}\includegraphics[width=3in]{\coursegraphic}\end{center}}
	
\vfill 
% copyleft
\begin{center} \includegraphics[width=1in]{by-nc-sa.png} \end{center}
\footnotesize{ This work is licensed under a Creative Commons Attribution-ShareAlike 3.0 Unported License. }

 % acknowledgments 
\footnotesize{
The Mathematics Of Doing, Understand, Learning, and Educating Secondary Schools (\MODULES) project is partially supported by funding from a collaborative grant of the National Science Foundation under Grant Nos.~DUE-1726707,1726804, 1726252, 1726723, 1726744, and 1726098.  Any opinions, findings, and conclusions or recommendations expressed in this material are those of the authors and do not necessarily reflect the views of the National Science Foundation.}
\newpage
%%%%%% TABLE OF CONTENTS %%%%%%%%%%%%%%% 	
\thispagestyle{plain} \pagenumbering{roman}  
\listoftodos
\tableofcontents
\newpage \pagenumbering{arabic}
%%%%%%%%%%%%%%%%%%%%%%%%%%%%%%%% 
%%%%%%%%%%%%%%%%%%%%%%%%%%%%%%%% 	
%%%%%% PART 1 %%%%%%%%%%%%%%%%%%%%%	
%%%%%%%%%%%%%%%%%%%%%%%%%%%%%%%% 
%%%%%%%%%%%%%%%%%%%%%%%%%%%%%%%% 	
\newpage 
\part{How We Talk and Explore Math} 

\setcounter{section}{-1}

%%%%%%%%%%%%%%%%%%%%%%%%%%%%%%%%%%%%%%%%%%%%%
%%%%%% 0 COMMUNICATING MATHEMATICS IN THIS COURSE AND BEYOND %%%%%%%%%
%%%%%%%%%%%%%%%%%%%%%%%%%%%%%%%%%%%%%%%%%%%%%
\newpage \section{Communicating Mathematics in this Course and Beyond}\label{section: communicating mathematics}

%%%%%%%% 0 REFERENCE: SET & LOGICAL NOTATION %%%%%%%%%%%
\subsection{Set and Logical Notation}



{\bf Set Notation}

\hspace*{6pt}{\bf Definition \ref {definition: set}.} A \emph{set} is a collection of objects, which are called the \emph{elements} of the set.


\vspace*{8pt}
\begin{tabular}{L{1in}L{5in}}
$x\in D$ & ``$x$ is an element of the set $D$"  (a proposition about $x$ and its {\it domain} $D$)\\

$P(x)$ & A proposition about the variable $x$; may be true or false depending on $x$ \\

$\{x \in D\st P(x)\}$ 

$\{x\in D \hspace*{2.5pt}|\hspace*{2.5pt} P(x)\hspace*{.25pt}\}$ & The set of all elements of $D$ for which $P(x)$ is true (a subset of $D$). 

Note that the ``rule'' for set membership may be given in many ways, not just an algebraic rule. For example, graphical, table, list, description.\\ 

$A \subseteq B$ & ``$A$ is a subset of $B$"  (a proposition about sets $A$ and $B$)\\

$A \subsetneq B$ & ``$A$ is a strict subset of $B$'', i.e., ``$A\subseteq B$ and $A \neq B$" \\
$A \supseteq B$ & ``$A$ is a superset of $B$'' or ``$B$ is a subset of $A$" \\

$A \supsetneq B$ &  ``$A$ is a strict superset of $B$'' or ``$B$ is a strict subset of $A$'',  i.e., ``$A\supseteq B$ and $A \neq B$"   \\ 


$A \cap B$ & The intersection of the sets $A$ and $B$  (a set)\\

$A \cup B$ & The union of the sets $A$ and $B$  (a set)\\

$\emptyset$ & The {\it empty set} (the set with no elements); also known as {\it null set}\\
$|A|$ & The cardinality (``size'') of $A$. When $A$ is finite, $|A|$ is the number of elements in $A$.
\end{tabular}

\vspace*{8pt}

\begin{tabular}{L{6in}}{\bf Note.} The notation for subset (without the bottom line) is ambiguous: some people use it to mean $A \subseteq B$ and others use it to mean $A \subsetneq B$. So we don't use it here. \\ 

 {\bf Definition \ref{definition: set equality}}.
Given sets $A$ and $B$. We say $A$ \emph{is equal to} $B$ if $A\subset B$ and $B\subset A$. 

{\it Notation:} $A=B$.
\end{tabular}

\vfill 
{\bf Logical notation}

\begin{tabular}{L{1in}L{5in}}
$\neg P(x)$ & The negation of $P(x)$ \\

$\forall x, P(x)$ & The proposition ``For all values of $x$, $P(x)$ is true." \\

$\exists x: P(x)$ & The proposition ``There exists a value of $x$ such that $P(x)$ is true." \\

$\forall x, P(x) \Rightarrow Q(x)$ & The proposition ``For all values of $x$, if $P(x)$ is true then $Q(x)$ is true."\\

$\forall x, P(x) \Leftrightarrow Q(x)$ & The proposition ``For all values of $x$, $P(x)$ is true if and only if $Q(x)$ is true." 
\end{tabular}


\vfill
{\bf Proof structures}

\begin{tabular}{L{1in}L{5in}}
To show that \dots & Requires showing that \dots \\
{\bf $x\in A$} & $x$ satisfies set membership rules for $A$ \\
{\bf $x\notin A$} & $x$ does not satisfy at least one set membership rule of $A$ \\ 
{\bf $A\subset B$} & If $x\in A$, then $x\in B$ \\ 
{\bf $A\subsetneq B$} & (1) $A\subset B$ \quad (2) there is an element of $B$ that is not in $A$ \\ 
{\bf $A=B$} & (1) $A\subset B$ \quad (2) $B\subset A$
\end{tabular}

\vfill 

{\bf Sets of numbers}

\begin{tabular}{ll}
$\N$ & The set of {\em natural numbers} (positive whole numbers) \\
$\Z$ & The set of {\em integers} (all whole numbers -- positive, negative, and zero) \\
$\Q$ & The set of {\em rational numbers} (all fractions) \\
$\R$ & The set of {\em real numbers} (all numbers on the real line; equivalently, all decimal numbers) \\
$\C$ & The set of {\em complex numbers} (all numbers of the form $a+bi$, where $a$ and $b$ are real)
\end{tabular}


%%%%%%%% 0 REFERENCE: PROPERTIES OF R and Z%%%%%%%%%%%
\newpage
\subsection{Properties of $\Z$, $\Q$, $\R$ and $\C$} \label{s: properties of R and Z}

{\bf Operations are well-defined}
\vspace*{-4pt}
\begin{itemize}
\item[] Well-defined: There is an answer, and there isn't more than one answer.

\item[] Operations $+, -, \times$ on $\R$ and $\C$ are well-defined: This means that when we add two numbers, we get exactly one answer (we don't expect there two be two answers to ``What is $a+b$?'' and we expect that there is an answer); similarly, when we subtract one number from another, or multiply two numbers, we get exactly one answer.

\item[] Division by nonzero numbers is well-defined. (There is no good numerical answer to ``What is $a/0$?'')
\end{itemize}

{\bf Arithmetic Properties of $\Z$, $\Q$, $\R$ and $\C$}

We state them below for $\Z$. They also hold for $\R$.

\begin{tabular}{C{0.2in}|L{3.5in}|L{2.25in}}
1 & $a,b \in \Z \implies a+b\in \Z$ &  $\Z$ is closed under addition \\ 
2 & $a,b,c\in \Z\implies$ $a+(b+c)=(a+b)+c$ & Addition in $\Z$ is associative \\
3 & $a,b \in \Z \implies a+b=b+a$&  Addition in $\Z$ is commutative \\ 
4 & $a\in \Z \implies a+0=a=0+a$ & $0$ is an additive identity in $\Z$ \\ 
5 & $\forall a\in \Z$, the equation $a+x=0$ has a solution in $\Z$ & Additive inverses exist in $\Z$ \\ 
6 & $a,b\in\Z \implies ab\in \Z$ & $\Z$ is closed under multiplication \\ 
7 & $a,b,c\in \Z \implies a(bc)=(ab)c$ & Multiplication in $\Z$ is associative \\ 
8 & $a,b,c\in \Z\implies$  $a(b+c)=ab+ac$ and $(a+b)c=ac+bc$& Distributive property \\
9 & $a,b\in \Z\implies ab=ba$ & Multiplication in $\Z$ is commutative \\  
10 &$a\in\Z\implies a\cdot 1=a=1\cdot a$ & $1$ is a multiplicative identity in $\Z$ \\ 
11 & $a,b\in \Z, ab=0 \implies a=0 \tn { or } b=0$ & $\Z$ has no zero divisors
\end{tabular} 

\vspace*{4pt}
{\bf Divides, Divisor, Factor} 

\vspace*{-8pt}
\begin{itemize}
\item  Given $a, b\in \Z$, not both zero. We say \underline{$b$ {\it divides} $a$}  if $a=bc$ for some integer $c$.   Notation: $b\divides a$
		 
		 These all mean the same thing:
		\begin{itemize}[label={$\circ$}]
		\item $b$ divides $a$
		\item $b$ is a divisor of $a$
		\item $b$ is a factor of $a$
		\item $b\divides a$
		\end{itemize}
		
		 If we want to say that $b$ does not divide $a$, we write $b\notdivides a$.

\item A factor of a number is \underline {\it trivial} if it is $\pm 1$ or the $\pm$ number.  A \underline{\it nontrivial} factor that is not trivial.

\item All nonzero natural numbers have a finite number of factors.

\item Let $a,b,c\in \Z$. If $a\divides b$ and $b\divides c$, then $a\divides c$.
\end{itemize}

{\bf Prime, Composite} 
\vspace*{-6pt}
\begin{itemize} 
\item An integer $p$ is \underline{{\it prime}} if $p\neq 0, \pm1 $ if the only divisors of $p$ are $\pm 1$ and $\pm p$. 

An integer $n$ is \underline{\it composite} if $n\neq 0, \pm 1$, and it is not prime.

\item Let $a\in \Z$. If $p,q$ are primes such that $p\divides a$ and $q\divides a$, and $p\neq q$, then $pq\divides a$.
\end{itemize}

{\bf Even number}
\vspace*{-6pt}
\begin{itemize} 
\item[] An integer $n$ is even if it is divisible by $2$.
\end{itemize}

{\bf Fundamental Theorem of Arithmetic} 
\vspace*{-6pt}
\begin{itemize}\item[] There is only one way to write any whole number as a product of positive primes (reordering doesn't count as a different way).
\end{itemize}


%%%%%%%% 0 REFERENCE: SAMPLE HANDWRITTEN PROOF %%%%%%%%%%%%

\newpage
\subsection{Sample handwritten proof}
\label{s: sample handwritten proof}

Let's use one of the proofs we did in class as an example.  We begin with the typed up version and then show one way that this same proof might be handwritten.

\begin{mdframed}
\begin{claim*}
If $A=\{ 3n \st n \in \Z\}$ and $B=\{ 6n \st n\in \Z\}$, then $B\subsetneq A$.
\end{claim*}
\begin{proof}
Given $A=\{ 3n \st n \in \Z\}$ and $B=\{ 6n \st n\in \Z\}$. 

\begin{enumerate}
\item {\it Why $B\subset A$}: This means showing: if $x\in B$, then $x\in A$.

Given $x\in B$. Then:
	\begin{eqnarray*}
	x &=&6k, k\in \Z, \tn{ by definition of $B$} \\ 
	   &=& 3\cdot 2k \\ 
	   &=& 3n, n\in \Z, \tn{ because $2\in \Z, k\in \Z$, and $\Z$ is closed under multiplication}
	\end{eqnarray*}
Therefore $x$ satisfies set membership rules of $A$, implying $x\in A$. 

We have shown that if $x\in B$, then $x\in A$. Thus $B\subset A$, by definition of subset. \qedpart{1}

\item {\it Why there is an element of $A$ that is not in $B$.} We find an element of $A$ that is not in $B$.
If $x\in B$, then $x$ is an even number because if $x=6k$ for some $k\in\Z$, then $x$ as $x=2\cdot (3k)$. Closure of multiplication in $\Z$ implies $3k\in \Z$, so $x$ satisfies the definition of even number. 

However, some members of $A$ are odd numbers: $3, 9, 15, \dots $. 

Hence there are elements of $A$ that are not in $B$. \qedpart{2}

\end{enumerate}
Why this means $B\subsetneq A$: (1) and (2) show that $B$ and $A$ satisfy the definition of strict subset, and we have $B\subsetneq A$.
\end{proof}
\end{mdframed}

\begin{mdframed}\begin{center}
\includegraphics[width=3in]{0_SampleProof.pdf}
\end{center}
\end{mdframed}



%%%%%%%% 0 REFERENCE: GOOD PROOF COMMUNICATION %%%%%%%%%%%
\newpage
\subsection{Good proof communication}
\label{s: good proof communication}

Here is the same proof, with key features pointed out. These features are explained at the bottom. In general, you want to incorporate most if not all of these features into any proof you write. Even though it might seem strange at first, you may find eventually that you learn math better when you develop the habits of incorporating these features into your own writing and being aware of these features in proofs you encounter.

\begin{mdframed}
\begin{center}
\includegraphics[width=3in]{0_SampleProof_Annotated.pdf}
\end{center}
\end{mdframed}


\begin{minipage}{5.5in}
\begin{mdframed}\raggedright\parskip2pt
Features of communicating proof well:

({\bf Essential features in bold})

\begin{enumerate*}
\item {\bf Label the claim.}
\item {\bf State the claim precisely.}
\item {\bf Label the proof beginning.}
\item Begin a proof by reminding yourself and readers of the starting point:\\ the conditions of the claim.
\item {\bf End the proof with where you need to go: the conclusions of the claim.}
\item Summarize your approach to the reader. 
\item {\bf Label the proof end.} A traditional way is to use a box.
\item {\bf Write up parts within a proof properly}. {\bf Label when they begin and end.}
	\begin{itemize} 
	\item Give them a name (e.g., Claim A) if it is a proof within a proof
	\item {\bf Use labels like $[\Rightarrow]$ and $[\Leftarrow]$ if doing an if and only if proof.}
	\end{itemize}

\item Diagrams are good only if you explain what you are showing. Give a caption.
\end{enumerate*}
\end{mdframed}
\end{minipage}

%%%%%%%%%%%%%%%%%%%%%%%%%%%%%%%% 	
%%%%%% LESSON 1 %%%%%%%%%%%%%%%%%%%%	
%%%%%%%%%%%%%%%%%%%%%%%%%%%%%%%% 
\newpage
\section{Sets, Claims, Negations (Week 1) (Length: 2.5 hours)}  % Use Title Case for Title
\label{s: sets}
%%%%%%%%% 1 OVERVIEW  %%%%%%%%%%%%%%%%%%%%
\subsection{Overview}
 
 \vspace*{-16pt}
\begin{tabular}{L{6.5in}} 
{\bf Content} \\ \hline \parskip4pt
\emph{``Parent'' relation}, implicitly defined as a relation which assigns elements of $\N$ to its factors; used to examine subsets, mathematical statements and their negations, properties of $\R$ and $\Z$, and to engage in mathematical practices. 

(Looking ahead:) The parent relation is used in Section \ref{s: relations} to introduce relations and inverse relations.

\emph{Subset}, \emph{superset}, \emph{strict subset}, and \emph{strict superset}; \emph{equality of sets} $A$ and $B$, defined as $A\subset B$ and $B\subset A$.

\emph{Mathematical statements}, defined as those which can be evaluated as true or false; and 

\emph{Negation} of mathematical statement $S$, defined as a statement which is false if and only if $S$ is true.

\emph{Properties of $\R$ and $\Z$} assumed. (These may have been introduced previously in an abstract algebra course.)
\end{tabular}

\begin{tabular}{L{3.2in}|L{3.2in}}
{\bf Proof Structures} & {\bf Mathematical/Teaching Practices} \\ 
\hline \parskip4pt
\emph{To show that $x\in A$} means showing that $x$ satisfies set membership rules for $A$; and \emph{to show that $x\notin A$} means showing that $x$ does not satisfy at least one set membership rule of $A$.

\emph{To show that $A\subset B$ }requires showing that if $x\in A$, then $x\in B$.

\emph{To show that $A\subsetneq B$} requires showing that: (1) $A\subset B$; (2) there is an element of $B$ that is not in $A$.

\emph{To show that $A=B$} requires showing that: (1) $A\subset B$; (2) $B\subset A$.
&
% Mathematical/Teaching Practices
\parskip6pt
\emph{Clarifying mathematical questions}, meaning to determine how different interpretations of question statements may have different mathematical consequences.
 
\emph{Conjecturing and being precise}, in the sense of giving ``satisfying'' answers to mathematical questions

\emph{Communicating proofs well}, which includes specifying claims, the body of the proof, and givens and conclusions explicitly, clearly, and correctly.
\end{tabular}



%%%%%%%%%%%%%%
\header{Summary}
 
We introduce the ``parent relation'' as a context for engaging in mathematical practices as well as learning how to work with each other on exploratory tasks.  The main tasks in this lesson are:
	\begin{itemize}
	\item {\it Which numbers have more than one pair of parents?}  
	\item {\it Is one of these sets a subset of the other set? Check the mathematically correct statements. If you put a check in the $A\neq B$ column, list an element that is in one but not the other.}		
		\begin{tabular}{C{2.1in}||C{0.37in}|C{0.37in}|C{0.37in}|C{0.37in}|C{0.37in}|C{0.4in}|C{0.5in}}
			& $A\subset B$ & $A\subsetneq B$ & $A\supset B$ & $A\supsetneq B$ & $A = B$& $A\neq B$ 
			& \tiny{Neither is subset of the other} \\ \hline	
		$A= \tn{multiples of 3}$, 
		$B= \tn{multiples of 6}$ 
			& & & & & & & \\ \hline
		$A= \tn{multiples of 6}$, 
		$B= \tn{multiples of 9}$ 
			 & & & & & & & \\  \hline
		$A=\{ n^2 \st n\in \N, n>0 \}$, 
		$B=\{1+3+\dots +(2n+1) \st n\in \N\}$ 
			& & & & & & & \\ \hline	
		$A=\tn{functions of the form $x\mapsto 16^{ax}$}$, 
		$B=\tn{functions of the form $x\mapsto 2^{ax}$}$
			& & & & & & & \\  \hline
		\end{tabular}
	\end{itemize}

Along the way we introduce notation for sets and subsets, discuss mathematical statements and their negations, and describe properties of $\R$ and $\Z$ assumed for now. There are also tasks in this lesson addressing these ideas.

%%%%%%%%%%%%%%
{\it Acknowledgements.} The structure and some tasks of \nameref{s: set notation} and \nameref{s: statements and negations} are from notes from Mira Bernstein and used with permission.

%%%%%%%%%%%%%%
\newpage
\begin{bignote}[Materials]
\begin{itemize}
\item All pages in Section \ref{section: communicating mathematics}: Communicating Mathematics (can be printed double-sided)
\item Handouts in In-Class Resources (can be printed double-sided)
\item Colored chalk / markers to highlight different parts of good proof communication
\end{itemize}
\end{bignote}


%%%%%%%%%%%%%%%%%%%%%%%%%%%%%%%%%%%%%%%
%%%%%%%%%%%%%%%%%%%%%%%%%%%%%%%%%%%%%%%
\subsection{Opening inquiry: Number parents}
We begin this lesson with the following inquiry:

\begin{task}
Two numbers are parents of a child if the child is their product. 

A child cannot be its own parent.

Which numbers have more than one pair of parents?

\begin{center}
\begin{tabular}{C{1.5in}|C{1.5in}}
Child & Parents \\
\hline 
 6 & 2, 3 \\  
 4 & ?? \\  
 12 & 4, 3 \\ 
 12 & 2, 6 
\end{tabular}
\end{center}
\end{task}


\smallnote{
Distribute handout with this question. As teachers work on it, circulate and listen to the questions and comments they make. They may say and do things that will lead into a discussion on clarifying the question, precision, and also what it means to have less or more satisfying answers to a question.
}

As we discussed this question, we learned some issues that arise when asking and answering mathematical questions:

\begin{itemize}
\item {\it Clarifying the question.}  Let's assume that we are only working with natural numbers (0, 1, 2, \dots), and that 2, 2 is a set of parents for 4. So we are looking for natural numbers that have more than one pair of parents. We allow pairs of parents to repeat parents.
\item {\it Finding and improving possible answers (conjecturing well).} Here are some possible answers (without explanations) to this question. Which is the most satisfying answer (without explanation)? Why?
	\begin{enumerate}
	\item 12 has more than one pair of parents.
	\item 12, 18, 20, 28, 30, 42, 44 each have more than one pair of parents.
	\item Any number with at least least three different factors has more than one pair of parents.
	\item Any number with at least three different factors (that aren't itself or 1) has more than one pair of parents.
	\item Any number with at least three different factors (that aren't itself or 1) has more than one pair of parents. There are no other numbers with more than one pair of parents.
	\end{enumerate}
\end{itemize}

\smallnote{The above are answers that prospective teachers in previous courses have given. You might use some of these answers as ringers for your own class discussion, or simply use a variety of answers that teachers in your class have given. The main thing is to have a variety of levels of how satisfying the answers are.}

We concluded that an answer is satisfying when it gives the most complete and correct understanding of a situation. We also gave the analogy of answering a question that a child asks, and that the quality of being ``satisfying'' when giving an answer to a mathematical may well be similar to what makes an answer ``satisfying'' to a child. 

\begin{itemize}
\item The first two are dissatisfying because they don't give any sort of pattern or big picture of what's going on. They
raise the question: ``Are those the only ones?'' 
\item The third one is almost there, but is actually slightly incorrect. The fourth one is getting there, and it is correct. But still, neither answer the question of whether there are more answers.
\item The fifth answer is the most satisfying because it provides the big picture of when a number works, and also says, yes, these are the only answers.
\end{itemize}

We also gave a name to the process of finding and improving answers to mathematical question: the practice of {\it conjecturing}. Before we get into proving or disproving our conjectures, we first talk about sets. This will give us a structure for addressing this inquiry more completely.

\vspace*{8pt}
\subsection{Sets, subsets, supersets, and set equality}

\subsubsection{Set notation}\label{s: set notation}


\vspace*{8pt}

\begin{definition}\label{definition: set} A \emph{set} is a collection of objects, which are called the \emph{elements} of the set.
\end{definition}

\vspace*{8pt}
\begin{tabular}{L{1in}L{5in}}
$x\in D$ & ``$x$ is an element of the set $D$"  (a proposition about $x$ and its {\it domain} $D$)\\

$P(x)$ & A proposition about the variable $x$; may be true or false depending on $x$ \\

$\{x \in D\st P(x)\}$ 

$\{x\in D \hspace*{3.5pt}|\hspace*{3.5pt} P(x)\hspace*{.5pt}\}$ & The set of all elements of $D$ for which $P(x)$ is true (a subset of $D$)\\ 

$A \subseteq B$ & ``$A$ is a subset of $B$"  (a proposition about sets $A$ and $B$)\\

$A \subsetneq B$ & ``$A$ is a strict subset of $B$'', i.e., ``$A\subseteq B$ and $A \neq B$" \\
$A \supseteq B$ & ``$A$ is a superset of $B$'' or ``$B$ is a subset of $A$" \\

$A \supsetneq B$ &  ``$A$ is a strict superset of $B$'' or ``$B$ is a strict subset of $A$'',  i.e., ``$A\supseteq B$ and $A \neq B$"   \\ 


$A \cap B$ & The intersection of the sets $A$ and $B$  (a set)\\

$A \cup B$ & The union of the sets $A$ and $B$  (a set)\\

$\emptyset$ & The {\it empty set} (the set with no elements); also known as {\it null set}\\
$|A|$ & The cardinality (``size'') of $A$. When $A$ is finite, $|A|$ is the number of elements in $A$.
\end{tabular}

\vspace*{8pt}

\begin{note} The notation for subset (without the bottom line) is ambiguous: some people use it to mean $A \subseteq B$ and others use it to mean $A \subsetneq B$. So we don't use it here.
\end{note}

\begin{definition}\label{definition: set equality}
Given sets $A$ and $B$. We say $A$ \emph{is equal to} $B$ if $A\subset B$ and $B\subset A$. We denote equality with $A=B$.
\end{definition}


\begin{task}
\begin{enumerate}
\item Let $A = \{1,2,\{3,4\}, \{5\}\}$.  
Decide whether each of the following statements is true or false: \\
	(\emph{Hint:} There are exactly six true statements.)
		\begin{align*}
			1 \in A,  && \{1, 2\} \in A, && \{1, 2\} \subseteq A, && \emptyset \in A,\\
			3 \in A,  && \{3, 4\} \in A, && \{3, 4\} \subseteq A, && \emptyset \subseteq  A,\\
			\{1\} \in A, && \{1\} \subseteq A, && \{5\} \in A, && \{5\} \subseteq A.
		\end{align*}


\item True or false? ``All students in this class who are under 5 years old are also over 100 years old.''
\end{enumerate}
\end{task}
\begin{solution}
\begin{enumerate}
\item $\;$ \\ \vspace*{-12pt} \begin{tabular}{lcclcclccl}
			(a) TRUE && (b) false && (c) TRUE && (d) false \\
			(e) false && (f) TRUE && (g) false && (h) TRUE\\
			(i) false && (j) TRUE && (k) TRUE && (l) false
		\end{tabular}
		
{\it Reasoning.} There are four elements of the set $A$:  
	\vspace*{-4pt}
	\begin{itemize*}
	\item 1 (the number 1)
	\item 2 (the number 2)
	\item $\{3,4\}$ (the set containing the numbers 3, 4)
	\item $\{5\}$  (the set containing the number 5)
	\end{itemize*}
	
The notation $\in$ means ``is an element of'' is . That's why (a), (f), (k) are TRUE and (b), (d), (e), (i)  are false.

The notation $\subset$ means ``is a subset of''. The set is a subset of $A$ if each of its elements are also elements of $A$.  That's why (c), (j) are TRUE and (g), (l) are false.

Finally, (h) is TRUE on a technicality. It contains no elements. So all zero of its elements are part of A.  The empty set is a subset of any set for this reason. 

\item For most sections of mathematics courses at university level, this statement should be TRUE.\end{enumerate}	

\vspace*{-18pt}$\;$
\end{solution}

\begin{note}
One helpful metaphor may be thinking of the braces (the $\{$ and $\}$ ) as permanent packaging, like gift wrap that doesn't come off. You can't take out what's inside the packaging. You can only hold the whole package. Even if only one thing is wrapped, you still can't hold the thing by itself, you can only hold it with its gift wrap. But if an object not wrapped, you can hold that object by itself.
\end{note}

{\bf Proof Structure: Showing set membership.} To show that $x\in S$ means showing that $x$ satisfies set membership rules for $S$; to show that $x\notin S$ means showing that $x$ does not satisfy at least one set membership rule of $A$.

\begin{task}
Let $S=\{ x\in \Q \st x \tn{ can be written as a fraction with denominator 2 and } |x|<2 \}$. 

True or false? \quad\quad $0.5\in S$, \quad\quad $3.5 \in S$, \quad\quad $0.25\in S$,\quad\quad $1\in S$.
\end{task}

\begin{solution}{\it (Partial)}

\vspace*{-8pt}
\begin{enumerate}[label=(\alph*)]
\item $0.5 \in S$ is TRUE because it can be written as the fraction $\frac{1}{2}$ and $|0.5|<2$. The number $0.5$ satisfies all the rules of membership of $S$, so it is an element of $S$.
\item $3.5 \in S$ is FALSE because even though it can be written as the fraction $\frac{7}{2}$, it does not satisfy the condition $|x|<2$. The number $3.5$ does not satisfy all the rules of membership of $S$, so it is not an element of $S$.
\item $0.25\in S$ is FALSE. (Why?)  
\item $1\in S$ is TRUE. (Why? Hint: The fraction does not have to be in lowest terms \dots)
\end{enumerate}

\vspace*{-18pt}$\;$
\end{solution}

\vspace*{-12pt}
\subsubsection{Subset exploration}\label{s: subset exploration}

\vspace*{-6pt}
\begin{task}
Is $A$ a subset of $B$ or vice versa? Complete this table with ``yes'' or ``no'' in each cell.

		\begin{tabular}{C{2in}||C{0.37in}|C{0.37in}|C{0.37in}|C{0.37in}|C{0.37in}|C{0.4in}|C{0.5in}}
			& $A\subset B$ & $A\subsetneq B$ & $A\supset B$ & $A\supsetneq B$ & $A = B$& $A\neq B$ 
			& \tiny{Neither is subset of the other} \\ \hline	
		$A= \tn{multiples of 3}$, 
		$B= \tn{multiples of 6}$ 
			& & & & & & & \\ \hline
		$A= \tn{multiples of 6}$, 
		$B= \tn{multiples of 9}$ 
			 & & & & & & & \\  \hline
		$A=\{ n^2 \st n\in \N, n>0 \}$, 
		$B=\{1+3+\dots +(2n+1) \st n\in \N\}$ 
			& & & & & & & \\ \hline	
		$A=\tn{functions of the form $x\mapsto 16^{ax}$}$, 
		$B=\tn{functions of the form $x\mapsto 2^{ax}$}$
			& & & & & & & \\  \hline
		\end{tabular}
\end{task}


\begin{bignote}[Teaching the subset exploration task]
Take this task one row at a time, emphasizing the mathematical practices of {\it clarifying the question} and then {\it finding and improving possible answers (aka conjecturing)}.  The goal is first to generate conjectures; then, after generating satisfying conjectures, to {\it prove (or disprove) the conjectures}.

Rows 1, 2, and 4 can be interpreted in different ways with different mathematical consequences. You may decide with your class to interpret:
	\begin{itemize*}
	\item Row 1, 2: Multiples should mean ``integer multiples''
	\item Row 4: $a$ should be considered in two cases, $a\in \Z$ and $a\in \Q$.
	\end{itemize*}

This means revising Row 4 and adding a Row 5 to the table:
		\begin{tabular}{C{2in}||C{0.5in}|C{0.5in}|C{0.5in}|C{0.5in}|C{1.2in}}
		$A=\tn{functions of the form $x\mapsto 16^{ax}$}$, 
		$B=\tn{functions of the form $x\mapsto 2^{ax}$}$, where $a\in \Z$
				& & & & & \\  \hline
		$A=\tn{functions of the form $x\mapsto 16^{ax}$}$, 
		$B=\tn{functions of the form $x\mapsto 2^{ax}$}$, where $a\in \Q$
				& & & & & \\  \hline				
		\end{tabular}

Row 3 may need clarification as far as set notation and what the ``\dots'' mean, but is otherwise precisely phrased.

Row 3 may be assigned as homework after discussing what there is to prove. 

This task is designed to show why equality of sets requires showing both that $A\subset B$ and $B\subset A$. Often we have found that students think of showing one direction as sufficient, and that this is reinforced by tasks where containment follows practically tautologically by definition. The examples in rows 3 and 4 do require inference from the definitions, not just the definitions themselves. 
\end{bignote}


{\it Clarifying the question.} We found that there were several ways that these questions needed to be clarified: In Row 1 and 2, we asked: what kind of multiples? We decided to consider only integer multiples. In Row 4, we asked: What is $a$? If $a\in \Z$, there are different consequences than when $a\in \Q$. We added this interpretation as a different row.

{\it Making conjectures/observations and improving them.}  Possible conjectures about this table include:

	\begin{itemize}
	\item (set of integer multiples of 3) $\supset$ (set of integer multiples of 6)
	\item (set of integer multiples of 3) $\supsetneq$ (set of integer multiples of 6)
	\item (set of integer multiples of 6) and (set of integer multiples of 9) are not subsets of each other
	\item (set of perfect squares) $=$ (set of sum of consecutive odd positive numbers)
	\item  When $a\in \Z$, (set of functions of the form $x\mapsto 16^{ax}$) $\subsetneq$ (set of functions of the form $x\mapsto 2^{ax}$)
	\item  When $a \in \Q$, (set of functions of the	form $x\mapsto 16^{ax}$) $=$ (set of functions of the form $x\mapsto 2^{ax}$)
	\end{itemize}

{\it Proving conjectures.}  
We will use the properties listed in Section \ref{s: properties of R and Z}. We also use the following proof structures.

{\bf Proof Structure: Showing one set is a subset or strict subset of another}.
	\begin{itemize}
	\item To show that $B\subset A$ requires showing: if $x\in B$, then $x\in A$.
	\item To show that $B\subsetneq A$ requires showing: (1) $B\subset A$; (2) there is an element of $A$ that is not in $B$.
	\end{itemize}

{\bf Proof Structure: Showing set equality}. 
	\begin{itemize}
	\item To show that $A=B$ requires showing: (1) $A\subset B$; (2) $B\subset A$.
	\end{itemize}
		
\begin{claim*}
If $A=\{ 3n \st n \in \Z\}$ and $B=\{ 6n \st n\in \Z\}$, then $B\subset A$.
\end{claim*}
\begin{proof}
Given $A=\{ 3n \st n \in \Z\}$ and $B=\{ 6n \st  n\in \Z\}$. Showing that $B\subset A$ means showing: if $x\in B$, then $x\in A$.

Given $x\in B$. Then:
	\begin{eqnarray*}
	x &=&6k, k\in \Z, \tn{ by definition of $B$} \\ 
	   &=& 3\cdot 2k \\ 
	   &=& 3n, n\in \Z, \tn{ because $2\in \Z, k\in \Z$, and $\Z$ is closed under multiplication}
	\end{eqnarray*}
Therefore $x$ satisfies set membership rules of $A$, implying $x\in A$. 

We have shown that if $x\in B$, then $x\in A$. Thus $B\subset A$, by definition of subset.
\end{proof}

\begin{claim*}
If $A=\{ 3n \st n \in \Z\}$ and $B=\{ 6n \st n\in \Z\}$, then $B\subsetneq A$.
\end{claim*}
\begin{proof}
Given $A=\{ 3n \st n \in \Z\}$ and $B=\{ 6n \st n\in \Z\}$. 

\begin{enumerate}
\item {\it Why $B\subset A$}: This means showing: if $x\in B$, then $x\in A$.

Given $x\in B$. Then:
	\begin{eqnarray*}
	x &=&6k, k\in \Z, \tn{ by definition of $B$} \\ 
	   &=& 3\cdot 2k \\ 
	   &=& 3n, n\in \Z, \tn{ because $2\in \Z, k\in \Z$, and $\Z$ is closed under multiplication}
	\end{eqnarray*}
Therefore $x$ satisfies set membership rules of $A$, implying $x\in A$. 

We have shown that if $x\in B$, then $x\in A$. Thus $B\subset A$, by definition of subset. \qedpart{1}

\item {\it Why there is an element of $A$ that is not in $B$.} We find an element of $A$ that is not in $B$.
If $x\in B$, then $x$ is an even number because if $x=6k$ for some $k\in\Z$, then $x$ as $x=2\cdot (3k)$. Closure of multiplication in $\Z$ implies $3k\in \Z$, so $x$ satisfies the definition of even number. 

However, some members of $A$ are odd numbers: $3, 9, 15, \dots $. 

Hence there are elements of $A$ that are not in $B$. \qedpart{2}

\end{enumerate}
Why this means $B\subsetneq A$: (1) and (2) show that $B$ and $A$ satisfy the definition of strict subset, and we have $B\subsetneq A$.
\end{proof}

\begin{claim*}
If $A=\{f:\R\to \R, x\mapsto 16^{ax} \st a\in \Q\}$ and $B=\{f:\R\to \R, x\mapsto 2^{ax} \st a\in \Q\}$, then $A=B$.
\end{claim*}
\begin{proof}[Sketch of proof] 
Given $A=\{f:\R\to \R, x\mapsto 16^{ax} \st a\in \Q\}$ and $B=\{f:\R\to \R, x\mapsto 2^{ax} \st a\in \Q\}$.

We outline the steps of the proof for you to fill in.

\begin{enumerate}
\item {\it Why $A\subset B$:} \qedpart{1}


\item {\it Why $B\subset A$:} \qedpart{2}
\end{enumerate}

Why the above means that $A=B$:
\end{proof}

\begin{bignote}[Modeling proof communication]
One way that we have found helpful for modeling proof communication is to write on the board a sample proof, that appears like what would want students to turn in the proof as homework. 

For instance, in writing up the proof that $B\subsetneq A$, you might write it on the board as shown in the sample handwritten version on p.~\pageref{s: sample handwritten proof}.  

To model proof structure/logic, you might start by writing the claim, followed by ``Proof'', ``given \dots'' and ``$\implies B\subsetneq A$'' (at the bottom of the board or wherever you expect the proof to end), and then remind the students that proving this claim means beginning with the givens and deducing the conclusion. We have found that doing this consistently for all proofs throughout the semester helps students understand the concept of proving a claim.

Then after completing the inside of the proof, you might annotate it with colored chalk as shown on p.~\pageref{s: good proof communication}, talking about the reasoning behind why these features are helpful.

You could then distribute the contents of pp.~\pageref{s: sample handwritten proof}-\pageref{s: good proof communication} and ask the students for comments or questions about communicating proof.
\end{bignote}

%%%%%%%%%%%%%%%%%%%%%
\subsection{Mathematical statements and their negations}
\label{s: statements and negations}

{\bf Logical notation}

\begin{tabular}{L{1in}L{5in}}
$P(x)$ & A proposition about the variable $x$; may be true or false depending on $x$ \\

$\neg P(x)$ & The negation of $P(x)$ \\

$\forall x, P(x)$ & The proposition ``For all values of $x$, $P(x)$ is true." \\

$\exists x: P(x)$ & The proposition ``There exists a value of $x$ such that $P(x)$ is true." \\

$\forall x, P(x) \Rightarrow Q(x)$ & The proposition ``For all values of $x$, if $P(x)$ is true then $Q(x)$ is true."\\

$\forall x, P(x) \Leftrightarrow Q(x)$ & The proposition ``For all values of $x$, $P(x)$ is true if and only if $Q(x)$ is true." 
\end{tabular}

\begin{task}
\begin{enumerate}
\item For each of the following statements, figure out what it means, and decide whether it is true, false, or neither. 
	 \begin{enumerate}
	 \item $\forall x\in \R, \exists y\in \R \st y+x\in\{z\in \Z \st z > 0\}$
	  \item $\forall x\in \Z, \exists y\in \Z \st y+x\notin \Z$
	 \item$\forall g:\R\to\R, x\mapsto 2^{ax},  \exists h:\R\to\R, x\mapsto 4^{bx} \st \forall x\in \R, g(x)=h(x)$
	  \item$\forall g:\R\to\R, x\mapsto 4^{ax},  \exists h:\R\to\R, x\mapsto 2^{bx} \st \forall x\in \R, g(x)=h(x)$
	 \end{enumerate}
\item Negate the following statements without using any negative words (``no'', ``not'', ''neither \dots nor'', etc.) Try to make your negation sound as much like normal English as possible. 
	\begin{enumerate}
	\item Every word on this page starts with a consonant and ends with a vowel.
	\item The set $A$ is equal to the set $B$.
	\item There is a book on this shelf in which every page has a word that starts and ends with a vowel.
	\item The set $A$ is a strict subset of the set $B$.
	\end{enumerate}
\end{enumerate}
\end{task}

\smallnote{There is typically only time to do one or two of each task, with the rest assigned for homework.  For (1), we recommend (a) or (b), and then as time allows, (c) or (d). For (2) We recommend doing least one word negation ((a) or (c)) in class, and then as time allows, one negation having to do with concepts of sets ((b) or (d)).}

\begin{solution}({\it Partial})
\begin{enumerate}
\item  \begin{enumerate}
	\item For each real number $x$, there is a real number $y$ so that $x+y$ is a positive integer.  TRUE.  
	
{\it Reasoning:} If $x\in \R$, then take $y=1-x$. Then $x+y=1$, which is a positive integer. Or take any positive integer $n$ and take $y=n-x$.

	\item  What it means: \dots (fill in the rest). FALSE.
	
	{\it Reasoning:}  (Why?)
	\item For each function $g(x)=2^{ax}$ there is a function $f(x)=4^{bx}$ so that $g(x)=h(x)$ on every possible real value of $x$. NEITHER. 
	
	{\it Reasoning:} The truth of this statement depends on the possible values of $a$ and $b$. If $a$ and $b$ must be integers, then there are some $a$ where $2^{ax}$ cannot equal $4^{bx}$. (All odd integers.) If $a$ and $b$ are rational or real, then for each $a$, we can take $b=\frac{a}{2}$, and then $2^{ax}=4^{bx}$
	\item What it means: \dots (fill in the rest). TRUE. 
	
	{\it Reasoning:}  (Why?)
	\end{enumerate}
\item 
	\begin{enumerate}
	\item THERE IS a word on this page that starts with a vowel OR ends with a consonant.
	\item The set $A$ has at least one element that is not in $B$ OR the set $B$ has at least one element that is not in $A$.
	\item EVERY book on this shelf (\dots fill in the rest)
	\item The set $A$ equals $B$ OR (\dots fill in the rest)
	\end{enumerate}
\end{enumerate}
\vspace*{-18pt}$\;$
\end{solution}

\subsection{Back to the opening inquiry}

We have now spent some time discussing set notation and logical notation.

We began this class considering ``parents'' of numbers. We conjectured that:

\begin{quote}
If a number has least three different factors (that are not itself or 1), then it has more than one pair of parents. There are no other numbers with more than one pair of parents.
\end{quote}

One way of saying ``factors of a number that are not itself or 1'' is to say ``nontrivial factors''.

{\bf Applying set notation.} Using set notation, we can interpret the conjecture as saying:

\begin{conjecture}[Number parent conjecture, take 1]\label{c: number parent 1}
Let \begin{eqnarray*}
	S&=&\{ n\in \N\st n\tn{ has at least three different non-trivial factors}\} \\ 
	T&=&\{ n\in \N \st n \tn{ has more than one pair of parents}\} 
	\end{eqnarray*}
Then $S=T$.	
\end{conjecture}

\begin{task}
How does this way of phrasing the conjecture match up with the original way? 
	\begin{itemize}
	\item Look up the definition of set equality. What does $S=T$ mean by definition of set equality?
	\item Which part of set equality implies the first sentence (``If a number has least three different nontrivial factors, then it has more than one pair of parents.'')? 
	\item Which part of set quality implies the second sentence? (``There are no other numbers with more than one pair of parents'') 
	\end{itemize}
\end{task}

\begin{solution}
By definition, $S=T$ means $S\subset T$ and $T\subset S$.

$S\subset T$ implies that ``If a number has least three different nontrivial factors, then it has more than one pair of parents.''

$T\subset S$ means that ``there are no other numbers with more than one pair of parents.'' 
\end{solution}

\bigskip
{\bf Applying logical notation.} There is another mathematically equivalent way of saying the conjecture using the logical notation we developed. 

\begin{conjecture}[Number parent conjecture, take 2]\label{c: number parent 2}
$\forall n\in \N$, $n$ has more than one pair of parents $\iff$ $n$ has at least three nontrivial factors.
\end{conjecture}

\begin{task}
How does this way of phrasing the conjecture match up with the original way? 
	\begin{itemize}
	\item What does ``if and only if'' mean? 
	\item Which part of ``iff'' implies the first sentence (``If a number has least three different nontrivial factors, then it has more than one pair of parents.'')?   (An abbreviation for ``if and only if'' is ``iff'')
	\item Which part of ``iff'' implies the second sentence? (``There are no other numbers with more than one pair of parents'') 
	\end{itemize}
\end{task}


\begin{solution}
By definition, $P$ iff $Q$ means that both $P\implies Q$ and $Q\implies P$ are true statements.

Given $n\in \N$, let the statement $P$ be ``$n$ has more than one pair of parents'', and the statement $Q$ be statement ``$n$ has at least three nontrivial factors''.

$Q\implies P$ being true implies that ``if a number has least three different nontrivial factors, then it has more than one pair of parents.'' 

$P\implies Q$ being true implies that ``there are no other numbers with more than one pair of parents.'' 
\end{solution}

\bigskip


(The following is stated in two equivalent ways) 
\begin{proposition}[Number parent proposition]\label{p: number parent}
\hspace*{-6pt}\begin{tabular}{L{3.5in}|L{2.8in}}
If $S=\{ n\in \N\st n\tn{ \it has at least three different non-trivial factors}\}$ and $T=\{ n\in \N \st n \tn{ \it has more than one pair of parents}\}$, then $S=T$.
& 
For all $n\in \N$, $n$ has more than one pair of parents if and only if $n$ has at least three nontrivial factors.
\end{tabular}
\end{proposition}

\begin{proof}
Given $S=\{ n\in \N\st n\tn{ \it has at least three different non-trivial factors}\}$ and $T=\{ n\in \N \st n \tn{ \it has more than one pair of parents}\}$.
\begin{enumerate}
\item {\it Why $S\subset T$:}\todo{Ch 1: Clean up this proof part 1. It is correct but confusing.} 
Let $n\in S$. Then there exist distinct $a, b, c \in \N$ such that $a\divides n$, $b\divides n$, and $c\divides n$. Either each of these are paired with another one of $a,b,c$ to be a pair of parents of $n$ or they are not. If they are not paired with any of each other, then $n$ has at least three pairs of parents, which is more than one. If one of them is paired with another, there is still a third factor that cannot be paired with the other two (because they are already paired). So it is part of a second pair of parents. Thus $n\in T$.


We have shown that if $n\in S$, then $n\in T$. By definition of subset, this shows $S\subset T$. \qedpart{1}

\item {\it Why $T\subset S$:} 
Let $n\in T$. Then there exist at least two pairs $a, a'\in \N$ and $b, b'\in \N$ such that $aa'=n$ and $b,b'=n$, and $\{a,a'\} \neq \{b, b'\}$.

If $a\neq a'$ and $b\neq b'$, then $n$ has at least four factors, so $n\in S$.

It may be true that $a=a'$ or $b=b'$. If $a=a'$, though, then $n$ is a perfect square and $b\neq b'$, since there is only one positive square root possible for every $n$. Similarly, if $b=b'$, then $a\neq a'$. In either case, $n$ has at least three factors (either $a, b, b'$ or $a, a', b$), so $n\in S$.

We have shown that if $n\in T$, then $n\in S$. By definition of subset, this shows $T\subset S$. \qedpart{2}
\end{enumerate}
We have shown that $S\subset T$ and $T\subset S$. By definition of set equality, we have shown $S=T$.
\end{proof}

%%%%%%%%%%%%%%%%%%%%%
\subsection{Summary of mathematical practices}
 
 
\subsubsection*{Clarifying the question}
 \begin{itemize}
 \item Make the best sense as you can of the question with what is available.
 \item Identify what is unambiguous, and then identify what is ambiguous.
 \item For the ambiguous parts, play around with different possibilities to see what is the most mathematically interesting possibility. Sometimes you may find that there are multiple interesting mathematical possibilities.
 \end{itemize}
\subsubsection*{Conjecturing and Claim making}

\begin{itemize}
\item Think of claims as an ``I bet'' statement. 

If you're the arbitrator for a bet between people, you would want to make absolutely sure that everyone knows exactly what the statement means, and also that everyone would agree on what evidence would count as showing the bet is true or not true! 

The same is true about mathematical statements. A mathematical statement needs to be crystal clear about what it means.

\item Mathematical claims should either be true or false; if they ``depend'' on something, this means that there is often a better claim that can be made. 

\item The more general a claim, the better it is. 

For instance, ``12 has more than one pair of parents'' is a true claim, but a better claim is ``All numbers with at least three distinct factors have more than one pair of parents'' is an even better claim.

\item The more ``directions'' a claim addresses, the better it is. 

For instance, ``All numbers with at least three distinct factors have more than one pair of parents'' is a true claim, but ``A number has more than one pair of parents if and only if it has at least three distinct factors'' goes even further to understanding the situation.
\end{itemize}

\subsubsection*{Exploring math: Our expectations}

\begin{itemize}
\item Make claims.

\item Try to prove them.

\item If you get stuck, consider the negations of the claim.

\item Try to prove those.

\item Consider the ``opposite direction'' claim. (The ``converse'' of the claim.)  

\item Try to prove those.

\item Aim to make the most satisfying claims possible.

\item Rewrite, rewrite, rewrite! Use the rewriting process to help things get clear for yourself, your future students, and your future self, and your peers.

\end{itemize}





\newpage

\begin{bignote}[Things to keep in mind on the first day]
This first lesson is an important place to do what can be called ``setting norms and expectations''. What this means is communicating to prospective teachers, both implicitly and explicitly, what productive conversation, exploration, questioning, and justification look and feel like. For instance, you may want to teach a class where:
	\begin{itemize*}
	\item {\it Students embrace learning from their own individual and each others' work} -- they view their own mistakes courageously and with an open mind; they accept that making errors and learning from them is a natural part of the mathematical process; they  recognize what is worthwhile about others' reasoning and what needs further thought, and they do so constructively; they celebrate others' ideas. 
	\item {\it Students view mathematical reasoning as the ultimate mathematical authority} -- they have faith in their ability to learn to reason mathematically; they come back to the mathematics rather than to a perceived authority figure such as an instructor or a ``smart'' student to figure out what works; they seek precision in language while also understanding that going from informal language to precise language may take some time, may not happen right away, but is a valuable goal.
	\item {\it Students persist in seeking mathematical questions and answers} -- they accept that setbacks are an important part of learning; they can work for an extended amount of time on one problem in productive ways; they celebrate when they do come to an understanding of a mathematical idea, especially one that is hard-won.
	\end{itemize*}

If these are values that you see a productive class expressing, you can do much to foster these values beginning the first day. There are many different things you can do and say, and certainly different things may work better or worse for different instructors and different students. Here are some examples of things to do and say that have helped previous \MODULES instructors:
	\begin{itemize*}
	\item {\it Praise thoughtful errors.} It's easy to spot ``right'' answers and there can be a temptation to run with the way that some students have found exactly the ``right'' way to approach a problem. There is also a temptation to respond to ``wrong'' answers with saying matter-of-factly, ``Not quite; what did others get?'' But if you respond in these ways, and exclusively so as your form of interacting with students about their thinking, what message does that send to students about the role of mistakes in the process of working through mathematics? It may well send the message that the best work is the work that is correct the first try, or worse, that the most worthy students are those that only do correct mathematics and make no mistakes. Instead, an alternative approach is to look for thoughtful errors -- the kind of thinking that is ultimately mathematically incorrect for some reason, but where thinking through the mistake has the potential to really get at something fundamental about the mathematics at hand or in the future. Moves that you can make to acknowledge thoughtful errors might include:
		\begin{itemize*}
		\item ``I am so glad that you brought that up, [student name]. Did everyone understand what [student name] said? Can someone say in their own words what they understand of [student name]'s reasoning?'' [If someone raises their hand to counter this idea] ''Right now we're not interested in whether we agree or disagree with [student name], we are trying to understand what [student name] is thinking. What might they thinking? Why does it make sense to do this?''
		\item ``Let's see what happens when we follow this reasoning.''
		\item ``We just learned a really important lesson about doing mathematics because of this reasoning. Thank you, [student name], for sharing your idea. This was incredibly helpful. Let's remember the lesson we learned throughout today and also as we move forward in this class.''  
		\end{itemize*}
	\item {\it Do not make a big deal when students get a correct answer right away.  Focus on the process of getting to the answer, and on understanding the answer, rather than the answer itself.}  The Fields Medalist William Thurston (1994) observed of his colleagues, ``I thought that what they sought was a collection of powerful proven theorems that might be applied to answer further mathematical questions. But that's only one part of the story. More than the knowledge,
people want {\it personal understanding.}'' (p.~51, emphasis by Thurston). The same is true of students, or at least we would like to be a truth about students. Moves that emphasize understanding over the answer might include:
		\begin{itemize*}
		\item (As a matter-of-fact first reaction to the correct answer) ``You answered $X$. What was your reasoning for that answer?''  \dots ``What do others think of this reasoning?'' 
		\item ``[Student name] arrived at the solution $X$, and just shared their reasoning. Did anyone else arrive at this solution? Did you have similar reasoning or different reasoning?'' 
		\item ``Let's think back on why this answer makes sense.''
		\end{itemize*} 
	\item {\it Relinquish your authority to the students and the mathematics.} A common question instructors hear is, ``What do you want?'' or ``Is this what you are looking for?'' Sometimes the answer to these questions really does rest with you, the instructor -- especially if it is about specific directions that you are setting for your students that can't be derived from mathematical reasoning. However, answering these questions from your authority as an instructor can be less useful if the questions are actually about mathematical reasoning, for instance, if the question is about whether a proof or solution is correct. In these cases, it can be more productive to return the responsibility of these questions to the students and the mathematics:
		\begin{itemize*}
		\item ``Can you tell me more about how you arrived at this?''
		\item ``Tell me about what's here.''
		\item ``How does this help to give a solution to the question we are working on?'' 
		\item ``How complete do you think it is?'' \dots ``What about your work are you sure about, and what are you less sure about?''
		\end{itemize*}
	
	\item {\it Give students ways to work constructively with each other.} Working with each other on mathematics is not necessarily a natural skill; it is a learned skill. Help your students find ways to talk to each other about their thinking. While students are working, stir the pot (meaning, find ways to provoke productive disagreement and/or discussion). 
		\begin{itemize*}
		\item ``I see that [student A] and [student B] have different answers. It looks like you have something to resolve. [Student A] and [Student B], will you share how you did your work with each other and figure out what's really going on?'' 
		\item ``I see that [student A] and [student B] have arrived at the same answer, but it looks like you've done it in different ways. Will you compare what you've done and see how they match each other or do not?'' 
		\item ``It looks like [student A] has drawn a graph and [student B] has used mostly equations. Are you thinking about the same thing? Will you talk to each other about how your thinking matches up or not?'' 
		\item ``It looks like [Student A] worked on [Case 1] whereas [student B] worked on [Case 2]. Are there more cases to consider? Are both cases necessary? You should talk to each other to figure this out.''
		\end{itemize*}
	\end{itemize*}
\end{bignote}

%%%%%%%%%%%%%%%%%%%%%%%%%%%%%%%%%%%%%%%%
%%%%%%%%% 1 IN-CLASS RESOURCES %%%%%%%%%%%%%%%%%%%%
%%%%%%%%%%%%%%%%%%%%%%%%%%%%%%%%%%%%%%%%

\newpage \subsection{In-Class Resources}  

%%%%%%%% 1 OPENING INQUIRY %%%%%%
\subsubsection{Opening Inquiry}

Two numbers are parents of a child if the child is their product. 

A child cannot be its own parent.

Which numbers have more than one pair of parents?

\begin{center}
\begin{tabular}{C{1.5in}|C{1.5in}}
Child & Parents \\
\hline 
 6 & 2, 3 \\  
 4 & ?? \\  
 12 & 4, 3 \\ 
 12 & 2, 6 
\end{tabular}
\end{center}

\vfill
\hrule

Clarifying what it means to be a pair of parents:

\vspace*{1in}
Notes on finding and improving answers to mathematical questions:
\vspace*{1.5in}
%%%%%%%% 1 GETTING TO KNOW SET AND LOGICAL NOTATION %%%%%%
\newpage
\subsubsection{Getting to know set notation}

\begin{enumerate}
\item Let $A = \{1,2,\{3,4\}, \{5\}\}$.  
Decide whether each of the following statements is true or false: \\
	(\emph{Hint:} There are exactly six true statements.)
		\begin{align*}
			1 \in A,  && \{1, 2\} \in A, && \{1, 2\} \subseteq A, && \emptyset \in A,\\
			3 \in A,  && \{3, 4\} \in A, && \{3, 4\} \subseteq A, && \emptyset \subseteq  A,\\
			\{1\} \in A, && \{1\} \subseteq A, && \{5\} \in A, && \{5\} \subseteq A.
		\end{align*}
\item True or false? ``All students in this class who are under 5 years old are also over 100 years old.''
\end{enumerate}
\begin{enumerate}[resume] 
\item Let $S=\{ x\in \Q \st x \tn{ can be written as a fraction with denominator 2 and } |x|<2 \}$. 

True or false? \quad\quad $0.5\in S$, \quad\quad $3.5 \in S$, \quad\quad $0.25\in S$,\quad\quad $1\in S$.
\end{enumerate}
\vfill


\hrule

\subsubsection{Getting to know logical notation}

\begin{enumerate}
\item For each of the following statements, figure out what it means, and decide whether it is true, false, or neither. 
	 \begin{enumerate}
	 \item $\forall x\in \R, \exists y\in \R \st y+x\in\{z\in \Z \st z > 0\}$
	 \item $\forall x\in \Z, \exists y\in \Z \st y+x\notin \Z$
	 %\item$\forall g:\R\to\R, x\mapsto 2^{ax},  \exists h:\R\to\R, x\mapsto 4^{bx} \st \forall x\in \R, g(x)=h(x)$
	  %\item$\forall g:\R\to\R, x\mapsto 4^{ax},  \exists h:\R\to\R, x\mapsto 2^{bx} \st \forall x\in \R, g(x)=h(x)$
	 \end{enumerate}
\item Negate the following statements without using any negative words (``no'', ``not'', ''neither \dots nor'', etc.) Try to make your negation sound as much like normal English as possible. 
	\begin{enumerate}
	\item Every word on this page starts with a consonant and ends with a vowel.
	\item The set $A$ is equal to the set $B$.
	%\item There is a book on this shelf in which every page has a word that starts and ends with a vowel.
	%\item The set $A$ is a strict subset of the set $B$.
	\end{enumerate}
\end{enumerate}

\vfill


%%%%%%%% 1 SUBSET EXPLORATION %%%%%%
\newpage
\subsubsection{Subset exploration}
Is $A$ a subset of $B$ or vice versa? Complete this table with ``yes'' or ``no'' in each cell.

		\begin{tabular}{C{2in}||C{0.37in}|C{0.37in}|C{0.37in}|C{0.37in}|C{0.37in}|C{0.4in}|C{0.5in}}
			& $A\subset B$ & $A\subsetneq B$ & $A\supset B$ & $A\supsetneq B$ & $A = B$& $A\neq B$ 
			& Neither is subset of the other \\ \hline	
		$A= \tn{multiples of 3}$, 
		$B= \tn{multiples of 6}$ 
			& & & & & & & \\ \hline
		$A= \tn{multiples of 6}$, 
		$B= \tn{multiples of 9}$ 
			 & & & & & & & \\  \hline
		$A=\{ n^2 | n\in \N, n>0 \}$, 
		$B=\{1+3+\dots +(2n+1) | n\in \N\}$ 
			& & & & & & & \\ \hline	
		$A=\tn{functions of the form $x\mapsto 16^{ax}$}$, 
		$B=\tn{functions of the form $x\mapsto 2^{ax}$}$
			& & & & & & & \\  \hline
		\end{tabular}



%%%%%%%% 1 BACK TO OPENING INQUIRY %%%%%%%%%%%
\newpage
\subsubsection{Back to the opening inquiry}
We began this class considering ``parents'' of numbers. We conjectured that:

\begin{mdframed}
\vspace*{1in}
\end{mdframed}

{\bf Applying set notation.} Using set notation, we can interpret the conjecture as saying:
\begin{mdframed}
\vspace*{1in}
\end{mdframed}

How does this way of phrasing the conjecture match up with the original way? 
	\begin{itemize}
	\item Look up the definition of set equality. What does $S=T$ mean by definition of set equality?
	\item Which part of set equality implies the first sentence (``If a number has least three different nontrivial factors, then it has more than one pair of parents.'')? 
	\item Which part of set quality implies the second sentence? (``There are no other numbers with more than one pair of parents'') 
	\end{itemize}

\vfill

{\bf Applying logical notation.} There is another mathematically equivalent way of saying the conjecture using the logical notation we developed:
\begin{mdframed}
\vspace*{1in}
\end{mdframed}
How does this way of phrasing the conjecture match up with the original way? 
	\begin{itemize}
	\item What does ``if and only if'' mean? 
	\item Which part of ``iff'' implies the first sentence (``If a number has least three different nontrivial factors, then it has more than one pair of parents.'')?   (An abbreviation for ``if and only if'' is ``iff'')
	\item Which part of ``iff'' implies the second sentence? (``There are no other numbers with more than one pair of parents'') 
	\end{itemize}
	
\vfill



%%%%%%%%% 1 HOMEWORK %%%%%%%%%%%%%%%%%%%%
\newpage \subsection{Homework}  


\begin{bignote}
For homework, you may want to make sure to assign \ref{p: parent relation intro} and \ref{p: cart prod}. These are used in Section \ref{s: relations} and beyond. 

For the remaining problems, it is not recommended to assign every piece of all problems; the intent is for you to pick and choose one or two per problem as you see fit based on what you know or want to know about your students and the intended purpose of the problems.

The purpose of the problems is as follows:
	\begin{enumerate}
	\setcounter{enumi}{-1}
	\item Making sure that students get into the habit of reading the notes. (You may want to assign this as a google form or online form so that you can read student responses before the next time you see students. Then you can address questions immediately.)
	\item Problems about proving set membership or set non-membership.
	\item Problems about establishing a set as a subset or as a strict subset or another set; establishing set equality.
	\item Proof comprehension: Helping students dive into the practices of understanding a proof. This is based on the framework for proof comprehension proposed by Mejia-Ramos, Fuller, Weber, Rhoads, and Samkoff (2012)\footnote{\footnotesize{Mejia-Ramos, J.P., Fuller, E., Weber, K., Rhoads, K., \& Samkoff, A.~(2012). An assessment model for proof comprehension in undergraduate mathematics. {\it Educational Studies in Mathematics, 79}: 3-18. doi: 10.1007/s10649-011-9349-7.}}: 	
		\begin{itemize}
		\item assessing local comprehension of proof: meanings of terms and statements; meaning of individual statements;
		\item logical status of statements and proof framework;
		\item chaining statements of proof together; and 
		\item holistic understanding of proof.
		\end{itemize}
	\item Introducing the parent relation and one representation of relations.
	\item Introducing Cartesian product in a way that is consistent with how relations are first defined in the next chapter.
	\end{enumerate}
\end{bignote}

\begin{enumerate}
\setcounter{enumi}{-1}
\item % Locating relevant parts of the text
\label{p: set ideas}
	In this chapter, we learned about:
		\begin{itemize}
		\item Showing that an element is or is not a member of a particular set
		\item Showing that a set is a subset or strict subset of another set
		\item The definition of set equality and how to show that two sets are equal
		\item Number parents and children, in particular, that if $S=\{n\in\N \st n \tn{ has at least three different non-trivial factors}\}$ and $T=\{n\in \N\st n \tn{ has more than one pair of parents}\}$, then $S=T$.
		\end{itemize}
	For each of these ideas:
		\begin{enumerate}
		\item Where in the text are these ideas located?
		\item Review this section of the text.	What definitions are most relevant? How do the examples use these definitions?	
		\item What questions or comments do you have about the ideas in this section?
		\end{enumerate}
\item % Problems about proving set membership or set non-membership
Let $S=\{x\in \Q\st x \tn{ can be written as a fraction with denominator 2 and } |x|<2\}$.

Let $B = \{1+3+\dots + (2n+1) \st n\in \N\}$.

Let $C$ be the set of functions of the form $x \mapsto 27^{ax}$. 

Let $D$ be the set of numbers with more than 5 parents.

	\begin{enumerate}
	\item Is ``$0.25\in S$'' a true statement? 
	\item Is ``$1\in S$'' a true statement?
	\item Is 25 an element of $B$? 
	\item Is 24 an element of $B$? 
	\item When $a\in \Z$, is $x\mapsto 3^{5x}$ contained in $C$?
	\item When $a\in \Q$, is $x\mapsto 3^{5x}$ contained in $C$?
	\item Find a number that is a member of both $B$ and $D$.
	\end{enumerate}

After reading this problem, think about: What mathematical idea(s) listed in Problem \ref{p: set ideas} does this problem provide opportunities to understand?  This is something helpful to think about for all the problems.

In your responses, articulate clearly:
	\begin{itemize}
	\item Whether you are showing that the element is or is not contained in the set;  
	\item How you used the definition of set membership to determine ``yes'' or ``no''; and
	\item Any definitions or ideas that you need in your reasoning.
	\end{itemize}
	
\item %Problems about establishing as a subset or as a strict subset
	% Problems about establishing set equality 

\begin{enumerate}
	\item Complete this table with ``True'' or ``False''.

		\hspace*{-24pt} \begin{tabular}{C{2.1in}||C{0.37in}|C{0.37in}|C{0.37in}|C{0.37in}|C{0.37in}|C{0.4in}|C{0.5in}}
			& $A\subset B$ & $A\subsetneq B$ & $A\supset B$ & $A\supsetneq B$ & $A = B$& $A\neq B$ 
			& \tiny{Neither is subset of the other} \\ \hline	
		$A= \tn{integer multiples of 14}$, 
		$B= \tn{integer multiples of 21}$ 
			 & & & & & & & \\  \hline
		$A=\{ n^2 \st n\in \N, n>0 \}$, 
		$B=\{1+3+\dots +(2n+1) \st n\in \N\}$ 
			& & & & & & & \\ \hline	
		$A=\tn{functions of the form $x\mapsto 27^{ax}$}$ 
		$B=\tn{functions of the form $x\mapsto 3^{ax}$}$, $a\in \Z$
			& & & & & & & \\  \hline
		$A=\tn{functions of the form $x\mapsto 27^{ax}$}$ 
		$B=\tn{functions of the form $x\mapsto 3^{ax}$}$, $a\in \Q$
			& & & & & & & \\  \hline
		\end{tabular}
	\item Prove your ``true'' responses to Row 1. 
	\item Prove your ``true'' responses to Row 2. 
	\item Prove your ``true'' responses to Row 3. 
	\item Prove your ``true'' responses to Row 4. 
	\end{enumerate}

\item % Proof comprehension question about parent relation proof

Read over the section containing Conjectures \ref{c: number parent 1} and \ref{c: number parent 2} and Proposition \ref{p: number parent}.  
	\begin{enumerate}
	% assessing local comprehension of proof
	% meanings of terms and statements; meaning of individual statements
	\item 	
		Write the following statements in your own words: Conjecture \ref{c: number parent 1} and Conjecture \ref{c: number parent 2}. Then explain why they are mathematically equivalent.
	\item \label{p: n a b} In the proof of Proposition \ref{p: number parent}, in the section on showing $T\subset S$, the proof states, ``Let $n\in T.$ Then there exist at least two pairs $a,a'\in \N$ and $b, b'\in \N$ such that $\{a,a'\}\neq \{b, b'\}$.'' Why is the implication (``there exist at least two pairs \dots'') true? 
	\item  In this section, the proof states, 
	
	``If $a\neq a'$ and $b\neq b'$, then $n$ has at least four factors, so $n\in S$. 
	
	It may be true that $a=a'$ or $b=b'$. If $a=a'$, though, then $n$ is a perfect square and $b\neq b'$, since there is only one positive square root possible for every $n$. Similarly, if $b=b'$, then $a\neq a'$. In either case, $n$ has at least three factors (either $a, b, b'$ or $a, a', b$), so $n\in S$.''
	
		\begin{enumerate}
		\item What is the negation of the statement ``$a=a'$ or $b=b'$''?
		\item \label{p: n a b true} Give an example of $n$ such that ``$a=a'$ or $b=b'$'' is true. 
		\item \label{p: n a b false} Give an example of $n$ such that ``$a=a'$ or $b=b'$'' is false. 
		\item Explain the meaning of the quoted passage using the examples you gave in (\ref{p: n a b true}) and (\ref{p: n a b false}).
		\end{enumerate}
	% Logical status of statements and proof framework
	\item What is the logical reason for needing to show both that  $S\subset T$ and $T\subset S$ to establish Proposition \ref{p: number parent}?
	% Chaining statements of proof together
	\item Why does the truth of the passage in \ref{p: n a b} mean that that every $n\in T$ is also a member of $S$? 
	% Assessing holistic comprehension of proof
	\item Walk through the steps of the entire proof of Proposition \ref{p: number parent} using the example of $n=12$.
	\item Why does the first line of each part of the proof (why $S\subset T$ and why $T\subset S$) not apply to the case $n=6$?
	\end{enumerate}

\item \label{p: parent relation intro}
% introduces parent relation
The diagram below shows $P$, a collection of arrows from a natural number to its parents.   Some arrows below have been filled in, for example $P$ assigns $6$ to $2$ and $3$, and assigns $12$ to $2, 3, 4, 6$.  Draw in three more arrows from a natural number to its parents.

\begin{center}
\begin{tikzpicture}
\def\pt{circle (0.05)}
\foreach \x in {0, 1, 2, 3, 4, 5, 6, 7, 8, 9, 10, 11, 12, 13, 14}
	{\filldraw(0,-0.5*\x) node[left]{$\x$} \pt;
	\filldraw(5,-0.5*\x) node[right]{$\x$} \pt;
	}
	\draw 
		(2.5, 1) node[]{$P$}
		(0,0.7) node[] {$\N$}
		(5,0.7) node[] {$\N$}
		(0, -0.5*15) node[]{$\vdots$}
		(5, -0.5*15) node[]{$\vdots$};
	\draw[->=stealth, yscale=-0.5] (0,6) -- (4.94, 1.9);
	\draw[->, yscale=-0.5] (0,6) -- (4.94, 2.9);
	\draw[->, yscale=-0.5] (0, 12)-- (4.94, 2.1);
	\draw[->, yscale=-0.5] (0, 12) -- (4.94, 3.1);
	\draw[->, yscale=-0.5] (0, 12) -- (4.94, 4);
	\draw[->, yscale=-0.5] (0, 12) -- (4.94, 6);
\draw[xshift=-1cm] (0,0) arc (180:0:1) (0,-7) arc (180:360:1) (0,0)-- (0,-7) (2,0) -- (2,-7); 
\draw[xshift=4cm] (0,0) arc (180:0:1) (0,-7) arc (180:360:1) (0,0)-- (0,-7) (2,0) -- (2,-7); 
\end{tikzpicture}
\end{center}


\item  % Introduce Cartesian product $D\times R$. 
\label{p: cart prod}
Let $D$ and $R$ be sets. The \emph{Cartesian product} of $D$ and $R$, which we will work with more next time, is defined as the set of ordered pairs $\{(x,y) \st x\in D, y\in R\}$. For example, if $D=\{1, 2, 3\}$ and $R=\{4, 5\}$, then $D\times R$ is the set $\{(1, 4),(1,5), (1, 6), (2, 4), (2, 5), (2, 6)\}$. One way to think about it is that it is all the pairs you can get by drawing all possible arrows from elements of $D$ to elements of $R$ when you draw the elements lined up in parallel to each other:

	\begin{center}
	\begin{tikzpicture}
	\def\pt{circle (0.05)}
	\foreach \x in {1, 2, 3}
	{\filldraw(0,-0.5*\x) node[left]{$\x$} \pt;
	\draw[->] (0,-0.5*\x) -- (4.94, -0.5-0.03*\x+0.03);
	\draw[->] (0,-0.5*\x) -- (4.94, -1.5-0.03*\x+0.03);
	}
	\foreach \y in {4, 5}
	{\filldraw(5,3.5-\y) node[right]{$\y$} \pt;}
	\end{tikzpicture}
	\end{center}
	
Each arrow represents an ordered pair, with the starting point of the arrow being the first coordinate of the ordered pair and the ending point of the arrow being the second coordinate of the ordered pair.

Let $A=\{5, 6, 10\}$ and $B=\{-1, -2, -3\}$. Draw an arrow representation of the Cartesian product $A\times B$ and list its elements as ordered pairs.
\end{enumerate}



%%%%%%%%%%%%%%%%%%%%%%%%%%%%%%%%
%%%%%%%%%%%%%%%%%%%%%%%%%%%%%%%% 
%%%%%%%%%%%%%%%%%%%%%%%%%%%%%%%% 	
%%%%%% PART II %%%%%%%%%%%%%%%%%%%%%	%
%%%%%%%%%%%%%%%%%%%%%%%%%%%%%%%% 
%%%%%%%%%%%%%%%%%%%%%%%%%%%%%%%% 	
%%%%%%%%%%%%%%%%%%%%%%%%%%%%%%%%

\newpage \part{Relations and Functions} 
%%%%%%%%%%%%%%%%%%%%%%%%%%%%%%%%
%%%%%%%%%%%%%%%%%%%%%%%%%%%%%%%% 	
%%%%%% LESSON 2 %%%%%%%%%%%%%%%%%%%%	
%%%%%%%%%%%%%%%%%%%%%%%%%%%%%%%%
%%%%%%%%%%%%%%%%%%%%%%%%%%%%%%%%
 
\section{Relations (Week 2) (Length: \about 3 hours)}\label{s: relations}  % 
%%%%%%%%% 2 OVERVIEW  %%%%%%%%%%%%%%%%%%%%
\subsection{Overview}


\begin{tabular}{L{6.5in}} 
{\bf Content} \\ \hline \parskip4pt
\emph{Cartesian product} of two sets $A$ and $B$, denoted $A\times B$, defined as the set of ordered pairs $\{ (a,b) \st a\in A, b\in B \}$.

\emph{Relation} from a set $D$ to set $C$, defined from three different perspectives: the ``middle school'', ``high school'', and ``university''; and their mathematically equivalence. 
\begin{itemize*}
\item The ``middle school'' version is described in terms of a set of arrows between an input and output space.
\item The ``high school'' version formalizes arrows to assignments. 
\item The ``university'' version defines a relation as a subset of the Cartesian product $D\times R$. 
\end{itemize*}
We call these definitions the middle school, high school, and university versions to refer to when they most likely arise. 

\emph{Inverse of a relation}, defined from these three perspectives; their mathematical equivalence.

\emph{Composition of relations} $P:D\to D$ then $Q:D\to D$, defined 
as the relation that assigns $x$ to $z$ whenever there is a $y\in D$ such that $P$ assigns $x\mapsto y$ and $Q$ assigns $y\mapsto z$. (See p.~\pageref{s: composition} for why we only consider the case $P:D\to D$ and $Q:D\to D$.)

\emph{Graph of a (real) relation}, defined as the set of points $(x,y)\in \R^2$ such that the relation assigns $x$ to $y$.

\emph{Graph of an (real) equation} in variables $x$ and $y$, defined as the set of points $(x,y)\in \R^2$ such that evaluating the equation at $x$ and $y$ results in a true statement.

\emph{Function} from a set $D$ to a set $R$, defined as a relation from $D$ to $R$ such that each input in $D$ is assigned to no more than one output in $R$; how this definition can be interpreted from the three perspectives for relation.
\end{tabular} 


\begin{tabular}{L{3.2in}|L{3.2in}}
{\bf Proof Structures} & {\bf Mathematical/Teaching Practices} \\ 
\hline \parskip4pt
% Proof structures
\emph{To show that a point $(x,y)$ is on a graph of a relation} means showing that the relation assigns $x$ to $y$.

\emph{To show that a point $(x,y)$ is on the graph of an equation} means showing that evaluating the equation at $x$ and $y$ results in a true statement.
&
% Mathematical/Teaching Practices
\parskip6pt
\emph{Connecting mathematically equivalent definitions}, meaning to understand how different but equivalent definitions can serve different pedagogical and mathematical purposes. 

\emph{Connecting different mathematical representations} of the same concept, meaning to think about different ways of drawing and describing the same mathematical idea. 
\end{tabular}

%%%%%%%%%%%%%%
\header{Summary}

One goal of this lesson is to introduce relations and functions from an advanced perspective. However, more importantly, the goal is to connect the advanced perspective to high school and middle school perspectives, so that teachers have a sense of where the math can go.

Using the parent relation as an opening example, we define {\it relation} in the three (mathematically equivalent) described above. We then define {\it domain}, {\it range}, {\it image} of a point and set, and {\it preimage} of a point and set. 

To highlight the universality of these concepts throughout high school and middle school mathematics, we use examples from algebra (from story problems and also graphs of relations such as $x=y^2$), trigonometry (the relation from $[0,360)$ to $\R$ defined by equivalence of angle measure in degrees), and geometry (rigid motions).

We then introduce {\it inverse relations}, {\it functions}, {\it graphs of functions}, and {\it compositions of functions}.  For each concept in this lesson, we ask teachers to consider how they might explain the connection between the concept and the middle school, high school, and university conceptions of relation, as well as how they might explain how different representations denote mathematically equivalent ideas.

% {\it Acknowledgements.}   <-- fill this in as appropriate.
%%%%%%%%%%%%%%
\newpage
\begin{bignote}[Materials]
\begin{itemize*}
\item Handouts from In-Class Resources (can be printed double-sided)
         % \item other things as necessary, such as colored chalk or markers; other handouts; other props
\end{itemize*}
\end{bignote}

%%%%%%%%%%%%%%%%%%%%%%%%%%%%%%%%%%%
%%%%%% 2 CONTENT %%%%%%%%%%%%%%%%%%%
%%%%%%%%%%%%%%%%%%%%%%%%%%%%%%%%%%%

%%%%%% 2 OPENER %%%%%%

\subsection{Opening example: Parent relation}

\begin{task}
We learned about natural number parents and children last time.

\begin{enumerate}
\item What is the definition of a parent of a natural number child? 
\item 
Let $P$ assign a natural number to each of its parents. We can represent $P$ as a set of arrows from $\N$ to $\N$.  Some arrows below have been filled in, for example $P$ assigns $6$ to $2$ and $3$, and assigns $12$ to $2, 3, 4, 6$.  Draw in more arrows.

\item Consider this statement: ``Some children have no parents, some children have exactly one parent, and some children have multiple parents.''  

Is this statement true or false? Why?

\item How about this statement? ``Some numbers have no children, and some numbers have multiple children.'' 
\end{enumerate}

\vfill
\begin{center}
\begin{tikzpicture}
\def\pt{circle (0.05)}
\foreach \x in {0, 1, 2, 3, 4, 5, 6, 7, 8, 9, 10, 11, 12, 13, 14}
	{\filldraw(0,-0.5*\x) node[left]{$\x$} \pt;
	\filldraw(5,-0.5*\x) node[right]{$\x$} \pt;
	}
	\draw 
		(2.5, 1) node[]{$P$}
		(0,0.7) node[] {$\N$}
		(5,0.7) node[] {$\N$}
		(0, -0.5*15) node[]{$\vdots$}
		(5, -0.5*15) node[]{$\vdots$};
	\draw[->=stealth, yscale=-0.5] (0,6) -- (4.94, 1.9);
	\draw[->, yscale=-0.5] (0,6) -- (4.94, 2.9);
	\draw[->, yscale=-0.5] (0, 12)-- (4.94, 2.1);
	\draw[->, yscale=-0.5] (0, 12) -- (4.94, 3.1);
	\draw[->, yscale=-0.5] (0, 12) -- (4.94, 4);
	\draw[->, yscale=-0.5] (0, 12) -- (4.94, 6);
\draw[xshift=-1cm] (0,0) arc (180:0:1) (0,-7) arc (180:360:1) (0,0)-- (0,-7) (2,0) -- (2,-7); 
\draw[xshift=4cm] (0,0) arc (180:0:1) (0,-7) arc (180:360:1) (0,0)-- (0,-7) (2,0) -- (2,-7); 
\end{tikzpicture}
\end{center}
\end{task}

\begin{solution}({\it Partial})
Given a number $n\in \N$, a parent of $n$ is a nontrivial factor of $n$.

The first statement is true: 

\vspace*{-8pt}
\begin{itemize*}
	\item $n$ has no parents when $n$ is 0, 1, or prime \\ 
	\item $n$ has exactly one parent when $n$ is a perfect square of a prime number \\
	\item $n$ has multiple parents otherwise
\end{itemize*}
\vspace*{-8pt}
These are represented by no arrows starting at a number, exactly one arrow starting at a number, and multiple arrows starting at a number.

The second statement is also true. $0$ and $1$ have no children. All other numbers have multiple children (infinitely many, in fact). These are represented by arrows ending at a number or not.
\end{solution}

%%%%%%% 2 CARTESIAN PRODUCT %%%%%%%%% 

\subsection{Defining relations}
\subsubsection{Cartesian products}
Let's discuss Cartesian products, which you first saw in your homework from last week.

\begin{definition}\label{d: cartesian product}
Let $D$ and $R$ be sets. The \emph{Cartesian product} of $D$ and $R$ is defined as the set of ordered pairs $\{ (x,y) \st x\in D, y\in R\}$. It is denoted $D\times R$.
\end{definition}

\begin{task}
Let $A=\{5, 6, 10\}$, $B=\{-1, -2, -3\}$, $C=\{-1,1\}$. Let $\N$ denote natural numbers, $\Z$ the integers,  and $\R$ the real numbers.

List the elements of the following Cartesian products:
	\begin{itemize}
	\item $A\times B$ 
	\item $A\times C$
	\item $\Z \times C$
	\item $C\times \Z$
	\item $\N \times \N$.
	\end{itemize}

Which of the above sets contains the element $(6, -1)$? How about $(-1, 10)$?

How would you describe $\R\times \R$? 

How about $\R\times (\R\times \R)$?
\end{task}

\begin{solution} ({\it Partial})
$(6,-1)\in A\times B, \Z \times C, \N\times \N$. It is not an element of any of the other sets.

$(-1,10) \in C\times \Z, \N\times \N$. It is not an element of any of the other sets.

$\R\times \R$ is the coordinate plane.

$\R\times (\R\times \R)$ can be thought of as all the coordinates of 3-space.
\end{solution}

\begin{note}
The Cartesian product is sometimes referred to as the ``cross'' or ``cross product'' 
or ``direct product'' of two sets.  The term ``cross product of two vectors'' also shows up in linear algebra, but this meaning is independent of the Cartesian product. It's an unfortunate coincidence that the same term is used.
\end{note}


%%%%%%% 2 RELATIONS %%%%%%%
\subsubsection{Relations}

In middle school, if relations are introduced, they are often done so in the form of a cloud diagram, such as drawn in the opening task. ({\it Question:} What do the ``$\dots$'' mean in the below diagram?) 

\vspace*{-12pt} 
\begin{center}
\begin{tikzpicture}[scale=0.5]
\def\pt{circle (0.05)}
\foreach \x in {0, 1, 2, 3, 4, 5, 6, 7, 8, 9, 10, 11, 12}
	{\filldraw[yscale=0.5](0,-\x) node[left]{\tiny{ $\x$}} \pt;
	\filldraw[yscale=0.5](5,-\x) node[right]{\tiny{$\x$}} \pt;
	}
	\draw 
		(2.5, 1) node[]{$P$}
		(0,0.7) node[] {$\N$}
		(5,0.7) node[] {$\N$}
		(0, -6.5) node[]{$\vdots$}
		(5, -6.5) node[]{$\vdots$}
		(4.5, -3.75) node[]{$\vdots$};
	\draw[->=stealth, yscale=-0.5] (0,4) -- (4.94, 2.1);
	\draw[->, yscale=-0.5] (0,6) -- (4.94, 2.9);
	\draw[->, yscale=-0.5] (0,6) -- (4.94, 1.9);
	\draw[->, yscale=-0.5] (0,8) -- (4.94, 1.85);
	\draw[->, yscale=-0.5] (0,8) -- (4.94, 3.9);
	\draw[->, yscale=-0.5] (0,9) -- (4.94, 2.85);
	\draw[->, yscale=-0.5] (0,10) -- (4.94, 1.8);
	\draw[->, yscale=-0.5] (0,10) -- (4.94, 4.9);
	\draw[->, yscale=-0.5] (0, 12)-- (4.94, 2.1);
	\draw[->, yscale=-0.5] (0, 12) -- (4.94, 3.1);
	\draw[->, yscale=-0.5] (0, 12) -- (4.94, 4);
	\draw[->, yscale=-0.5] (0, 12) -- (4.94, 6);
\draw[xshift=-1cm, yscale=0.75] (0,0.5) arc (180:0:1) (0,-9) arc (180:360:1) (0,0.5)-- (0,-9) (2,0.5) -- (2,-9); 
\draw[xshift=4cm, yscale=0.75] (0,0.5) arc (180:0:1) (0,-9) arc (180:360:1) (0,0.5)-- (0,-9) (2,0.5) -- (2,-9); 
\end{tikzpicture}
\end{center}

\vspace*{-12pt} 
In our example, a relation $P$ maps numbers in $\N$ to numbers in $\N$, and the map is represented by arrows connecting input numbers to output numbers. 

\begin{definition}[Relation: Middle school version]\label{d: relation middle school}
A \emph{relation} from a set $D$ to a set $R$ is a set of arrows going from elements of $D$ to elements of $R$.  


If there is an arrow from and element $x$ to an element $y$, we say the relation \emph{maps} or \emph{assigns} $x$ to $y$.
\end{definition}

We use the notation $r:D\to R$ to mean a relation from $D$ to $R$ called $r$.

\begin{definition}[Parent relation]\label{d: parent relation}
The parent relation $P:\N\to\N$ is the set of arrows from each element of $\N$ to its nontrivial factors.
\end{definition}

\begin{note}
A relation $P:D\to R$ may map an element of $D$ to no elements of $R$, exactly one element of $R$, or multiple elements of $R$. 

An element of $R$ may have no elements of $D$ mapping to it, exactly one element of $D$ mapping to it, or multiple elements of $D$ mapping to it.
\end{note}

\begin{definition}\label{d: candidate domain etc}
For a relation $r:D \to R$, we say that 
	\begin{itemize}
	\item $D$ is the \emph{candidate domain};
	\item $R$ is the \emph{candidate range} (or \emph{codomain});
	\item the \emph{image} of an element $x\in  D$ is the set of elements of $R$ that $x$ is mapped to. Similarly, the \emph{image} of a subset of $S\subset D$ is the subset of $R$ containing the images of all elements of $S$.
	\item When a element in $D$ maps to no element in $R$, we say it has \emph{empty image}. If an element in $D$ does map to at least one element in $R$, we say it has \emph{nonempty image}.
	\item the \emph{domain} of $r$ is the subset $D'\subset D$ of elements with nonempty image.
	\item the \emph{preimage} of an element of $R$ is the set of elements of $D$ that map to that element of $R$. Similarly, the \emph{preimage} of a subset $T\subset R$ is the subset of $D$ containing the preimages of all elements of $T$.
	\item When a element in $R$ has no element in $D$ mapping to it, we say it has \emph{empty preimage}. If an element in $R$ does have an element in $D$ mapping to it, we say it has \emph{nonempty preimage}.
	\item the \emph{range} (or \emph{image}) of the relation $r$ is the subset $R'\subset R$ with nonempty preimage.
	\end{itemize}
\end{definition}

\begin{note}
In these materials, we will use the terms ``codomain'' and ``candidate range'' interchangeably. We will also use the terms ``range'' and ``image'' interchangeably. The terms ``codomain'' and ``image'' typically do not show up in K-12 materials; they are typically introduced in university or graduate mathematics. The term ``range'' is standard to middle school and high school materials, though sometimes ``range'' is used to mean ``candidate range'' and other times it is used to mean ``the set of elements with nonempty preimage''. In these materials, ``range'' only refers to the latter.
\end{note}

\begin{bignote}
In writing these materials, we sought to find a standard term for what we call the ``candidate domain.'' To our knowledge, there is no well-known standard term for this concept. This is perhaps in part because the distinction matters primarily in undergraduate level mathematics and beyond, for instance in defining function composition; and perhaps also because a certain amount of notational interpretation is assumed and we loosen the restriction. For instance, meromorphic functions on C are really holomorphic functions on the complement of a discrete set in $\mathbb{C}$. Functions in
$L^2(\mathbb{R})$ are technically only well defined up to equality of all integrals, so
they don't have a strict notion of  ``domain''; however, mathematicians still talk about the ``domain'' of an  $L^2$ function.

For prospective high school teachers, we wanted to be careful about differentiating between the ``candidate domain'' and the ``domain'', so we could problematize the difference between them. The term ``candidate domain'' was suggested to us by a high school mathematics teacher. We also considered ``corange'', as suggested by some mathematicians. We decided on ``candidate domain'' because the term seemed more down-to-earth, and it suggests the question, ``Is this really the domain? If not, how can we fix it to be the domain?''

The term ``candidate range'' was chosen to mirror ``candidate domain''. The phrase ``a relation (or function) from a candidate domain to a candidate range'' is more straightforward to high school teachers than ``from a candidate domain to a codomain''.  \end{bignote}

We have seen examples of most of these concepts in the parent relation. 

\begin{example}
\label{ex: relations}
Other examples of relations might be:


	\begin{itemize}
	\item (car-owner relation) The relation from the candidate domain of all cars in the world to candidate range of all people in the world, mapping a car to its owner(s).
	\item (course assignment relation) The relation from the candidate domain of rooms in the mathematics building to candidate range of courses taking place at 1pm, mapping a room to the course being taught in it at 1pm.
	\end{itemize}
	
	We will use the following throughout this chapter.  Let: 
	\begin{itemize}
	\item $T$ be the relation that maps each day of the year 2030 to the its average temperature in $\degrees F$ that day. 

	\item $A$ be the relation that maps each degree in the interval $[0\degrees, 360\degrees)$ to all degrees in the interval $(-\infty,\infty)$ that give an equivalent angle measure. (This relation connects to trigonometry)

	\item $\rho$ be the relation that maps a point in the plane to its rotation about the origin by $90\degrees$. (This means $90\degrees$ {\it counter}clockwise.) (This relation connects to middle school and high school geometry)

	\item $G$ be the relation that maps $x$ to every $y$ such that $x=y^2$. (This relation connects to graphing and high school algebra)
	\end{itemize}
\end{example} 


In the above examples, we can see how each condition of the note about relations may apply.


\begin{task}
% connections to word problems in MS and HS math
What are the domain and range of the car-owner relation and the room-course relation? 

Suppose this table contains course assignments to rooms at 1pm. What is the image of Math Bldg Room 100? What is the preimage of Math 996, Math 405, Math 100, and Math 221 under the room-course relation? 

\begin{center}
\begin{tabular}{|c|c|}
\hline
Room & Course in room at 1pm \\ \hline
Math Bldg room 100 & Math 996 \\ 
Math Bldg room 104 & Math 100 \\ 
Engineering Bldg room 750 & Math 405 \\ 
not being offered this term & Math 221 \\ \hline
\end{tabular}
\end{center}

% Let $T$ be the relation that maps each day of the year 2030 to the its average temperature in $\degrees F$ that day. 
For the relation $T$, describe a possible candidate domain, domain, candidate range, and range of this relation.

% Let $A$ be the relation that maps each degree in the interval $[0\degrees, 360\degrees)$ to all degrees in the interval $(-\infty,\infty)$ that give an equivalent angle measure. 
For the relation $A$, what is the preimage of $361\degrees$? What is the image of $0\degrees$? 

% connection to geometry
% Let $\rho$ be the relation that maps a point in the plane to its rotation about the origin by $90\degrees$. (This means $90\degrees$ {\it counter}clockwise.) 
For the relation $\rho$, what is the image of the point $(1,0)$? What is the preimage of the point $(-2,0)$? 

% connection to graphing 
% Let $G$ be the relation that maps $x$ to every $y$ such that $x=y^2$. 
For the relation $G$, what is the image of $4$? What is the preimage of $-6$?

\end{task}

\begin{task}
Interpret the definitions of candidate domain, domain, image, preimage, candidate range, and range in terms of arrows and their start and end points.
\end{task}

We can think of the definition as a verbal representation of these concepts and the cloud diagrams as another. When connecting different representations, it is often helpful to keep these elements in mind:

\begin{itemize}
\item Explain how representations appear different and how they appear the same.
\item Identify how the representations can be used highlight different features. 
\item Go back and forth between the representations: where are the features of one representation located in the other representation?
\end{itemize}

At the high school level, textbooks generally do not use cloud diagrams any more, nor do they talk about arrows. Instead, discussion of relations (and functions) are in terms of assignments. The definition in high school is mathematically equivalent to the middle school version, but stated in a way that more directly allows for defining concepts like the graph of a relation or later, the behavior of a function.  (We note that as we will discuss later, a function is a kind of relation.)

\begin{definition}[Relation: High school version]\label{d: relation high school}
A \emph{relation} $P$ from a set $D$ to a set $R$ a set of assignments from elements of $D$, called inputs, to elements of $R$, called outputs.
\end{definition}

\begin{note}
We use the notation $P:D\to R$ to mean a relation from $D$ to $R$, and the notation $x\mapsto y$ to denote an assignment from $x\in D$ to $y\in R$. Something to keep in mind for ``assignment'' is that an assignment has to map something to something. So we think of an assignment not just as an ``arrow'' but as an arrow with specific start and end points.
\end{note}

The next two tasks use the relation $A$ as defined in Example \ref{ex: relations}.

\begin{task}
What are some example assignments of the relation $A$? Use the $x\mapsto y$ notation to write down your examples. 
\end{task}

\begin{solution}
Some examples of assignments are: $0\degrees \mapsto 360\degrees$, $0\degrees \mapsto 0\degrees$, $359\degrees \mapsto -1\degrees$, $90\degrees \mapsto 810\degrees$.
\end{solution}

\begin{task}
Suppose we were to graph the relation $A$. What might this graph look like? What are some examples of coordinates that are contained in this graph?
\end{task}

\begin{solution}
It would look like the set of all lines of the form $y=x+360n$, where $n\in \Z$. Some example coordinates are $(0, 360), (0,0), (359, -1), (90, 810)$.
\end{solution}

\begin{task}
Suppose we were to graph the parent relation. What might this graph look like? What are some examples of coordinates that are contained in this graph?
\end{task}

In undergraduate courses such as real analysis as well as in graduate courses in analysis, we go one step farther. Rather than defining relations in terms of assignments, we define relations in terms of ordered pairs. The ordered pairs represent the assignments. 

\begin{definition}[Relation: University version]\label{d: relation university}
A \emph{relation} $r:D\to R$ is a subset of $D\times R$, i.e., $r\subset D\times R$.
\end{definition}

One way to think about this definition is that we are defining the relation as its graph in the space $D\times R$. 

\begin{note}
One question that might come up is: if the candidate domain could be anything, then why bother finding good candidate domains? Why not instead let the candidate domain be the largest set that we could think of? For instance, we might set the candidate domain to be something like this:
$$\mathbb{R}\cup (\textnormal{all cars in the world}) \cup(\textnormal{all rooms in all buildings in the world})\cup \dots \,.$$ One reason that we would not want to do this is that eventually, we want to construct and compare graphs of relations and functions. The candidate domain and candidate range of comparable relations and functions are likely to be similar to each other, and the graphs live in the space $D\times R$ where $D$ is the candidate domain and $R$ is the candidate range for these relations or functions. \end{note}

%%%%%%% 2 BUILDING RELATIONS %%%%%%%
\subsection{Building relations from existing relations}

There are two main ways to build relations from existing relations: inverses and composition. We will see these again, in more depth, when we discuss functions.

%%%%%%% 2 INVERSE OF A RELATION %%%%%%%
\subsubsection{Inverse of a relation}

\begin{definition}[Inverse relation: Middle school version]\label{d: inverse relation middle school}
If $r$ is a relation from a set $D$ to $R$, then the \emph{inverse relation} of $r$ is the relation that swaps the direction of the arrows of $r$.  The arrows of the inverse relation go from elements of $R$ to elements of $D$.

In other words, there is an arrow from $x$ to $y$ in $r$ if and only if there is an arrow from $y$ to $x$ in $r^{-1}$.
\end{definition}

\begin{definition}[Inverse relation: High school version]\label{d: inverse relation high school}
Given a relation $r:D\to R$, the \emph{inverse relation} of $r$ is the set of assignments $y\mapsto x$ such that $x\mapsto y$ is an assignment of $r$. The inverse relation is denoted $r^{-1}$. 

In other words, $x\mapsto y$ is an assignment of a relation $r$ if and only if $y\mapsto x$ is an assignment of $r^{-1}$. 
\end{definition}

\begin{definition}[Inverse relation: University version]\label{d: inverse relation university}
Given a relation $r:D\to R$, the \emph{inverse relation} of $r$ is defined  
	$$r^{-1} = \{(y, x) \st (x,y)\in P\}.$$
	
In other words,	$(x,y)\in  r$ if and only if $(y,x)\in r^{-1}$.
\end{definition}

\vspace*{-2pt}
\begin{minipage}{4in}
As an example, let's look at the parent relation. The inverse of the parent relation could be represented like the following. ({\it Question:} What do the ``$\dots$'' mean in this representation?) 

\vspace*{-2pt}
\begin{task}
What other arrows does the inverse of the parent relation contain? What might be a good name for this relation?

What is the inverse of the car-owner relation? How about the relations $T$, $A$, $\rho$, and $G$?
\end{task}
\end{minipage}
\begin{minipage}{2.4in}
 \begin{center}
\begin{tikzpicture}[xscale=-1, scale=0.4]
\def\pt{circle (0.05)}
\foreach \x in {0, 1, 2, 3, 4, 5, 6, 7, 8, 9, 10, 11, 12}
	{\filldraw[yscale=0.5](0,-\x) node[left]{\tiny{ $\x$}} \pt;
	\filldraw[yscale=0.5](5,-\x) node[right]{\tiny{$\x$}} \pt;
	}
	\draw 
		(2.5, 1) node[]{$P$}
		(0,0.7) node[] {$\N$}
		(5,0.7) node[] {$\N$}
		(0, -6.5) node[]{$\vdots$}
		(5, -6.5) node[]{$\vdots$}
		(2.5, -5) node[]{$\vdots$};
	\draw[<-=stealth, yscale=-0.5] (0,4) -- (4.94, 2);
	\draw[<-, yscale=-0.5] (0,6) -- (4.94, 3);
	\draw[<-, yscale=-0.5] (0,6) -- (4.94, 2);
	\draw[<-, yscale=-0.5] (0,8) -- (4.94, 2);
	\draw[<-, yscale=-0.5] (0,8) -- (4.94, 4);
	\draw[<-, yscale=-0.5] (0,9) -- (4.94, 3);
	\draw[<-, yscale=-0.5] (0,10) -- (4.94, 3);
	\draw[<-, yscale=-0.5] (0,10) -- (4.94, 5);
	\draw[<-, yscale=-0.5] (0, 12)-- (4.94, 2);
	\draw[<-, yscale=-0.5] (0, 12) -- (4.94, 3);
	\draw[<-, yscale=-0.5] (0, 12) -- (4.94, 4);
	\draw[<-, yscale=-0.5] (0, 12) -- (4.94, 6);
\draw[xshift=-1cm, yscale=0.75] (0,0.5) arc (180:0:1) (0,-9) arc (180:360:1) (0,0.5)-- (0,-9) (2,0.5) -- (2,-9); 
\draw[xshift=4cm, yscale=0.75] (0,0.5) arc (180:0:1) (0,-9) arc (180:360:1) (0,0.5)-- (0,-9) (2,0.5) -- (2,-9); 
\end{tikzpicture}
\end{center}
\end{minipage}

\vspace*{-6pt}
\smallnote{One reasonable name for the inverse of the parent relation might be the ``child relation'', as this relation maps natural numbers to their children. The inverse of the car-owner relation is the relation from all people in the world to all cars in the world that map a person to all the cars they own. The relation $T^{-1}$ maps possible temperatures to the days on which that temperature was the day's average temperature. The relation $A^{-1}$ maps an element of $\R$ to the element of $[0,360)$ which represents its angle measure. The relation $\rho^{-1}$ is rotation about the origin by $-90\degrees$, which is $90\degrees$ clockwise. The relation $G^{-1}$ is the relation that maps $y$ to every $x$ such that $x=y^2$.}

\vspace*{-8pt}
\begin{task}
Discuss the three versions of the definition of inverse of a relation. What do they each say? How would you represent them? What makes them mathematically equivalent?
\end{task}

Here are some elements to keep in mind when connecting mathematically equivalent definitions; how did they each come up in your discussion?

\begin{itemize}
\item Explain how they appear different and how they appear the same.
\item Explain why they are mathematically equivalent. 
\item Analyze the mathematical and pedagogical purposes of each definition and why they may be appropriate for different levels of mathematical study, or why one version makes more sense after having worked with another version.
\end{itemize}


\vspace*{-4pt}
\begin{task}
Let $(P^{-1})^{-1}$ be the inverse of the relation $P^{-1}$. What is this relation? How do you know?
Let $(G^{-1})^{-1}$ be the inverse of the relation $G^{-1}$. What is this relation? How do you know?

In general, given a relation $r$, what is $(r^{-1})^{-1}$?  Explain how you know.
\end{task}

To show that two relations are the ``same'', we can go back to two definitions: that of relation, and that of set equality (Definition \ref{definition: set equality}). A relation is defined as a set of assignments (or, in the language of middle school, arrows; or, in the language of university, ordered pairs). Two sets are equal if they are both subsets of each other.  Hence, to show that two relations are equal to each other, we show that their assignments are both subsets of each other.

\begin{proposition}\label{p: inverse of an inverse}
The inverse of the inverse of a relation is the original relation. 
\end{proposition}

\begin{proof}[Proof (using high school version)]
The definition of inverse of a relation states that $x\mapsto y$ is an assignment of a relation $r$ if and only if $y\mapsto x$ is an assignment of $r^{-1}$.  We can also use the definition to say that $y\mapsto x$ is an assignment of $r^{-1}$ if and only if $x\mapsto y$ is an assignment of $(r^{-1})^{-1}$.  Hence, combining these if and only if statements, we have that  $x\mapsto y$ is an assignment of a relation $r$ if and only if it is also an assignment of $(r^{-1})^{-1}$. The set of assignments of $r$ are equal to the set of assignments of $(r^{-1})^{-1}$, and so $r=(r^{-1})^{-1}$.
\end{proof}


We can interpret this proof in terms of sets. Let $A$ be the set of assignments in $r$, and $B$ be the set of assignments in $(r^{-1})^{-1}$. We have $x\mapsto y$ is an element of $A$ if and only if it is an element of $B$. The ``if'' part of this statement means that $B\subset A$; the ``only if'' part of this statement means $A\subset B$. Hence $A=B$, so the set of assignments of $r$ are equal to the set of assignments of $(r^{-1})^{-1}$, and so $r=(r^{-1})^{-1}$.

%%%%%%% 2 COMPOSITION OF RELATIONS %%%%%%%%%
\subsubsection{Composition of relations}
\label{s: composition}

In the remainder of this chapter, we work almost exclusively with cases where  $D$ and $R$ are $\N$, $\R$, or $\R^2$. We make this choice for two main reasons:
	\begin{itemize*}
	\item Most examples of composition in middle school and high school mathematics are those where the candidate domain and candidate range can be both $\N$ (in middle school algebra), both $\R$ (in algebra), or both $\R^2$ (in geometry).
	\item When the candidate domain and candidate range do not equal each other, the details of some results require more technical bookkeeping.The idea behind these results is more important for high school teaching than learning the technical bookkeeping.
	\end{itemize*}

\begin{task}
In high school and college, what did you learn that the notation $f\circ g(x)$ means? Circle your answer.
\centerline{Do $f$ then $g$ \quad\quad\quad Do $g$ then $f$}
\end{task}

\begin{solution}
It means to perform $g$ and then $f$. For example, if $g(x)=x^2$ and $f(x)=5x$, then $f\circ g(x)=f(x^2)=5x^2$, wherease $g\circ f(x)=g(5x)=(5x)^2=25x^2$.
\end{solution}

\smallnote{It may seem excessive to do this task recalling the definition of function notation. However, in our experience, it is better to spend a minute making sure that all students recall function notation correctly prior to using it and are primed to use it, rather than losing those students who may not remember. In our experience, almost all students do remember {\it if prompted}. If not prompted, we have found that there are a few students who do not remember, and then extra time is spend remediating.}

\begin{definition}[composition]\label{d: relation composition}
Given two relations $P:D\to D$ and $Q:D\to D$, we define the \emph{composition} of $P$ then $Q$ 
as the relation that assigns $x$ to $z$ whenever there is a $y\in D$ such that $P$ assigns $x\mapsto y$ and $Q$ assigns $y\mapsto z$.
\end{definition}

\vspace*{-8pt}
\begin{task}
Let $P$ be the parent relation and let $t:\N\to \N, x\mapsto 2x$. How would you represent $t\circ P$ using the middle school version of relation? 
\end{task}

\begin{task}
For the relation $t\circ P$, what is the image of $6$? Of $12$? Of $9$?

For the relation $t\circ P$, what is the preimage of $4$? Of the set $\{1, 3, 5\}$? Of the set $\{4,14\}$?

What are the domain and range of the relations $t\circ P$?

How would you represent $P\circ t$? What are its domain and range?
\end{task}

\begin{solution}({\it Partial})
We can represent compositions with concatenated cloud diagrams. 

\vspace*{-8pt}
\begin{center}

\begin{tabular}{c@{\hspace*{1in}}c}
$t\circ P$ & $P\circ t$ \\
& \\
\begin{tikzpicture}[scale=0.4]
\def\pt{circle (0.05)}
\foreach \x in {0, 1, 2, 3, 4, 5, 6, 7, 8, 9, 10, 11, 12}
	{
	\filldraw[yscale=0.5](0,-\x) node[left]{\tiny{ $\x$}} \pt;
	\filldraw[yscale=0.5](5,-\x) node[right]{\tiny{$\x$}} \pt;
	\filldraw[yscale=0.5](10,-\x) node[right]{\tiny{$\x$}} \pt;
	}
	\draw 
		(0,0.7) node[] {$\N$}
		(2.5, 1) node[]{$t$}		
		(5,0.7) node[] {$\N$}
		(7.5, 1) node[]{$P$}
		(10,0.7) node[] {$\N$}
		(0, -6.5) node[]{$\vdots$}
		(4.5, -3.75) node[]{$\vdots$}
		(5, -6.5) node[]{$\vdots$}
		(10, -6.5) node[]{$\vdots$}
		(9.5, -6) node[]{$\vdots$};
	% arrows for P
	\draw[->=stealth, yscale=-0.5] (0,4) -- (4.94, 2.1);
	\draw[->, yscale=-0.5] (0,6) -- (4.94, 2.9);
	\draw[->, yscale=-0.5] (0,6) -- (4.94, 1.9);
	\draw[->, yscale=-0.5] (0,8) -- (4.94, 1.85);
	\draw[->, yscale=-0.5] (0,8) -- (4.94, 3.9);
	\draw[->, yscale=-0.5] (0,9) -- (4.94, 2.85);
	\draw[->, yscale=-0.5] (0,10) -- (4.94, 1.8);
	\draw[->, yscale=-0.5] (0,10) -- (4.94, 4.9);
	\draw[->, yscale=-0.5] (0, 12)-- (4.94, 2.1);
	\draw[->, yscale=-0.5] (0, 12) -- (4.94, 3.1);
	\draw[->, yscale=-0.5] (0, 12) -- (4.94, 4);
	\draw[->, yscale=-0.5] (0, 12) -- (4.94, 6);
	% arrows for t
	\draw[->, yscale=-0.5] (5,0) -- (9.94, 0);
	\draw[->, yscale=-0.5] (5,1) -- (9.94, 2.1);
	\draw[->, yscale=-0.5] (5,2) -- (9.94, 4.1);
	\draw[->, yscale=-0.5] (5,3) -- (9.94, 6.1);
	\draw[->, yscale=-0.5] (5,4) -- (9.94, 8.1);
	\draw[->, yscale=-0.5] (5,5) -- (9.94, 10.1);
	\draw[->, yscale=-0.5] (5,6) -- (9.94, 12.1);
\draw[xshift=-1cm, yscale=0.75] (0,0.5) arc (180:0:1) (0,-9) arc (180:360:1) (0,0.5)-- (0,-9) (2,0.5) -- (2,-9); 
\draw[xshift=4cm, yscale=0.75] (0,0.5) arc (180:0:1) (0,-9) arc (180:360:1) (0,0.5)-- (0,-9) (2,0.5) -- (2,-9); 
\draw[xshift=9cm, yscale=0.75] (0,0.5) arc (180:0:1) (0,-9) arc (180:360:1) (0,0.5)-- (0,-9) (2,0.5) -- (2,-9); 
\end{tikzpicture}
%
&
%
\begin{tikzpicture}[scale=0.4]
\def\pt{circle (0.05)}
\foreach \x in {0, 1, 2, 3, 4, 5, 6, 7, 8, 9, 10, 11, 12}
	{
	\filldraw[yscale=0.5](0,-\x) node[left]{\tiny{ $\x$}} \pt;
	\filldraw[yscale=0.5](5,-\x) node[right]{\tiny{$\x$}} \pt;
	\filldraw[yscale=0.5](10,-\x) node[right]{\tiny{$\x$}} \pt;
	}
	\draw 
		(0,0.7) node[] {$\N$}
		(2.5, 1) node[]{$t$}		
		(5,0.7) node[] {$\N$}
		(7.5, 1) node[]{$P$}
		(10,0.7) node[] {$\N$}
		(0, -6.5) node[]{$\vdots$}
		(4.5, -6) node[]{$\vdots$}
		(5, -6.5) node[]{$\vdots$}
		(10, -6.5) node[]{$\vdots$}
		(9.5, -3.75) node[]{$\vdots$};
	% arrows for t
	\draw[->=stealth, ->, yscale=-0.5] (0,0) -- (4.94, 0);
	\draw[->=stealth, ->, yscale=-0.5] (0,1) -- (4.94, 2.1);
	\draw[->=stealth, ->, yscale=-0.5] (0,2) -- (4.94, 4.1);
	\draw[->, yscale=-0.5] (0,3) -- (4.94, 6.1);
	\draw[->, yscale=-0.5] (0,4) -- (4.94, 8.1);
	\draw[->, yscale=-0.5] (0,5) -- (4.94, 10.1);
	\draw[->, yscale=-0.5] (0,6) -- (4.94, 12.1);
	% arrows for P
	\draw[->, yscale=-0.5] (5,4) -- (9.94, 2.1);
	\draw[->, yscale=-0.5] (5,6) -- (9.94, 2.9);
	\draw[->, yscale=-0.5] (5,6) -- (9.94, 1.9);
	\draw[->, yscale=-0.5] (5,8) -- (9.94, 1.85);
	\draw[->, yscale=-0.5] (5,8) -- (9.94, 3.9);
	\draw[->, yscale=-0.5] (5,9) -- (9.94, 2.85);
	\draw[->, yscale=-0.5] (5,10) -- (9.94, 1.8);
	\draw[->, yscale=-0.5] (5,10) -- (9.94, 4.9);
	\draw[->, yscale=-0.5] (5, 12)-- (9.94, 2.1);
	\draw[->, yscale=-0.5] (5, 12) -- (9.94, 3.1);
	\draw[->, yscale=-0.5] (5, 12) -- (9.94, 4);
	\draw[->, yscale=-0.5] (5, 12) -- (9.94, 6);
\draw[xshift=-1cm, yscale=0.75] (0,0.5) arc (180:0:1) (0,-9) arc (180:360:1) (0,0.5)-- (0,-9) (2,0.5) -- (2,-9); 
\draw[xshift=4cm, yscale=0.75] (0,0.5) arc (180:0:1) (0,-9) arc (180:360:1) (0,0.5)-- (0,-9) (2,0.5) -- (2,-9); 
\draw[xshift=9cm, yscale=0.75] (0,0.5) arc (180:0:1) (0,-9) arc (180:360:1) (0,0.5)-- (0,-9) (2,0.5) -- (2,-9); 
\end{tikzpicture}
\end{tabular}

\begin{tabular}{lll}
	composition & domain & range \\ 
	\hline
$t\circ P$ & $\{n\in \N: n\neq 0, 1 \tn{ and $n$ is not prime}\}$ & All even natural numbers $\geq 4$ \\ 
$P \circ t$ & $\{n\in \N: n \geq 2\}$ & $\{n\in \N: n\geq 2\}$
\end{tabular}
\end{center}

Note that $1$ is {\it not} in the domain of $P\circ t$, because $P$ does not assign $t(1)$ to any element. So $P(t(1))$ is undefined. 
\end{solution}


\smallnote{One way to discuss the domain of $t\circ P$ is to ask, ``Raise your hand if 5 was in your domain \dots 4? \dots 3? \dots 2? \dots 1? \dots 0?'' Make a note of any disagreements or wide agreements to the class orally. Then ask for students' reasoning about $5$, $1$, and $0$.}

Now let's take on the challenge of combining two ideas that we've been working with: inverse of a relation and composition.
\begin{task}
Sketch a representation of $A^{-1}\circ A$. What are its domain and range? What is the image of $45\degrees$? What is the preimage of $45\degrees$?
\end{task}

\begin{task}
We often think about inverses as ``undoing'' something. How well does this analogy work in the case of relations? What goes well with the analogy? What goes wrong with the analogy?
\end{task}

%%%%%%%2 GRAPH OF A RELATION %%%%%%%%%%%%
\subsection{Working with graphs of relations}

In this section, we work exclusively with graphs of functions whose candidate domain and candidate range are subsets of $\R$. 
 
\begin{definition}[graph of a relation]\label{d: graph of a relation}
The \emph{graph of a relation} $r:D\to R$ is defined as the set of points $(a, b)\in \R^2$ such that $r$ assigns $a\mapsto b$.
\end{definition}

\vspace*{-8pt}
\begin{task}
Graph these relations: $P$, $P^{-1}$, $t\circ P$, the relation from $\R$ to $\R$ that maps $x$ to $y$ such that $x=y^2$; the relation from $\R$ to $\R$ that maps $x$ to $y$ such that $y-2^{|x|}=0$.
\end{task}

\vspace*{-12pt}
\smallnote{It may help to do these examples with sample points. It may be helpful to ask the pre-service teachers to explain why the graph of a relation defined by an equation is the graph of that equation.}

As we have just seen, {\it any equation in $x$ and $y$ defines a relation}! For example, the equation $x=y^2$ defines the relation that maps $x$ to all $y$ such that $x=y^2$.  One way to understand this is to think about the ``university version'' of the definition of relation.

\begin{definition}[graph of an equation]\label{d: graph of an equation}
The \emph{graph of an equation} in $x$ and $y$ is defined as the set of points $(a, b) \in \R^2$ such that evaluating the equation at $x=a$ and $y=b$ results in a true statement.
\end{definition}

{\bf Proof structure.}
To show that a point $(a, b)$ is on a graph of a relation means showing that the relation assigns $a$ to $b$.

As an example of this structure, consider:
\begin{task}
What are all the points on the graph of $A$ with $x$-coordinate $45\degrees$? With $y$-coordinate $45\degrees$?
\end{task}

Now try:

\vspace*{-4pt}
\begin{task}
What are all the points on the graph of $A^{-1}$ with $x$-coordinate $60\degrees$? With $y$-coordinate $60\degrees$?
 
Is the point $(60, 400)$ on the graph of $A^{-1}$? How about $(430, 70)$? $(70, 430)$? (10, 200)? Why or why not?
  
What is the graph of $A^{-1}$? How do you know you have graphed all points and not graphed any extra points?
\end{task}

\vspace*{-10pt}
\smallnote{In the above, attend to the reasoning why or why not, and make sure that students are {\it explicitly} referencing the definition of the graph of a relation. It can be helpful to prompt with questions like, ``How are you using the definition of the graph of a relation?'' Something else that is helpful if there is hesitation in using the definition is to go around the room and ask each person to state the definition of graph of a relation. This may sound over the top, but when used on occasion, it can be an effective technique for both remembering key definitions and also impressing the importance of particular statements. When students are reluctant to engage in this, it is more about not having done anything like this before than a fundamental aversion; in our experience, students eventually take this in good humor and appreciate being given the time to commit a definition to memory in a public and verbal way.}


\begin{task}
Show that for any relation $r:D\to D$, if $x\in D$ is in the domain of $r$, then $(x,x)$ lies on the graph of $r^{-1}\circ r$. 
\end{task}

We work with graphs of equations in a similar way to working with graphs of relations. 

{\bf Proof structure.}
To show that a point $(x, y)$ is on the graph of an equation means showing that evaluating the equation at $x=a$ and $y=b$ results in a true statement.

\begin{task}
Is the point $(1,2)$ on the graph of $x^2+\frac{y^2}{4}=1$?  How about the point $(\frac{1}{2},\frac{2}{3})$? 

Find a point that is on the graph of $y=x^2+1$ and the graph of $y=6x-4$. Is that the only point? Are there more points? How do you know?

Describe the graph of the equation $x=0$. Describe the graph of the equation $y=0$. How do you know that these graphs look this way?

Where does $x^2+\frac{y^2}{4}=1$ intersect the $x$-axis? Where does $x^2+\frac{y^2}{4}=1$ intersect the $y$-axis?

What is a point that is on the graph of $y=x^2+1$ and the graph of $y=5$. Then solve for $x$ when in $y=x^2+1$ when $y$ is $5$. Explain the numerical coincidence.
\end{task}

\smallnote{Again, in doing this task, attend to pre-teachers' reasoning and make sure that they are {\it explicitly} referencing the definition of graph of equations, not just plugging in numbers without saying why it makes sense to plug in numbers. It is also important for pre-service teachers to explicitly use the fact that the $x$-axis ($y$-axis) consists of all points whose $y$-coordinate ($x$-coordinate) is 0.}

In this task, we were using these ideas:

\begin{definition}\label{d: intercepts}
Given an equation in $x$ and $y$ and its graph, all points $(a,0)$ on the graph are called \emph{$x$-intercepts} of the graph. A graph may have 0, 1, or multiple $x$-intercepts.

Given an equation in $x$ and $y$ and its graph, all points $(0,b)$ on the graph are called \emph{$y$-intercepts} of the graph. A graph may have 0, 1, or multiple $y$-intercepts.
\end{definition}

\begin{definition}\label{d: intersection}
We say that two graphs \emph{intersect} each other at a point $(a,b)$ when $(a,b)$ is contained in both graphs.  

Graphs may intersect at zero points, one point, multiple points, and sometimes even infinitely many points.
\end{definition}

\begin{task}
What are all the $x$-intercepts of the graphs of $A$ and $A^{-1}$?

What are all the $y$-intercepts of the graphs of $A$ and $A^{-1}$?

What is the intersection of the graphs $A$ and $A^{-1}$?
\end{task}

\smallnote{In discussing this task, make sure students are {\it explicitly} referencing the definition of $x$- and $y$-intercepts, as well as writing down the intercepts as coordinates, not numbers. Watch that they are explaining their reasoning, not just plugging in numbers without saying why it makes sense to plug in numbers.}

One takeaway from these examples is that when we attending to and explicitly referring to the definition of graph of a relation or graph of an equation, we are in a better position to help students understand concepts such as:
	\begin{itemize*}
	\item $x$-intercepts and $y$-intercepts
	\item intersections of graphs
	\item solving for particular values of an equation
	\end{itemize*}

Now let's see how this might show up in an actual classroom.  

\smallnote{What follows is an approximately 6 minute clip showing how these definitions can come up in teaching, and why it is important to go back to the definition. If there is time, it is worth showing this clip to students. Otherwise, this also works as a homework assignment, especially followed up by a quick discussion the next day.}

\begin{bignote}
When introducing video clips of teaching, we have found it useful to:
	\begin{itemize*}
	\item Emphasize that we are {\it not} viewing videos to judge the teacher or students or their interactions. Instead, we are practicing how to observe {\it without judgment} to understand what is going on.
	\item Provide specific viewing questions.
	\item Emphasize that comments should be based on evidence in the video.
	\end{itemize*}
	
Even in professional development with long-time teachers, it is helpful to provide the reminder about observation rather than judgment; this goes doubly (or more) with novice teachers. We have provided some sample text below for how to explain this way of viewing teaching.

The reason it is helpful to have specific viewing questions is that it directs the conversation and allows you as an instructor to remind the students of the purpose of watching the video and how it fits into the mathematical or other goals of the lesson. Otherwise conversations can have a way of meandering unproductively; setting viewing questions up front provides a structure to hold a productive conversation.

The reason to base discussions on evidence is that otherwise, it is too easy to project one's own experiences and interpretations and leave the video context. However, the only shared context is that of the video, so grounding the conversation in the video provides some insurance for a coherent conversation. Prompts that you may use with the prospective teachers include:
	\begin{itemize*}
	\item ``Can you say what part of the video you are basing your comment on?''
	\item ``That's interesting. What interaction in the video were you thinking about?''
	\end{itemize*}
\end{bignote}

\begin{task}
We will watch a short video of teaching by Ms.~Barbara Shreve of San Lorenzo High School. The video shows her teaching an intervention class called Algebra Success. The students in this class have been previously unsuccessful in Algebra 1. They are working on finding intercepts of equations to get ready for working with quadratics. 

As you watch the video, it may be tempting to think about what you personally think is good or not as good about the teaching, or what you might have done differently. But before getting to these kinds of judgments, it is more important to simply observe what is going on, what the students' reasoning is, and what the story line is. (This is just like when working with students, as we will see later in this class and you will learn in your methods class: before evaluating students' work, we must first observe and understand students' work without judgment.) Here, we will practice observing the interactions between teachers and 

As you watch the video, think about the following questions:
	\begin{itemize*}
	\item How does the teacher emphasize to students to explain their reasoning?
	\item How does the teacher help students feel comfortable sharing their reasoning?
	\item How was the definition of $x$-intercept or $y$-intercept used?
	\end{itemize*}
	
\hrule 
\vspace*{2pt}

Here is a link to the video:
\url{http://www.insidemathematics.org/classroom-videos/public-lessons/9th-11th-grade-math-quadratic-functions/introduction-part-b}

\vspace*{2pt}
\hrule
\vspace*{2pt}

Again, our discussion will be about the viewing questions. We will have time for general comments later.  Let's take the viewing questions one at a time.

When addressing the viewing questions, be specific about your evidence from the video to support what you are saying.

\vspace*{2pt}
\hrule
\vspace*{2pt}

Now that we have discussed the viewing questions, what other thoughts or questions come to mind?
\end{task}

\subsection{Putting it all together: Investigating graphs of inverses}
To finish this chapter, let's investigate the examples we have worked with and try to generalize.

Here are some graphs that we've seen: 

\begin{center}
\begin{tabular}{cccccc}
$P$ & $P^{-1}$ & $A$ & $A^{-1}$ & $G$ & $G^{-1}$ \\ 
\begin{tikzpicture} [scale=0.1]
\draw(-0.5,0) -- (20.5,0) (0,-0.5) -- (0,20.5);
\def\pt{circle (0.09)}
\foreach \x in {(4,2),(6,2),(6,3),(8,2),(8,4),(9,3),(10,2),(10,5), (12, 2),(12,3),(12,4),(12,6), (14,2),(14,7),(15,3),(15,5),(16,2),(16,4),(16,8),(18,2),(18,9),(20,2),(20,4),(20,5),(20,10)}
	{	
	\filldraw \x \pt;
	}
\end{tikzpicture}
& 
\begin{tikzpicture} [scale=0.1, rotate=45, yscale=-1, rotate=-45]
\draw(-0.5,0) -- (20.5,0) (0,-0.5) -- (0,20.5);
\def\pt{circle (0.09)}
\foreach \x in {(4,2),(6,2),(6,3),(8,2),(8,4),(9,3),(10,2),(10,5), (12, 2),(12,3),(12,4),(12,6), (14,2),(14,7),(15,3),(15,5),(16,2),(16,4),(16,8),(18,2),(18,9),(20,2),(20,4),(20,5),(20,10)}
	{	
	\filldraw \x \pt;
	}
\end{tikzpicture} 
&
\begin{tikzpicture} [scale=0.014]
\draw(-36*3-0.05,0) -- (36*3+.05,0) (0,-36*3-0.05) -- (0,36*3+.05);
\foreach \x in {0, 1, -1, 2, -2}
	{
	\filldraw (0,0+\x*36) circle (3);
	\draw(0,0+\x*36) -- (36,\x*36+36);
	\filldraw [draw=black, fill=white] (36,\x*36+36) circle (3);
	}
\end{tikzpicture}
&
\begin{tikzpicture} [scale=0.014, rotate=45, yscale=-1, rotate=-45]
\draw(-36*2.5-0.05,0) -- (36*2.5+.05,0) (0,-36*2.5-0.05) -- (0,36*2.5+.05);
\foreach \x in {0, 1, -1, 2, -2}
	{
	\filldraw (0,0+\x*36) circle (3);
	\draw(0,0+\x*36) -- (36,\x*36+36);
	\filldraw [draw=black, fill=white] (36,\x*36+36) circle (3);
	}
\end{tikzpicture}
&
\begin{tikzpicture} [scale=0.07]
\draw (-20, 0) -- (20,0) (0,-20)--(0,20);
\draw[domain=-4.5:4.5,smooth,variable=\y]  plot ({\y*\y},{\y});
\end{tikzpicture}
&
\begin{tikzpicture} [scale=0.07, rotate=45, yscale=-1, rotate=-45]
\draw (-20, 0) -- (20,0) (0,-20)--(0,20);
\draw[domain=-4.5:4.5,smooth,variable=\y]  plot ({\y*\y},{\y});
\end{tikzpicture}
\end{tabular}
\end{center}

And here are those graphs again, this time pairing relations and their inverse relations.

\begin{center}
\begin{tabular}{ccc}
$P$, $P^{-1}$ & $A$, $A^{-1}$ & $G$, $G^{-1}$ \\ 
\begin{tikzpicture} [scale=0.1]
\draw(-0.5,0) -- (20.5,0) (0,-0.5) -- (0,20.5);
\def\pt{circle (0.09)}
\foreach \x in {(4,2),(6,2),(6,3),(8,2),(8,4),(9,3),(10,2),(10,5), (12, 2),(12,3),(12,4),(12,6), (14,2),(14,7),(15,3),(15,5),(16,2),(16,4),(16,8),(18,2),(18,9),(20,2),(20,4),(20,5),(20,10)}
	{	
	\filldraw \x \pt;
	\filldraw[ rotate=45, yscale=-1, rotate=-45] \x \pt;
	}
\end{tikzpicture}
&
\begin{tikzpicture} [scale=0.014]
\draw(-36*3-0.05,0) -- (36*3+.05,0) (0,-36*3-0.05) -- (0,36*3+.05);
\foreach \x in {0, 1, -1, 2, -2}
	{
	\filldraw (0,0+\x*36) circle (3);
	\draw(0,0+\x*36) -- (36,\x*36+36);
	\filldraw [draw=black, fill=white] (36,\x*36+36) circle (3);
	\filldraw[rotate=45, yscale=-1, rotate=-45] (0,0+\x*36) circle (3);
	\draw[rotate=45, yscale=-1, rotate=-45](0,0+\x*36) -- (36,\x*36+36);
	\filldraw [rotate=45, yscale=-1, rotate=-45, draw=black, fill=white] (36,\x*36+36) circle (3);
	}
\end{tikzpicture}
&
\begin{tikzpicture} [scale=0.07]
\draw (-20, 0) -- (20,0) (0,-20)--(0,20);
\draw[domain=-4.5:4.5,smooth,variable=\y]  plot ({\y*\y},{\y});
\draw[domain=-4.5:4.5,smooth,variable=\y, rotate=45, yscale=-1, rotate=-45]  plot ({\y*\y},{\y});
\end{tikzpicture}
\end{tabular}
\end{center}

\begin{task}
What do you notice about these graphs? Find a way to fold this page so that when you hold up the folded page to the light, the graphs of $P$ and $P^{-1}$ are lined up with each other. Find a way to fold the paper in this way for $A$ and $A^{-1}$, and then $G$ and $G^{-1}$. 
\end{task}

\begin{task}
When you fold the coordinate plane in this way, where does $(1,2)$ go? Where goes $(-4, 3)$ go? Where goes $(-100, -100001)$ go?

In general, when you fold the coordinate plane in this way, where does the point $(a,b)$ go? Why does this make sense?
\end{task}

\begin{task}
Explain why it makes sense that folding the plane this way should always bring a graph of a relation to the graph of its inverse. Do this in two ways:
	\begin{itemize*}
	\item First, explain this discovery and why it makes sense using $P$ and $P^{-1}$, including some specific examples. Make sure to use specific examples and also explain why no other fold will work.
	\item Then, explain this discovery and why it makes sense in general. Your explanation here should rely only on the definition of relation and its graph, and not depend on any particular examples. 
	\end{itemize*}
\end{task}

In teaching, it is often useful to be able to provide these two kinds of explanation: specific and general. Specific explanations have to do with a particular example and may help the students be able to keep an image in mind. General explanations help students understand why the reasoning for a specific example applies to a larger class of objects.

\newpage
\subsection{Summary}

\subsubsection*{Content}
This chapter had a lot going on! We defined relation in three different ways, which we called the middle school, high school, and university ways. We then talked about various properties of relations, such as its domain and range, as well as the image and preimage of points and subsets. We ended by talking about graphs of relations and equations.

Throughout this discussion, we saw algebraic, graphical, and cloud diagram ways of representing relations. 

We then discussed inverse relations and compositions of relations, which also can be understood in terms of these different representations. 
 
 Another common thread was Cartesian products, which is how we defined ordered pairs.  This allowed us to define relations the university way, and it also allowed us to talk meaningfully about graphs of relations.  The graph of a relation from $\R$ to $\R$ lives in the space given by the Cartesian product $\R\times \R$, otherwise known as $\R^2$. 
 
The two explorations we did tied together representations and the concepts we discussed:
	\begin{itemize}
	\item Given any relation $r$, we discovered that the graph of the relation $r^{-1}\circ r$ always contains all points of the form $(a,a)$ where $a$ is in the domain of $r$.
	\item Given any relation $r$, we discovered that the graph of the relation $r^{-1}$ can be obtained by reflecting (folding) the graph of $r$ about the line $y=x$.
	\end{itemize}
 
This last one may seem familiar: in high school we often teach this statement with the graph of functions. But as you learned, this statement applies to relations in general! You will examine this from a teaching perspective for homework, as well as finish the proofs of these explorations.

In the proofs, you will use the two proof structures we learned: 
\begin{itemize}
\item To show that a point $(x, y)$ is on a graph of a relation
means showing that the relation assigns $x$ to $y$.
\item 
To show that a point $(x, y)$ is on the graph of an equation means showing that evaluating the equation at $x$ and $y$ results in a true statement.
\end{itemize}
  
\subsubsection*{Connecting mathematically equivalent definitions}

\begin{itemize}
\item Explain how they appear different and how they appear the same.
\item Explain why they are mathematically equivalent. 
\item Analyze the mathematical and pedagogical purposes of each definition and why they may be appropriate for different levels of mathematical study, or why one version makes more sense after having worked with another version.
\end{itemize}

\subsubsection*{Connecting different mathematical representations}

\begin{itemize}
\item Explain how representations appear different and how they appear the same.
\item Identify how the representations can be used highlight different features. 
\item Go back and forth between the representations: where are the features of one representation located in the other representation?
\end{itemize}

For the teaching practices of connecting mathematically equivalent definitions and connecting different mathematical representations, it is helpful to be able to provide both specific and general explanations. 	\begin{itemize}
	\item Specific explanations have to do with a particular example and may help the students be able to keep an image in mind. 
	\item General explanations help students understand why the reasoning for a specific example applies to a larger class of objects.
	\end{itemize}
%%%%%%%%%%%%%%
% \header{Preparation for next lesson}

%%%%%%%%%%%%%%%%%%%%%%%%%%%%%%%%%%%%%%%%
%%%%%%%%% 2 IN-CLASS RESOURCES %%%%%%%%%%%%%%%%%
%%%%%%%%%%%%%%%%%%%%%%%%%%%%%%%%%%%%%%%%

\newpage \subsection{In-Class Resources}  

%%%%%%%%%% 2 OPENING EXAMPLE %%%%%%%%%%
\handout{Opening Example: Parent Relation}


We learned about natural number parents and children last time.

\begin{enumerate}
\item What is the definition of a parent of a natural number child? 
\item 
Let $P$ assign a natural number to each of its parents. We can represent $P$ as a set of arrows from $\N$ to $\N$.  Some arrows below have been filled in, for example $P$ assigns $6$ to $2$ and $3$, and assigns $12$ to $2, 3, 4, 6$.  Draw in more arrows.

\item Consider this statement: ``Some children have no parents, some children have exactly one parent, and some children have multiple parents.''  

Is this statement true or false? Why?

\item How about this statement? ``Some numbers have no children, and some numbers have multiple children.'' 
\end{enumerate}

\begin{center}
\begin{tikzpicture}
\def\pt{circle (0.05)}
\foreach \x in {0, 1, 2, 3, 4, 5, 6, 7, 8, 9, 10, 11, 12, 13, 14}
	{\filldraw(0,-0.5*\x) node[left]{$\x$} \pt;
	\filldraw(5,-0.5*\x) node[right]{$\x$} \pt;
	}
	\draw 
		(2.5, 1) node[]{$P$}
		(0,0.7) node[] {$\N$}
		(5,0.7) node[] {$\N$}
		(0, -0.5*15) node[]{$\vdots$}
		(5, -0.5*15) node[]{$\vdots$};
	\draw[->=stealth, yscale=-0.5] (0,6) -- (4.94, 1.9);
	\draw[->, yscale=-0.5] (0,6) -- (4.94, 2.9);
	\draw[->, yscale=-0.5] (0, 12)-- (4.94, 2.1);
	\draw[->, yscale=-0.5] (0, 12) -- (4.94, 3.1);
	\draw[->, yscale=-0.5] (0, 12) -- (4.94, 4);
	\draw[->, yscale=-0.5] (0, 12) -- (4.94, 6);
\draw[xshift=-1cm] (0,0) arc (180:0:1) (0,-7) arc (180:360:1) (0,0)-- (0,-7) (2,0) -- (2,-7); 
\draw[xshift=4cm] (0,0) arc (180:0:1) (0,-7) arc (180:360:1) (0,0)-- (0,-7) (2,0) -- (2,-7); 
\end{tikzpicture}
\end{center}

%%%%%%%%%% 2 EXAMPLES FOR RELATION %%%%%%%%%%
\newpage 

\handout{Getting familiar with relations and associated concepts}

\subsubsection*{Cartesian products}
(We will define $A$, $B$ $C$ on the board.)

\begin{enumerate}
\item 
	\begin{enumerate}
	\item  List the elements of the following Cartesian products:
	\begin{itemize}
	\item $A\times B$ 
	\item $A\times C$
	\item $\Z \times C$
	\item $C\times \Z$
	\item $\N \times \N$.
	\end{itemize}

	\item Which of the above sets contains the element $(6, -1)$? How about $(-1, 10)$?

	\item How would you describe $\R\times \R$? 

	\item How about $\R\times (\R\times \R)$?
	\end{enumerate}
\end{enumerate}

\vfill 
\subsubsection*{Domain, range, image, preimage}

Do \#2. Do not proceed to \#3 or \#4 yet.

\begin{enumerate}[resume]
\item \begin{enumerate}
	\item What are the domain and range of the car-owner relation and the room-course relation? 

	\item Suppose this table contains course assignments to rooms at 1pm. What is the image of Math Bldg Room 100? What is the preimage of Math 996, Math 405, Math 100, and Math 221 under the room-course relation? 

    \begin{center}
    \begin{tabular}{|c|c|}
    \hline
    Room & Course in room at 1pm \\ \hline
    Math Bldg room 100 & Math 996 \\ 
    Math Bldg room 104 & Math 100 \\ 
    Engineering Bldg room 750 & Math 405 \\ 
    not being offered this term & Math 221 \\ \hline
    \end{tabular}
    \end{center}

\item Let $T$ be the relation that maps each day of the year 2030 to the its average temperature in $\degrees F$ that day. Describe a possible candidate domain, domain, candidate range, and range of this relation.

% connection to trig
\item Let $A$ be the relation that maps each degree in the interval $[0\degrees, 360\degrees)$ to all degrees in the interval $(-\infty,\infty)$ that give an equivalent angle measure. What is the preimage of $361\degrees$? What is the image of $0\degrees$? 

% connection to geometry
\item Let $\rho$ be the relation that maps a point in the plane to its rotation about the origin by $90\degrees$.  (This means $90\degrees$ {\it counter}clockwise.) What is the image of the point $(1,0)$? What is the preimage of the point $(-2,0)$? 

% connection to graphing 
\item Let $G$ be the relation that maps $x$ to every $y$ such that $x=y^2$. What is the image of $4$? What is the preimage of $-6$?
\end{enumerate}
\end{enumerate}

\newpage
\subsubsection*{Connecting different representations}
\begin{enumerate}[resume]
\item  Interpret the definitions of candidate domain, domain, image, preimage, candidate range, and range in terms of arrows and their start and end points.
\item Interpret the definitions of these same concepts in terms of the graph of a relation. Use the graphs of $A$ and $P$ to illustrate what you mean.
\end{enumerate}


\vfill
{\it Mathematical/teaching practice: Connecting different representations}
\begin{itemize*}
\item Explain how representations appear different and how they appear the same.
\item Identify how the representations can be used highlight different features. 
\item Go back and forth between the representations: where are the features of one representation located in the other representation?
\end{itemize*}

\newpage 
\handout{Getting familiar with inverses of relations}

\subsubsection*{Back to the opening example: Inverting the parent relation}
\begin{minipage}{4in} \raggedright \parskip4pt
The inverse of the parent relation could be represented like this:

What other arrows does the inverse of the parent relation contain? 

What might be a good name for this relation?
\end{minipage}
\begin{minipage}{2.4in}
 \begin{center}
\begin{tikzpicture}[xscale=-1, scale=0.7]
\def\pt{circle (0.05)}
\foreach \x in {0, 1, 2, 3, 4, 5, 6, 7, 8, 9, 10, 11, 12}
	{\filldraw[yscale=0.5](0,-\x) node[left]{\tiny{ $\x$}} \pt;
	\filldraw[yscale=0.5](5,-\x) node[right]{\tiny{$\x$}} \pt;
	}
	\draw 
		(2.5, 1) node[]{$P$}
		(0,0.7) node[] {$\N$}
		(5,0.7) node[] {$\N$}
		(0, -6.5) node[]{$\vdots$}
		(5, -6.5) node[]{$\vdots$}
		(2.5, -5) node[]{$\vdots$};
	\draw[<-=stealth, yscale=-0.5] (0,4) -- (4.94, 2);
	\draw[<-, yscale=-0.5] (0,6) -- (4.94, 3);
	\draw[<-, yscale=-0.5] (0,6) -- (4.94, 2);
	\draw[<-, yscale=-0.5] (0,8) -- (4.94, 2);
	\draw[<-, yscale=-0.5] (0,8) -- (4.94, 4);
	\draw[<-, yscale=-0.5] (0,9) -- (4.94, 3);
	\draw[<-, yscale=-0.5] (0,10) -- (4.94, 3);
	\draw[<-, yscale=-0.5] (0,10) -- (4.94, 5);
	\draw[<-, yscale=-0.5] (0, 12)-- (4.94, 2);
	\draw[<-, yscale=-0.5] (0, 12) -- (4.94, 3);
	\draw[<-, yscale=-0.5] (0, 12) -- (4.94, 4);
	\draw[<-, yscale=-0.5] (0, 12) -- (4.94, 6);
\draw[xshift=-1cm, yscale=0.75] (0,0.5) arc (180:0:1) (0,-9) arc (180:360:1) (0,0.5)-- (0,-9) (2,0.5) -- (2,-9); 
\draw[xshift=4cm, yscale=0.75] (0,0.5) arc (180:0:1) (0,-9) arc (180:360:1) (0,0.5)-- (0,-9) (2,0.5) -- (2,-9); 
\end{tikzpicture}
\end{center}
\end{minipage}



\subsubsection*{Connecting different but mathematically equivalent definitions}
Discuss the three versions of the definition of inverse of a relation. What do they each say? How would you represent them? What makes them mathematically equivalent?

\vfill 
{\it Mathematical/teaching practice: Connecting different but mathematically equivalent definitions}
\begin{itemize}
\item Explain how they appear different and how they appear the same.
\item Explain why they are mathematically equivalent. 
\item Analyze the mathematical and pedagogical purposes of each definition and why they may be appropriate for different levels of mathematical study, or why one version makes more sense after having worked with another version.
\end{itemize}

\newpage
\subsubsection*{Working with compositions algebraically}
Let $P$ be the parent relation and let $t:\N\to \N, x\mapsto 2x$. 

\begin{enumerate}
\item \begin{enumerate}
	\item For the relation $t\circ P$, what is the image of $6$? Of $12$? Of $9$?

	\item For the relation $t\circ P$, what is the preimage of $4$? Of the set $\{1, 3, 5\}$? Of the set $\{4,14\}$?
	
	\item What are the domain and range of the relations $t\circ P$?

	\end{enumerate}
	\bigskip
	\item How would you represent $P\circ t$? What are its domain and range?
\end{enumerate}

\vfill 

\begin{enumerate}[resume]
\item  \begin{enumerate}
	\item Sketch a representation of $A^{-1}\circ A$. 
	\item What are its domain and range? 
	\item What is the image of $45\degrees$? 
	\item What is the preimage of $45\degrees$? 
	\end{enumerate}
\end{enumerate}	


\vfill 

We often think about inverses as ``undoing'' something. How well does this analogy work in the case of relations? What goes well with the analogy? What goes wrong with the analogy?

\vfill 
%%%%%%%%%2 GRAPHS OF RELATIONS %%%%%%%%%%%%
\newpage
\handout{Graphs of relations}

\begin{mdframed}
{\bf Proof structures:}

To show that a point $(a, b)$ is on a graph of a relation means showing that the relation assigns $a$ to $b$.

To show that a point $(x, y)$ is on the graph of an equation means showing that evaluating the equation at $x=a$ and $y=b$ results in a true statement.
\end{mdframed}

\vfill 
\begin{enumerate}
\item What are all the points on the graph of $A$ with $x$-coordinate $45\degrees$? With $y$-coordinate $45\degrees$?

\vfill 
\item What are all the points on the graph of $A^{-1}$ with $x$-coordinate $60\degrees$? With $y$-coordinate $60\degrees$?
 
Is the point $(60, 400)$ on the graph of $A^{-1}$? How about $(430, 70)$? $(70, 430)$? (10, 200)? Why or why not?
  
What is the graph of $A^{-1}$? How do you know you have graphed all points and not graphed any extra points?

\vfill
\item Show that for any relation $r:D\to D$, if $x\in D$ is in the domain of $r$, then $(x,x)$ lies on the graph of $r^{-1}\circ r$. 

\vfill

\item Is the point $(1,2)$ on the graph of $x^2+\frac{y^2}{4}=1$?  How about the point $(\frac{1}{2},\frac{2}{3})$? 

Find a point that is on the graph of $y=x^2+1$ and the graph of $y=6x-4$. Is that the only point? Are there more points? How do you know?

Describe the graph of the equation $x=0$. Describe the graph of the equation $y=0$. How do you know that these graphs look this way?

Where does $x^2+\frac{y^2}{4}=1$ intersect the $x$-axis? Where does $x^2+\frac{y^2}{4}=1$ intersect the $y$-axis?

What is a point that is on the graph of $y=x^2+1$ and the graph of $y=5$. Then solve for $x$ when in $y=x^2+1$ when $y$ is $5$. Explain the numerical coincidence.
\vfill 

\item What are all the $x$-intercepts of the graphs of $A$ and $A^{-1}$?

What are all the $y$-intercepts of the graphs of $A$ and $A^{-1}$?

What is the intersection of the graphs $A$ and $A^{-1}$?
\end{enumerate}
\vfill

\newpage
\handout{Closing inquiry}

To finish this chapter, let's investigate the examples of relations and their inverses.  Here are some graphs that we've seen: 

\begin{center}
\begin{tabular}{ccc}
$P$, $P^{-1}$ & $A$, $A^{-1}$ & $G$, $G^{-1}$ \\ 
\begin{tikzpicture} [scale=0.1]
\draw(-0.5,0) -- (20.5,0) (0,-0.5) -- (0,20.5);
\def\pt{circle (0.09)}
\foreach \x in {(4,2),(6,2),(6,3),(8,2),(8,4),(9,3),(10,2),(10,5), (12, 2),(12,3),(12,4),(12,6), (14,2),(14,7),(15,3),(15,5),(16,2),(16,4),(16,8),(18,2),(18,9),(20,2),(20,4),(20,5),(20,10)}
	{	
	\filldraw \x \pt;
	\filldraw[ rotate=45, yscale=-1, rotate=-45] \x \pt;
	}
\end{tikzpicture}
&
\begin{tikzpicture} [scale=0.014]
\draw(-36*3-0.05,0) -- (36*3+.05,0) (0,-36*3-0.05) -- (0,36*3+.05);
\foreach \x in {0, 1, -1, 2, -2}
	{
	\filldraw (0,0+\x*36) circle (3);
	\draw(0,0+\x*36) -- (36,\x*36+36);
	\filldraw [draw=black, fill=white] (36,\x*36+36) circle (3);
	\filldraw[rotate=45, yscale=-1, rotate=-45] (0,0+\x*36) circle (3);
	\draw[rotate=45, yscale=-1, rotate=-45](0,0+\x*36) -- (36,\x*36+36);
	\filldraw [rotate=45, yscale=-1, rotate=-45, draw=black, fill=white] (36,\x*36+36) circle (3);
	}
\end{tikzpicture}
&
\begin{tikzpicture} [scale=0.07]
\draw (-20, 0) -- (20,0) (0,-20)--(0,20);
\draw[domain=-4.5:4.5,smooth,variable=\y]  plot ({\y*\y},{\y});
\draw[domain=-4.5:4.5,smooth,variable=\y, rotate=45, yscale=-1, rotate=-45]  plot ({\y*\y},{\y});
\end{tikzpicture}
\end{tabular}
\end{center}

What do you notice about these graphs? Find a way to fold this page so that when you hold up the folded page to the light, the graphs of $P$ and $P^{-1}$ are lined up with each other. Find a way to fold the paper in this way for $A$ and $A^{-1}$, and then $G$ and $G^{-1}$. 

\vfill 
When you fold the coordinate plane in this way, where does $(1,2)$ go? Where goes $(-4, 3)$ go? Where goes $(-100, -100001)$ go?

In general, when you fold the coordinate plane in this way, where does the point $(a,b)$ go? Why does this make sense?

\vfill
Explain why it makes sense that folding the plane this way should always bring a graph of a relation to the graph of its inverse. Do this in two ways:
	\begin{itemize*}
	\item First, explain this discovery and why it makes sense using $P$ and $P^{-1}$, including some specific examples. Make sure to use specific examples and also explain why no other fold will work.
	\item Then, explain this discovery and why it makes sense in general. Your explanation here should rely only on the definition of relation and its graph, and not depend on any particular examples. 
	\end{itemize*}
\vfill 

%%%%%%%%%% 2 DEFINITIONS RELATED TO RELATION %%%%%%%%%%
\newpage 
\handout{Reference: Relations}

{\bf Definition \ref{d: cartesian product}.}
Let $D$ and $R$ be sets. The \emph{Cartesian product} of $D$ and $R$ is defined as the set of ordered pairs $\{ (x,y) \st x\in D, y\in R\}$. It is denoted $D\times R$.

\vfill 
{\bf Definitions \ref{d: relation middle school}, \ref{d: relation high school}, \ref{d: relation university}:} \\ 
\begin{tabular}{L{1.25in}|L{5in}}
\hline 
Middle school version  & A \emph{relation} from a set $D$ to a set $R$ is a set of arrows going from elements of $D$ to elements of $R$.   \\  \hline
High school version & A \emph{relation} $P$ from a set $D$ to a set $R$ a set of assignments from elements of $D$, called inputs, to elements of $R$, called outputs. \\  \hline
University version & A relation $r:D\to R$ is a subset of $D\times R$, i.e., $r\subset D\times R$. \\  \hline
\end{tabular}
{\bf Notation:} $r:D\to R$ refers to a relation from $D$ to $R$; and $x\mapsto y$ refers to an assignment from $x\in D$ to $y\in R$

A relation may map an element of $D$ to 0, exactly 1, or multiple elements of $R$.  An element of $R$ may have 0, exactly 1, or multiple elements of $D$ mapping to it.

\vfill 
{\bf Definition \ref{d: candidate domain etc}.} 
For a relation $r:D \to R$, we say that 
\vspace*{-4pt}
\begin{itemize}
	\item $D$ is the \emph{candidate domain};
	\item $R$ is the \emph{candidate range} (or \emph{codomain});
	\item the \emph{image} of an element $x\in  D$ is the set of elements of $R$ that $x$ is mapped to. Similarly, the \emph{image} of a subset of $S\subset D$ is the subset of $R$ containing the images of all elements of $S$.
	\item When a element in $D$ maps to no element in $R$, we say it has \emph{empty image}. Otherwise it has \emph{nonempty image}.
	\item the \emph{domain} of $r$ is the subset $D'\subset D$ of elements with nonempty image.
	\item the \emph{preimage} of an element of $R$ is the set of elements of $D$ that map to $R$. Similarly, the \emph{preimage} of a subset $T\subset R$ is the subset of $D$ containing the preimages of all elements of $T$.
	\item When a element in $R$ has no element in $D$ mapping to it, we say it has \emph{empty preimage}. Otherwise it has \emph{nonempty preimage}.
	\item the \emph{range} (or \emph{image}) of the relation $r$ is the subset $R\subset D$ with nonempty preimage.
	\end{itemize}

\vfill 

{\bf Definitions \ref{d: inverse relation middle school}, \ref{d: inverse relation high school}, \ref{d: inverse relation university}:}  \\ 
\begin{tabular}{L{1.25in}|L{5in}}
\hline 
Middle school version & 
If $r$ is a relation from a set $D$ to $R$, then the \emph{inverse relation} of $r$ is the relation that swaps the direction of the arrows of $r$.  The arrows of the inverse go from elements of $R$ to elements of $D$. \\ \hline 
High school version & 
Given a relation $r:D\to R$, the \emph{inverse relation} of $r$ is the set of assignments $y\mapsto x$ such that $x\mapsto y$ is an assignment of $r$. \\ \hline  
High school version & 
Given a relation $r:D\to R$, the \emph{inverse relation} of $r$ is defined as the subset 
	$\{(y, x) \subset R\times D \st (x,y)\in r\}.$\\ \hline
\end{tabular}
{\bf Notation: $r^{-1}$.}


\vfill 

{\bf Definition \ref{d: relation composition}.}
Given two relations $P:D\to D$ and $Q:D\to D$, we define the \emph{composition} of $P$ then $Q$ 
as the relation that assigns $x$ to $z$ whenever there is a $y\in D$ such that $P$ assigns $x\mapsto y$ and $Q$ assigns $y\mapsto z$.

\vfill
{\bf Definition \ref{d: graph of a relation}.}
The \emph{graph of a relation} $r:D\to R$ is defined as the set of points $(a, b)\in \R^2$ such that $r$ assigns $a\mapsto b$.

{\bf Definition \ref{d: graph of an equation}.}
The \emph{graph of an equation} in $x$ and $y$ is defined as the set of points $(a, b) \in \R^2$ such that evaluating the equation at $x=a$ and $y=b$ results in a true statement.

\vfill

{\bf Definition \ref{d: intercepts}.}
Given an equation in $x$ and $y$ and its graph, all points $(a,0)$ on the graph are called \emph{$x$-intercepts} of the graph. All points $(0,b)$ on the graph are called \emph{$y$-intercepts} of the graph. A graph may have 0, 1, or multiple $x$-intercepts and $y$-intercepts.

\vfill
{\bf Definition \ref{d: intersection}.}
Two graphs \emph{intersect} each other at $(a,b)$ when $(a,b)$ is contained in both graphs.  Graphs may intersect at zero points, one point, multiple points, and sometimes even infinitely many points.

\vfill
{\bf Proof structures:}
\begin{itemize}
\item To show that a point $(a, b)$ is on a graph of a relation means showing that the relation assigns $a$ to $b$.

\item To show that a point $(x, y)$ is on the graph of an equation means showing that evaluating the equation at $x=a$ and $y=b$ results in a true statement.
\end{itemize}

\vfill 

%%%%%%%%% 2 HOMEWORK %%%%%%%%%%%%
\newpage \subsection{Homework}

\begin{bignote}
The homework questions used in the next chapter are \ref{h: key graphs}, \ref{h: composition correspondence}, and \ref{h: function definition}. 

Among the remaining questions, we recommend assigning one or two parts of each of a subset of the questions, depending on your perception of what your students would most benefit from. If you assign part of question 0, you might use google forms or another online submission so that you can read over responses prior to the next time you see your class. 

When grading the proof questions, we have often awarded full credit only to proofs that are both correct and that follow the proof communication guidelines established in Chapters 0 and 1. 

The purpose of the problems here are as follows:
	\begin{enumerate}
	\item definitions;  connecting different yet mathematically equivalent definitions.
	\item inverse and graph; used in next chapter.
	\item composition and graph; used in next chapter.
	\item (watching and responding to video of an experienced high school teacher's instruction) intercept; connecting different representations (graphical and algebraic).
	\item intercept; connecting different representations (graphical and algebraic); followup to previous problem.
	\item intersection of graphs
	\item showing that a point $(x,y)$ is on the graph of a relation
	\item showing that a point $(x,y)$ is on the graph of an equation
	\item showing that a point $(x,y)$ is on the graph of a relation; showing that a point $(x,y)$ is on the graph of an equation. (Variations on this question can be created by substituting in the example function with another invertible function, and making sure that the examples of points provided lie on the graph of that example function.)
	\item showing that a point $(x,y)$ is on the graph of an equation; connection to calculus. 
	\item definition of function; connecting different representations; used in next chapter.
	\end{enumerate}
\end{bignote}

\begin{enumerate}
\setcounter{enumi}{-1}
\item In this chapter, we learned about:
 	\begin{itemize}
	\item Defining:
		\begin{itemize*} 
		\item Relation, in three equivalent ways; candidate domain, domain, candidate range, range, preimage of a point and of a subset, image of a point and of a subset
		\item Inverse relation
		\item Composition of relations
		\item Graph of a (real) relation; $x$- and $y$-intercepts; intersection of graphs
		\end{itemize*}
	\item Showing that a point $(x,y)$ is on the graph of a relation
	\item Showing that a point $(x,y)$ is on the graph of an equation
	\item The mathematical/teaching practices of:
		\begin{itemize*}	
		\item Connecting mathematically equivalent definitions
		\item Connecting different mathematical representations
		\end{itemize*}
	\end{itemize}

For each of these ideas: 
	\begin{enumerate}
	\item Where in the text are these ideas located?
	\item Review this section of the text. What definitions and results were important? How do examples use these definitions and results?
	\item What questions or comments do you have about the ideas in this section?
	\end{enumerate}
	
	
\item 
% definitions 
% focal practice: connecting different yet mathematically equivalent definitions
Describe the following concepts in terms of the middle school and university versions of the definition of relation: 
		\begin{enumerate}
		\item candidate domain, domain
		\item image of a point, image of a subset
		\item candidate range, range
		\item preimage of a point, preimage of a subset
		\item $x$- and $y$-intercepts 
		\item intersection of the graphs of two relations
		\item inverse relation
		\item composition of two relations
		\end{enumerate}  
		
		In your work for each, 
		\begin{enumerate}[label=(\roman*)]
		\item address both the middle school and university versions of the definition of relation
		\item include diagrams
		\item describe the concepts using a specific example, using a relation such as the parent relation $P$ or angle relation $A$ as defined in Example \ref{ex: relations}. 
		\item then phrase the descriptions so that they can apply without any changes to any possible example of a relation
		\item explain why the descriptions in terms of both versions are mathematically equivalent
		\end{enumerate}
\end{enumerate}

In questions \ref{h: key graphs},  \ref{h: composition correspondence}, and \ref{h: reflect over y=x}, we examine relations which are functions. This means that each input in the domain is assigned to only one output element, so the notation $f(x)$ makes sense: there is only one element that $f$ sends $x$ to. We will also examine this definition more in question \ref{h: function definition}.

\begin{enumerate}[resume]
\item \label{h: key graphs} 
% concept of inverse
% use in next chapter 
	\begin{enumerate}
	\item Graph the following and their inverse relations. (You may need to look up the relations in a high school resource online or elsewhere.) We will use these examples in the next chapter.
		\begin{itemize}
		\item sine function ($x\mapsto \sin(x)$ for $x\in \R$)
		\item cosine function ($x\mapsto \cos(x)$ for $x\in \R$)
		\item absolute value function ($\tn{Abs}: \R\to\R, x\mapsto |x|$)
		\item squaring function  ($\tn{Sq}: \R\to\R, x\mapsto x^2$)
		\item cubing function ($\tn{Cu}: \R\to\R, x\mapsto x^3$)
		\end{itemize}	
	\item In each of the graphs of the given functions, label any maxima or minima with its coordinate. In the graphs of their inverses, label any points that are leftmost or rightmost with its coordinate. Use exact form (this means to not use approximations, including decimal approximations; use symbols such as $\pi$ as needed).
	\item What are the following images and preimages?
		\begin{enumerate}
		\item What is the image of $[0,\pi]$ for the sine function? How about for the cosine function?
		\item What is the preimage of $[0.5,1]$ for the sine function? How about for the cosine function?
		\item What is the image of $[0.5,1]\cup[-3,-1]\cup[3,\infty]$ for Abs?
		\item What is the preimage of $[100, 144]$ for Abs?
		\item What is the image of $[0.5,1]\cup[-3,-1]\cup[3,\infty]$ for Sq?
		\item What is the preimage of $[100, 144]$ for Sq?
		\item What is the image of $[0.5,1]\cup[-3,-1]\cup[3,\infty]$ for Cu? (for this question, you may use approximations)
		\item What is the preimage of $[100, 144]$ for Cu? (for this question, you may use approximations)
		\end{enumerate}
	\end{enumerate}

\item \label{h: composition correspondence} 
% composition 
% focal practice: connecting different representations (here, graphical and algebraic)
% use in next chapter
Below are graphs of the relations $f, g, h$. The pieces of these graphs are lines and line segments, and their  turning points are integer coordinate points. Consecutive tick marks on the axes are distance 1 from each other.
	\begin{enumerate}
	\item Find the values of $g\circ f(-4), \quad g\circ f(-1), \quad g\circ f(0), \quad g\circ f(3), \quad g\circ f(4)$. 
	\item Where does $h\circ g$ map each of $0, 2, 4$?
	\item Where does $g\circ h$ map each of $0, 2, 4$?
	\end{enumerate}
\end{enumerate}

\begin{center}
\begin{tikzpicture}[scale=0.5]
\draw (0,-2.5) -- (0, 3.5) node[above]{$y$};
\draw (-6.25,0) -- (8.25,0) node [right]{$x$};
\foreach \x in {-6,-5,-4,...,8}
	\draw (\x,-0.1) -- (\x, 0.1);
\foreach \x in {-2,-1, ..., 3}
	\draw (-0.1,\x) -- (0.1,\x);
\draw[xshift=1cm, ultra thick, red] (-2,-2) -- (2,2);
\draw[ultra thick, red] (3,2) -- (5,2) node[above]{$f$} -- (8,2);
\draw[ultra thick, red] (-1,-2) -- (-6,-2);
\draw[yshift=1cm, ultra thick, dashed, blue] (-2,2) -- (2,-2);
\draw[ultra thick, dashed, blue] (2,-1) -- (5,-1)node[below]{$g$}-- (7.5,-1);
\draw[ultra thick, dashed, blue] (-2,3) -- (-5.5,3); 
\end{tikzpicture}
%
\hspace*{24pt}
%
\begin{tikzpicture}[scale=0.5]
\draw (0,-2.5) -- (0, 3.5) node[above]{$y$};
\draw (-6.25,0) -- (8.25,0) node [right]{$x$};
\foreach \x in {-6,-5,-4,...,8}
	\draw (\x,-0.1) -- (\x, 0.1);
\foreach \x in {-2,-1, ..., 3}
	\draw (-0.1,\x) -- (0.1,\x);
\draw[ultra thick, green] (-6,-2) -- (6,2) node[above]{$h$}  (-6,2) -- (6,-2) ;
\end{tikzpicture}
\end{center}

\begin{enumerate}[resume]
\item % intercept
% focal practice: connecting different representations
\label{h: barbara shreve video}
In this task, you will watch a short video of teaching by Ms.~Barbara Shreve of San Lorenzo High School. The video shows her teaching an intervention class called Algebra Success. The students in this class have been previously unsuccessful in Algebra 1. They are working on finding intercepts of equations to get ready for working with quadratics. 

As you watch the video, it may be tempting to think about what you personally think is good or not as good about the teaching, or what you might have done differently. But before getting to these kinds of judgments, it is more important to simply observe what is going on, what the students' reasoning is, and what the story line is. (This is just like when working with students, as we will see later in this class and you will learn in your methods class: before evaluating students' work, we must first observe and understand students' work without judgment.) Here, we will practice observing the interactions between teachers and 

As you watch the video, think about the following questions:
	\begin{enumerate}
	\item How does the teacher emphasize to students to explain their reasoning?
	\item How does the teacher help students feel comfortable sharing their reasoning?
	\item How was the definition of $x$-intercept or $y$-intercept used?
	\end{enumerate}

Here is a link to the video:
\url{http://www.insidemathematics.org/classroom-videos/public-lessons/9th-11th-grade-math-quadratic-functions/introduction-part-b}

Talk through your responses to the discussion questions with a fellow teacher in our class. As you talk, be sure to point to the evidence that you are basing your ideas on. Then write down your response to these discussion questions. Describe the evidence you are using by selecting quotes from the transcript (found on the video site).

\item % intercept
% focal practice: connecting different representations (here, graphical and algebraic)
(This task can be thought of as a followup to question \ref{h: barbara shreve video}.)

Suppose that you are teaching about intercepts of graphs and you are going over a solution to the problem: {\it Find the $y$-intercept of the graph of the equation $y=(x-3)^2$.} Your class has the following conversation:
	
		\begin{quote}
		You: How did you start this problem? \\ 
		Student A: I put in a 0 for the $x$-value. \\ 
		You: Let's talk about what Student A did. If we're finding a $y$-intercept, why do we start by putting a 0 for the $x$?  Anyone have an idea? \\
		Student B: Because zero's the easiest thing. \\ 
		Student C: Because zero is where the line crosses. \\
		Student B: Wait, it's because you want to cancel it out. 
		\end{quote}
	
	\begin{enumerate}
	\item Solve the problem that the class is working on.
	\item Explain the logic of student B's thinking.
	\item Explain the logic of student C's thinking.
	\item In the equivalent of at most 4 tweets, explain why `` If we're finding a $y$-intercept, we start by putting a 0 for the $x$.''. (A tweet is 140 characters.) Your explanation should tie together the definition of graph and the definition of $y$-intercept.
	\item What questions might you pose to students to get at the ideas in this explanation?
	\end{enumerate}

\item % intersection of graphs 
Suppose you are teaching about how to find the intersection of graphs of quadratic and linear equations, and you plan for your class to work on this exploratory task: 
	\begin{quote}
	{\it How many points do the graphs of $y=x^2$ and $y=x$ intersect at? What are those points?}
	
	{\it How many points do the graphs of $y=x^2+a$ and $y=x$ intersect at? What are those points? How does the answer depend on $a$?}
	\end{quote}

An important initial step of planning to teach a task is to solve that task yourself. 
	\begin{enumerate}
	\item Keeping in mind what we have learned about satisfying answers to mathematical questions: What is a good answer to this task? Describe your conjecture.
	\item Prove your conjecture. 
	\end{enumerate}
	
Another step in planning is to figure out how you might explain how key ideas are used to solve the task.
	\begin{enumerate}[resume]
	\item Explain as you would to high school students in this class how the definition of intersection of graph is used in finding solutions to the exploration.
	\end{enumerate}
	
\item 
% showing that a point $(x,y)$ is on the graph of a relation
Show that for any relation $r:D\to D$, if $x\in D$ is in the domain of $r$, then $(x,x)$ lies on the graph of $r^{-1}\circ r$. 

\item 
% showing that a point $(x,y)$ is on the graph of an equation
(Based on Chazan (1993)\footnote{Chazan, D. (1993). F(x)=G(x)?: An approach to modeling with algebra. {\it For the Learning of Mathematics, 13}, 22-26.}). Suppose you are teaching algebra and a student asks, ``
Why do we call `$x$' a variable in equations like $6x+5 = 10$ when it stands for just one
number?'' 

	\begin{enumerate}
	\item What is the student thinking? How might they have arrived at this question?
	\item What are you sure that the student understands? What are you unsure that the student understands?
	\item What is the mathematical issue here?
	\item How do you respond?
	\end{enumerate}
	
\item 
% showing that a point $(x,y)$ is on the graph of a relation
% showing that a point $(x,y)$ is on the graph of an equation
% note that variations on this question can be created by substituting in the example function with another invertible function, and making sure that the examples of points provided lie on the graph of that example function.
\label{h: reflect over y=x}
Suppose that you are introducing the idea that one way to obtain the graph of an inverse of a relation is to reflect the graph of the relation over the line $y=x$. 
	\begin{enumerate}
	\item To help students understand this reflection, you use examples such as ``Where would the point $(3,0)$ go if you reflected it about the line $y=x$? What about the point $(0, 15)$? $(3,15)$?'' 	You then follow up with other examples.
	 
	 Using the examples $(3,0)$, $(0,15)$, and $(3,15)$, explain geometrically why it makes sense that point $(3,0)$ is mapped to the point $(0,3)$ by reflection over the line $y=x$, and similarly for $(0, 15)$ to $(15,0)$, and $(3, 15)$ to $(15, 3)$.
	 
	 Then explain, why, in general, the point $(a,b)$ is mapped to the point $(b,a)$ when reflected over the line $y=x$. This part of your explanation should apply to any possible example of a point without referring specifically to any examples.

	\item You then use the example of $f(x)=5x$. What is the equation of its inverse relation $f^{-1}$?
	
	\item Using the example of $f(x)=5x$, explain why it makes sense that  $f^{-1}$ must have the graph of the equation you found in (b). In your explanation, draw on the definition of inverse relation, the definition of graph of a relation, and the definition of graph of an equation, as well as your general explanation in part (a). Be explicit about where you use these definitions.
	
	\item Then explain why, in general, it is true that one way to obtain the graph of an inverse of a relation is to reflect the graph of the relation over the line $y=x$. 
	\end{enumerate}
		
\item 
% showing that a point $(x,y)$ is on the graph of an equation
% connection to calculus
We can think of definite integrals such as $\int_{0}^{x} t^2 \,dt$ as a function with input variable $x$.  

Using the definition of graph of an equation, explain why when you shift every point on the graph of $y=\int_{0}^{x} t^2 \,dt$ down by $\frac{1}{3}$, you obtain the graph of $y=\int_{1}^{x} t^2 \,dt$.

\item 
% getting ready for next week
% definition of function 
\label{h: function definition}
Read the following definition:

{\bf Definition.} A {\it function} from $D$ to $R$ is defined as a relation from $D$ to $R$ where each input in $D$ is assigned to no more than one output in $R$.

Complete the following table by placing checkmarks to indicate: which of the relations below are also functions?

\begin{center}
\begin{tabular}{C{1in}|C{0.5in}|C{0.5in}|C{0.5in}|C{0.5in}|C{0.5in}|C{0.5in}|}
	& $r$ & $r^{-1}$ & $s$ & $s^{-1}$ & $t$ & $t^{-1}$ \\ \hline
is a function & &&&&& \\ \hline
is NOT a function &&&&&& \\ \hline 
\end{tabular}
\end{center}
\end{enumerate}

\bigskip

\begin{tabular}{C{2in}C{2in}C{2in}}
$r$ & $s$ & $t$ \\ 
%r
\begin{tikzpicture}
% [cloud diagram r - is not a function, inverse is not a function]
	\def\pt{circle (0.05)}
	\foreach \x in {0, 1, 2}
	{\filldraw(0,-0.5*\x)  \pt;
	\draw[->] (0,-0.5*\x) -- (2.94, -0.5-0.03*\x+0.03);
	\draw[->] (0,-0.5*\x) -- (2.94, -1.5-0.03*\x+0.03);
	}
	\foreach \y in {0, 1, 2, 3}
	{\filldraw(3,-0.5*\y) \pt;}
	\draw[xshift=-0.5cm] (0,0) arc (180:0:0.5) (0,-1.5) arc (180:360:0.5) (0,0)-- (0,-1.5) (1,0) -- (1,-1.5); 
	\draw[xshift=2.5cm] (0,0) arc (180:0:0.5) (0,-1.5) arc (180:360:0.5) (0,0)-- (0,-1.5) (1,0) -- (1,-1.5); 
\end{tikzpicture}
&
\begin{tikzpicture}
% [cloud diagram s - is a function and its inverse is a function]
	\def\pt{circle (0.05)}
	\foreach \x in {0, 1, 2}
	{\filldraw(0,-0.5*\x)  \pt;}
	\foreach \y in {0, 1, 2, 3}
	{\filldraw(3,-0.5*\y) \pt;}
	\draw[xshift=-0.5cm] (0,0) arc (180:0:0.5) (0,-1.5) arc (180:360:0.5) (0,0)-- (0,-1.5) (1,0) -- (1,-1.5); 
	\draw[xshift=2.5cm] (0,0) arc (180:0:0.5) (0,-1.5) arc (180:360:0.5) (0,0)-- (0,-1.5) (1,0) -- (1,-1.5); 
	%\draw[->] (0,-0.5*\x) -- (2.94, -0.5*\y)
	\draw[->] (0,-0.5*0) -- (2.94, -0.5*3);
	\draw[->] (0,-0.5*2) -- (2.94, -0.5*1);
\end{tikzpicture}
&
\begin{tikzpicture}
% [cloud diagram t - inverse is a function but it is not itself a function]
	\def\pt{circle (0.05)}
	\foreach \x in {0, 1, 2}
	{\filldraw(0,-0.5*\x)  \pt;}
	\foreach \y in {0, 1, 2, 3}
	{\filldraw(3,-0.5*\y) \pt;}
	\draw[xshift=-0.5cm] (0,0) arc (180:0:0.5) (0,-1.5) arc (180:360:0.5) (0,0)-- (0,-1.5) (1,0) -- (1,-1.5); 
	\draw[xshift=2.5cm] (0,0) arc (180:0:0.5) (0,-1.5) arc (180:360:0.5) (0,0)-- (0,-1.5) (1,0) -- (1,-1.5);
	%\draw[->] (0,-0.5*\x) -- (2.94, -0.5*\y)
	\draw[->] (0,-0.5*0) -- (2.94, -0.5*3);
	\draw[->] (0,-0.5*2) -- (2.94, -0.5*1); 
	\draw[->] (0,-0.5*1) -- (2.94, -0.5*2); 
	\draw[->] (0,-0.5*2) -- (2.94, -0.5*0); 
\end{tikzpicture}
\end{tabular}



%%%%%%%%%%%%%%%%%%%%%%%%%%%%%%%%
%%%%%%%%%%%%%%%%%%%%%%%%%%%%%%%% 	
%%%%%% LESSONS 3-4 %%%%%%%%%%%%%%%%%%%%	
%%%%%%%%%%%%%%%%%%%%%%%%%%%%%%%%
%%%%%%%%%%%%%%%%%%%%%%%%%%%%%%%% 
%%%%%%%%%%%%%%%%%%%%%%%%%%%%%%%%%%
%%%%%%%%%%%%%%%%%%%%%%%%%%%%%%%%%%

\newpage
\section{Functions: Correspondence View (Week 3) and Covariation View (Week 4) (Length: \about 6 hours)}   \label{s: function correspondence}

%%%%%%%%% 3-4 OVERVIEW  %%%%%%%%%%%%%%%%%%%%

\vspace*{-8pt}\subsection{Overview}

\vspace*{-8pt}
This lesson focuses on honing mathematical/teaching practices in the following content.

\vspace*{-4pt}
\begin{tabular}{L{6.5in}} 
{\bf Content} \\ \hline \parskip4pt
\emph{Functions}, defined as a relation from $D$ to $R$ where each input in $D$ is assigned to no more than one output in $R$.

\emph{Invertible function}, defined as a function whose inverse relation is a function; \emph{non-invertible function}, defined as a function whose inverse relation is not a function.

\emph{Partial inverse} of a function $f$, defined implicitly as a function $g$ that is an inverse of $f$ on a restricted domain of $f$, and where the domain of $g$ equals the range of $f$.

Additionally, we revisit \emph{composition} and \emph{inverse} so that we can use them to compare and contrast correspondence and covariation views. By these terms, we mean:
	\begin{itemize*}
	\item % correspondence 
	(Correspondence) Conceiving of functions and their behavior primarily in terms of maps from individual elements of the domain to individual elements of the range. 
	\item % covariation
	(Covariation) Conceiving of functions and their behavior primarily in terms of coordinating how changes in the value of one variable impact the value of the other variable.
	\end{itemize*} 
\end{tabular} 


\vspace*{-12pt}
\begin{tabular}{L{6.5in}} 
{\bf Mathematical/Teaching Practices} \\ \hline \parskip4pt
\emph{Introducing a definition}, involving introductory examples and non-examples, a precise statement of the definition, and interpreting the definition and its terminology using examples, non-examples, and various representations. 

\emph{Explaining a mathematical ``test'' of a property}, meaning to introduce what the test does, how it works, how to pass or fail the test, and why the test works.

\emph{Noticing student thinking}, meaning to observe what the student may be thinking, interpreting what the student may understand or not, as well as what you are not sure whether they understand, and responding.

\emph{Recognizing and explaining correspondence and covariation views}, meaning to attend to which view or combination of views is being used, and constructing explanations from both views. 
\end{tabular}

%%%%%%%%%%%%%%
\header{Summary}

This chapter comes in three parts: reviewing key examples from Homework 2, working with functions from a correspondence point of view, and working with functions from a covariation point of view.

{\bf Review of Homework 2.} We review how graphs can be used to represent relations (and hence functions) using Homework 2 Problem \ref{h: composition correspondence}. We also review the functions used in Homework 2 Problem \ref{h: key graphs} so that the discussion of partial inverses can focus on how to construct partial inverses as opposed to details about the functions themselves, particularly sine and cosine graphs.

{\bf Functions and the correspondence view.} After defining {\it functions}, and reviewing Homework 2 Problem \ref{h: function definition}, we introduce the teaching practices of {\it introducing a definition} and {\it explaining a mathematical ``test'' of a property}. The latter uses the example of the {\it vertical line test}.   We then define {\it invertible function}. We continue practicing how to explain a mathematical test of a property by looking at the {\it horizontal line test}. We then define {\it partial inverses of functions} and discuss {\it how to construct viable partial inverses of functions}. We construct partial inverses for sine and cosine functions.

{\bf Covariational view.} We use the Morgan Minicase to introduce the difference between correspondence and covariational views. By covariational view, we mean understanding how changing the value of one variable impacts the value of the other variable, and learning to coordinate changes in one variable with changes in the other. This minicase introduces the teaching practice of {\it recognizing and explaining correspondence and covariation views}. We revisit composition and inverse to compare and contrast these views and provide an opportunity engage in this teaching practice. 

{\it Acknowledgements.} The Morgan Minicase activities are based on materials in progress for the Content Knowledge for Teaching Minicases project of the Educational Testing Service and is used with permission. The Morgan Minicase is one of a suite of items developed by the Measures of Effective Teaching project and examined through a collaborative grant between the Educational Testing Service and University of Nebraska-Lincoln (NSF \#DGE-1445551/1445630).

% Morgan item; acknowledge ETS

%%%%%%%%%%%%%%
\newpage
\begin{bignote}[Materials]
\begin{itemize*}
\item Handouts from In-Class Resources (can be printed double-sided)
         % \item other things as necessary, such as colored chalk or markers; other handouts; other props
\end{itemize*}
\end{bignote}

%%%%%%%%%%%%%%%%%%%%%%%%%%%
%%%%%%%% 3-4 CONTENT %%%%%%%%%%%%%
%%%%%%%%%%%%%%%%%%%%%%%%%%%%
% 3-4
%%%%%%% 3-4 REVIEW OF KEY EXAMPLES  %%%%%%%
\vspace*{-12pt}
\subsection{Review of key examples}

%{ \bf Review of Homework 2.} We review how graphs can be used to represent relations (and hence functions) using Homework 2 Problem \ref{h: composition correspondence}. These set up the discussion of how definitions for working with relations specialize to the case when the relation is a function. We also review the functions used in Homework 2 Problem \ref{h: key graphs} so that the discussion of partial inverses can focus on how to construct partial inverses as opposed to details about the functions themselves, particularly sine and cosine graphs.

%Review:
% Finding points in composition of graph.
% Sine, cosine, absolute value, square, cube graphs

\vspace*{-4pt}
\subsubsection{Using the definition of graph of a relation}
In the previous homework, you were given the graphs of two functions $f$ and $g$ and asked to find values such as $g(f(1))$ or $g(g(f(2)))$.  Let us now consider how graphs can be used to find these values, and how the reasoning relies on the definition of graph of a relation.

\begin{task}
Compare your responses to Homework 2 Problem \ref{h: composition correspondence}. How did you use the definition of graph in your reasoning? Use $g\circ f(3)$ as an example to illustrate your use of these definitions. Then illustrate your reasoning using $h\circ g$ and where it maps $2$.

\vspace*{-12pt}
\begin{center}
\begin{tikzpicture}[scale=0.3]
\draw (0,-2.5) -- (0, 3.5) node[above]{$y$};
\draw (-6.25,0) -- (8.25,0) node [right]{$x$};
\foreach \x in {-6,-5,-4,...,8}
	\draw (\x,-0.1) -- (\x, 0.1);
\foreach \x in {-2,-1, ..., 3}
	\draw (-0.1,\x) -- (0.1,\x);
\draw[xshift=1cm, ultra thick, red] (-2,-2) -- (2,2);
\draw[ultra thick, red] (3,2) -- (5,2) node[above]{$f$} -- (8,2);
\draw[ultra thick, red] (-1,-2) -- (-6,-2);
\draw[yshift=1cm, ultra thick, dashed, blue] (-2,2) -- (2,-2);
\draw[ultra thick, dashed, blue] (2,-1) -- (5,-1)node[below]{$g$}-- (7.5,-1);
\draw[ultra thick, dashed, blue] (-2,3) -- (-5.5,3); 
\end{tikzpicture}
%
\begin{tikzpicture}[scale=0.3]
\draw (0,-2.5) -- (0, 3.5) node[above]{$y$};
\draw (-6.25,0) -- (8.25,0) node [right]{$x$};
\foreach \x in {-6,-5,-4,...,8}
	\draw (\x,-0.1) -- (\x, 0.1);
\foreach \x in {-2,-1, ..., 3}
	\draw (-0.1,\x) -- (0.1,\x);
\draw[ultra thick, green] (-6,-2) -- (6,2) node[above]{$h$}  (-6,2) -- (6,-2) ;
\end{tikzpicture}
\end{center}

\end{task} 

\begin{solution}({\it Partial.})
By definition of a graph of an equation, $(a, b)$ is on the graph of $y=f(x)$ if and only if $b=f(a)$. There is only one coordinate on the graph of $f$ with the $x$-value 3, so that coordinate's $y$-value must be $f(3)$. We then find where $g$ sends the input $f(3)$ using similar logic.

For finding where $h\circ g$ maps $2$, we first seek where $g$ maps $2$. That coordinate's $y$-value must be $g(2)$. There are multiple possible values for where $h\circ g$ maps $2$, because the graph of $h$ contains multiple coordinate points whose $x$-value is $g(2)$.
\end{solution}

\vspace*{-4pt}
\subsubsection{Some functions and relations we will examine further today}

In the homework, you worked with the graphs of $y=\sin(x)$, $y=\cos(x)$, $y=|x|$, $y=x^2$, and $y=x^3$ and their inverse relations. We also learned last time that a point $(a,b)$ is on the graph of a relation $r$ if and only if $(b,a)$ is on the graph of its inverse relation $r^{-1}$. 

You may have obtained graphs that looked like this:

\begin{tabular}{C{1.1in}C{1.1in}C{1.1in}C{1.1in}C{1.1in}}
$\sin$ and its inverse relation & $\cos$ and its inverse relation & $\tn{Abs}$ and its inverse relation
 & $\tn{Sq}$ and its inverse relation &   $\tn{Cu}$ and its inverse relation \\ 
\begin{tikzpicture} [scale=0.14]
\draw (-10, 0) -- (10,0) (0,-10)--(0,10);
\draw[domain=-9.7:9.7,smooth,variable=\x]    plot (\x,{sin(\x r)});
\draw[domain=-9.7:9.7,smooth,variable=\y] plot ({sin(\y r)},\y);
\end{tikzpicture}
&
\begin{tikzpicture} [scale=0.14]
\draw (-10, 0) -- (10,0) (0,-10)--(0,10);
\draw[domain=-9.7:9.7,smooth,variable=\x]    plot (\x,{cos(\x r)});
\draw[domain=-9.7:9.7,smooth,variable=\y] plot ({cos(\y r)},\y);
\end{tikzpicture}
& 
\begin{tikzpicture} [scale=0.14]
\draw (-10, 0) -- (10,0) (0,-10)--(0,10);
\draw[domain=-9.5:9.5,smooth,variable=\x]    plot (\x, {abs(\x)});
\draw[domain=-9.5:9.5,smooth,variable=\y] plot ({abs(\y)},\y);
\end{tikzpicture}
&
\begin{tikzpicture} [scale=0.14]
\draw (-10, 0) -- (10,0) (0,-10)--(0,10);
\draw[domain=-3.2:3.2,smooth,variable=\x]  plot ({\x},{\x*\x});
\draw[domain=-3.2:3.2,smooth,variable=\y] plot ({\y*\y},{\y});
\end{tikzpicture}
&
\begin{tikzpicture} [scale=0.14]
\draw (-10, 0) -- (10,0) (0,-10)--(0,10);
\draw[domain=-2.2:2.2,smooth,variable=\x]  plot ({\x},{\x*\x*\x});
\draw[domain=-2.2:2.2,smooth,variable=\y] plot ({\y*\y*\y},{\y});
\end{tikzpicture} 
\end{tabular}

\smallnote{Together with the class, label some maxima and minima of these graphs, as well as a few $x$-intercepts and $y$-intercepts. Emphasize again that $(x,y)$ is on the graph of a relation if and only if $(y,x)$ is on the graph of its inverse.}

\newpage

%%%%%%% 3-4 FUNCTIONS AND THE CORRESPONDENCE VIEW  %%%%%%%
\subsection{Functions and the correspondence view}
% {\bf Functions and the correspondence view.} After defining {\it functions}, we discuss how some definitions encountered in the previous chapter, such as composition and graph, specialize to the case of function. We then introduce the teaching practice of {\it explaining a mathematical ``test'' of a property} using the example of the {\it vertical line test}.   We then define {\it partial inverses of functions} and discuss {\it how to construct viable partial inverses of functions}, and then define {\it invertible function}. We continue practicing how to explain a mathematical test of a property by looking at the {\it horizontal line test}.  We conclude this part of the lesson by revealing that thus far we have been viewing relations and functions from a {\it correspondence view} of the equation $y=f(x)$, meaning only thinking about functions in terms of maps from elements of the domain to elements of the range. 

% Part 1 of this lesson:
% Define functions

As you saw in the previous homework Problem \ref{h: function definition} (p.~\pageref{h: function definition}), a function is a special kind of relation, where every element of the domain is assigned to exactly one element of the range. Here are two ways to think about functions: 

\begin{definition}[Function: Middle and High school version]\label{d: function middle high school}
A \emph{function} $f$ from $D$ to $R$ is a relation from $D$ to $R$ where each input in $D$ is assigned to no more than one output in $R$.
\end{definition} 
 
\begin{definition}[Function: University version]\label{d: function university} 
A \emph{function} is a relation $f: D \to R$, such that if $(x,y), (x, y')\in f$, then $y=y'$.
\end{definition}

\vspace*{-6pt}
With relations in general, we can't say exactly where an input is mapped to, because it may be mapped to multiple outputs such as in the parent function. However, with functions, an input determines the output uniquely in the sense that it is either undefined or it is exactly one value. It is because of this property that we have function notation. The notation $f(x)$ means ``the at most one value that $f$ maps $x$ to''.

\smallnote{
 The following is provided more as a reference than as something to address explicitly in class. }

Here is a table summarizing how we can use this notation to interpret concepts associated to relations:

\begin{tabular}{L{1in}L{2.5in}|L{2.5in}}
	& Relations in general & When the relation is a function \\ \hline
Domain &  All elements $a\in D$ with nonempty image 
	& All elements $a\in D$ such that $f(a)$ is defined \\
Range & All element $b \in R$ with nonempty preimage
	 & All elements $b\in R$ such that $b=f(x)$ has a solution in $x$ \\
Composition $g\circ f$ & The relation contains all assignments $x\to z$ such that there is a $y$ where $f$ maps $x$ to $y$ and $g$ maps $y$ to $z$  
	& This relation maps $x$ to $f(g(x))$. It is also a function. \\ 
Graph of $f$ & The point $(a,b)$ is on the graph of $f$ if and only if $f$ maps $a$ to $b$
	& The point $(a,b)$ is on the graph of $f$ if and only if $b=f(a)$. 
	{\it Or}: The point $(a,b)$ is on the graph of $f$ if and only if $a$ is a solution to $b=f(x)$. \\
\end{tabular}

The above table uses the following definitions:

\begin{definition}
Given a function $f:D\to R$, we say that $f(x)$ is \emph{defined} if $x$ has nonempty image. Otherwise, if $x$ has empty image, we say $f(x)$ is \emph{undefined}.

Given $b\in R$, We say that the equation $b=f(x)$ \emph{has a solution in $x$} if there is an $a\in D$ such that $b=f(a)$ is a true statement. Otherwise, if there is no $a\in D$ such that $b=f(a)$ is true, then we say the equation \emph{has no solution}.
\end{definition}


\vspace*{-12pt}
\subsubsection{Teaching graphs of functions as a case of teaching definitions}
\label{s: teaching definitions}

Functions and relations are fundamental concepts in middle and high school. When you are teaching the function and relation, you will mostly likely find it useful to discuss their definitions in the context of different representations, such as cloud diagrams, T-charts (table of input and output values) and graphs.  As we have been experiencing, teaching definitions often contains these components:

\begin{mdframed}
\begin{center}
{\bf Teaching definitions}
\end{center}
\vspace*{-4pt}
	\begin{itemize*}
	\item Introductory examples and/or non-examples of the definition
	\item Precise statement of the definition
	\item Interpreting the precise statement, especially any new terminology or key rules, in terms of the introductory example and/or non-example
	\item Interpreting the terminology and rules in terms of the introductory examples, often using different \\ representations that students will continue to encounter.
	\end{itemize*}
\end{mdframed}

If you were teaching the definition of function, you might use examples such as the ones we have used today, or other ones that build on or review what your students are familiar with.

\begin{task}
Review the definition of the graph of a relation. We can use this definition for functions, too, because all functions are relations.

Graph the relations $f(x)=\frac{1}{x}$, $g(x)=\sqrt{x}$ and $\tn{Abs}^{-1}$ on separate graphs. For each, how could you use the graph to understand the definition of function?

Suppose your students are seeing graphs of functions for the very first time. How would you explain the definition of a function using a graph, in general terms? (meaning, not relying on any particular example; an explanation whose wording would apply to all possible cases)
\end{task}

Remember that your students haven't seen anything called a ``vertical line test'', so they don't know this. However, the above might be a good way of getting to the idea of the ``vertical line test.''

\begin{bignote}
When asking pre-service teachers to explain their thinking, many often fall back on phrases like ``vertical line test''. If this happens, then press on what such phrases mean in terms of the definition of function. It may be the case that they are using this phrase to mean a procedure that they do not understand the meaning of; this is useful background information for you as an instructor, because the ``vertical line test'' and how to teach this procedure with meaning is addressed next. However, whatever the discussion, there must be some sort of complete explanation that uses at least one definition of function explicitly.
\end{bignote}

\subsubsection{Teaching the vertical line test as a case of explaining a mathematical ``test'' of a property}
\label{s: mathematical test of a property}

A common part of high school lessons on graphs is discussing the ``vertical line test''. This is a ``test'' in the sense that it will ``test'' a mathematical thing (a graph of a relation) for a particular property (whether the relation is a function). There are many ``tests'' like this in high school mathematics; for instance, you may remember ``convergence tests'' from calculus. In general, here are some elements that may be useful to keep in mind when explaining a mathematical ``test'' of a property, using the vertical line test as a way to illustrate a way to teach mathematical ``tests'' of properties. 


\begin{mdframed}
\begin{center}
{\bf Explaining a Mathematical ``Test'' of a Property}
\end{center}
\begin{tabular}{|L{2.2in}|L{3.7in}|}
\hline
	& {\it Example}: \\ 
Introduce {\it what}:	
	\begin{itemize*}
	\item Name the test.
	\item What is the test supposed to tell us? (Be precise!) 
	\item What are you testing? (Be precise!) 
	\end{itemize*} \vspace*{-12pt}$\;$
	& 
	The Vertical Line Test
	\begin{itemize*}
	\item	Today we will learn something called the \emph{Vertical Line Test}. 	
	\item This test is a way of telling \emph{whether a relation is a function}.
	\item We will test the \emph{graph of a relation} to tell this.
	\end{itemize*}
	\\ \hline 
Describe {\it how}:
	\begin{itemize*}
	\item How do you do the test?
	\item How do you tell whether the thing passes or fails the test?
	\end{itemize*}
	& 
	Here is how we do the Vertical Line Test: 
	\begin{itemize*}
	\item Graph the relation.
	\item Think about all vertical lines in the plane. 
	
		 Look at whether the vertical lines intersect the graph of the relation.
	\end{itemize*}
	
	Here is how to pass or fail:
	\begin{itemize*}
	\item If all of the vertical lines cross the graph zero or one times and no more, then the graph passes. 
		Otherwise the graph fails.
	\end{itemize*} \vspace*{-12pt}$\;$
	\\ \hline 
Deliver the {\it punchline}: What happens when the thing ``passes'' the test? What happens when the thing ``fails'' the test?
	& 
	If the graph of a relation passes, then the relation is a function. If the graph of a relation fails, then the relation is not a function. 
	 \\ \hline 
Explain {\it why} the test ``works'': 
	& 
	[This explanation is for your homework] \\ \hline 
\end{tabular}
\end{mdframed}


To warm up to understand this table, we first discussed the following:

\vspace*{-4pt}
\begin{task}
What does the ``vertical line test'' do? How do you perform this ``test''? How does a graph of a relation to pass or fail the ``vertical line test''? What does it mean about the relation if its graph passes or fails? Why is this conclusion true?
\end{task}

As we discussed these issues, we talked about the elements of the table.


\smallnote{The class is unlikely to have time to fill in the next table completely. You might direct them to focus primarily on the first three rows, and then go over them. For the last row, you might instruct them to write down the claim, in the form of an ``if-then'' statement, that they are to explain.}

\begin{task}
Based on the discussion we have had, how would you do the following?

\begin{center}
{\bf Explaining a Mathematical ``Test'' of a Property: The Vertical Line Test}

\begin{tabular}{|L{4in}|L{1.5in}|}
\hline 
Introduce {\it what}:	
	\begin{itemize*}
	\item Name the test.
	\item What is the test supposed to tell us? (Be precise!) 
	\item What are you testing? (Be precise!) 
	\end{itemize*} \vspace*{-12pt}$\;$
	& 

	\\ \hline 
Describe {\it how}:
	\begin{itemize*}
	\item How do you do the test?
	\item How do you tell whether the thing passes or fails the test?
	\end{itemize*} \vspace*{-12pt}$\;$
	& 

	\\ \hline 
Deliver the {\it punchline}: What happens when the thing ``passes'' the test? What happens when the thing ``fails'' the test?
	& 

	 \\ \hline 
Explain {\it why} the test ``works'': 
	& \\ & 
	\\ \hline 
\end{tabular}
\end{center}
\end{task}



\begin{task}
What do you think of these elements? What might the purpose of each element be as far as teaching and high school students go?
\end{task}


\subsubsection{Invertible functions and the horizontal line test}
% Define invertible functions
% Interpret definition in terms of graph; horizontal line test.
% (Continue practicing explaining a mathematical ``test'' of a property.)


As we learned in the previous chapter, all functions have inverse {\it relations}.

This is because all relations have inverse relations, and functions are relations.

Only some functions have inverse {\it functions}, meaning inverse relations that happen to be functions.

\begin{definition}\label{d: invertible function}
 A function $f:D\to R$ is \emph{invertible function} if its inverse relation $f^{-1}:R\to D$ is a function.

A function $f:D\to R$ is \emph{non-invertible function} if its inverse relation $f^{-1}:R\to D$ is a not a function.
\end{definition}

\begin{task}
Examples of functions we have seen include sine, cosine, Abs, Sq, Cu, $x\mapsto \frac{1}{x}$, and $x\mapsto \sqrt{x}$. 

How would you use these examples as a way to teach the definitions of {\it invertible function} and {\it non-invertible function}? As you think about this, focus on this element of explaining definitions: {\bf interpreting the terminology and rules in terms of the introductory examples}, using representations that students will continue to encounter.
\end{task}

When you are teaching invertible functions to students who are seeing it for the first time, they may not have heard of anything called a ``horizontal line test.''  It is good practice to try to base explanations of invertibility as much as possible on the definitions of function and invertible/non-invertible functions as a way to help students understand the ideas conceptually. This provides a more solid foundation for ultimately understanding the horizontal line test.

\begin{bignote}
Just as in the discussion about functions in the previous section, it is important to press teachers on how their reasoning explicitly draws on the definitions of function and invertible/non-invertible function. 
\end{bignote}

\begin{task}
What is the ``horizontal line test''? Explain this test of a mathematical property.  

\begin{center}
{\bf Explaining  \underline{\hspace*{2in}}}
\end{center}
\begin{tabular}{|L{2.2in}|L{3.7in}|}
\hline 
Introduce {\it what}:	
	\begin{itemize*}
	\item Name the test.
	\item What is the test supposed to tell us? (Be precise!) 
	\item What are you testing? (Be precise!) 
	\end{itemize*} \vspace*{-12pt}$\;$
	& 

	\\ \hline 
Describe {\it how}:
	\begin{itemize*}
	\item How do you do the test?
	\item How do you tell whether the thing passes or fails the test?
	\end{itemize*}
	& 

	\\ \hline 
Deliver the {\it punchline}: What happens when the thing ``passes'' the test? What happens when the thing ``fails'' the test?
	& 

	 \\ \hline 
Explain {\it why} the test ``works'': 
	& 
	\\ \hline 
\end{tabular}
\end{task}

We have discussed previously that inverse often means ``undoing''. In the case of invertible functions, the ``undoing'' metaphor works well.
\begin{theorem}[Non-invertible may not undo, invertible always undos]
Let $f:D\to R$ be a function and $f^{-1}$ be its inverse relation.
	\begin{itemize}
	\item When $f$ is non-invertible, $f^{-1}\circ f$ may map an element $x$ to an element other than $x$.
	\item When $f$ is invertible, we have $f^{-1}\circ f(x)=x$ for all $x$ in the domain of $f$.
	\end{itemize}
\end{theorem}
We can see this in our examples of non-invertible functions, such as $\tn{Sq}$, $\sin$, or $\cos$. The proof below summarizes the general phenomenon.

\begin{proof}
Given $f:D\to R$ is a function and $f^{-1}$ is its inverse relation. Let $a$ be an element of the domain of $f$, and suppose $f(a)=b$. When $f$ is non-invertible, $f^{-1}$ may map $a$ to multiple elements, so $f^{-1}\circ f$ may then map $a$ to multiple elements. When $f$ is invertible, $f^{-1}(b)=a$, so $f^{-1}\circ f(a)=a$. 
\end{proof}

\smallnote{The above theorem is phrased in a non-standard way, discussing both the cases of non-invertible as well as invertible, as opposed to only invertible functions. The reason for this is that we have found that undergraduate mathematics students in general can ignore the conditions of a theorem to hold (e.g., whether a function is invertible), especially when the conclusion is a very nice and seemingly intuitive one (e.g., $f^{-1}\circ f(x)=x$ for all $x$ in the domain of $f$). Our solution to this is to phrase theorems to account for the nice and not-so-nice conditions. While not a panacea, we have found it generally helpful in that students are more likely to remember how different conditions may lead to different conclusions.}

This brings us to the definition of the inverse of an invertible function seen most often in high school.

\begin{definition}[Inverse of a function: High school version] \label{d: inverse of a function}
If $f:D\to R$ is not an invertible function, then it does not have an inverse function.

If $f:D\to R$ is an invertible function, then the {\bf inverse function} of $f$ is the function $f^{-1}$ such that for all $x$ in the domain of $f$, we have
	$$f^{-1}\circ f(x)=x.$$
\end{definition}

\begin{task}
Suppose $f$ and $g$ are invertible functions. Is $g\circ f$ invertible? If so, what is its inverse?
\end{task}

\subsubsection{Constructing partial inverses of functions (aka ``fake inverses'')}
% Define partial inverses of functions
% (Construction: viable partial inverses of functions.)


There are two procedures that are often discussed in high school related to inverse of a function:
	\begin{itemize}
	\item To find the graph of the inverse of a function, reflect the graph of the function over the line $y=x$
	\item To find the formula for the inverse of a function, switch the $y$'s and $x$'s then solve for $y$.
	\end{itemize}
We previously discussed why the first procedure makes sense. It is because reflecting over the line $y=x$ always sends the coordinate $(a,b)$ to the coordinate $(b,a)$. The coordinates represent assignments, and inverses switch input with output. We can use similar reasoning to explain why the second procedure makes sense. 

\begin{task}
Explain why the following procedure works: {\it To find the formula for the inverse of a function, switch the $y$'s and $x$'s then solve for $y$.} Your explanation should consist of two parts: specific and general. For the specific version, you might use an example such as $f(x)=x^3+1$. The general version should not rely on any example and instead apply to all functions.
\end{task}

\begin{solution} [Homework.] \end{solution}

What if a function is non-invertible but we would like to be able to create some sort of inverse anyway? 

\begin{task}
The function $\tn{Sq}:\R\to\R, x\mapsto x^2$ does not have an inverse function. However, suppose you wanted to construct an function that is like an inverse, given that we know we can't have a true inverse.
Rank these from best to worst as candidates for an ``inverse'' for $\tn{Sq}$: (Remember that $\sqrt{x}$ is defined as the zero or positive root of $x$, when a real root exists.)

\begin{itemize}[label=$\circ$]
\item $\tn{Rt}_1(x)=-\sqrt{x}$ 
\item $\tn{Rt}_2(x)=\sqrt{x}$
\item $\tn{Rt}_3(x)$ is defined to be $-\sqrt{x}$ when $x\in[0,1)$ and $\sqrt{x}$ when $x\in [1,\infty)$.
\item $\tn{Rt}_4(x)$ is defined to be $-\sqrt{-x}$ when $x\leq 0$ and $\sqrt{x}$ when $x\geq 0$.
\end{itemize}
\end{task}

Each of the functions $\tn{Rt}_1, \tn{Rt}_2, \tn{Rt}_3, \tn{Rt}_4$ are called ``partial inverses''. They are like inverse functions, except that they don't work everywhere; they only work some of the time. So they are ``fake'' in the sense that they don't work on the entire domain, but they are still an ``inverse'' in the sense that they will be an inverse in a restricted subset of the domain.

\smallnote{In our experience, pre-service teachers find the nickname ``fake inverse'' memorable and helpful for understanding partial inverses. We coined this nickname prior to 2017. In 2017, we tried  ``good enough inverse'' to avoid reference to current event usage of ``fake'', but this did not have the same effect.  Why do we nickname things that already have perfectly good names? As a memory aid. The concept of ``partial inverse'' is important to high school mathematics, especially trigonometry. However, if teachers never remember the idea of ``partial inverse'', then the fact that arcsine or arccosine are not true inverses will not be passed along. We have found nicknames on the whole more beneficial than detrimental.}

\smallnote{We recommend doing this next activity quickly, as a class, with numerical examples.}
\begin{task}
What is the domain of $\tn{Sq}$? 

On what subset of the domain of $\tn{Sq}$ do each of  $\tn{Rt}_1, \tn{Rt}_2, \tn{Rt}_3, \tn{Rt}_4$ serve as a true inverse?  (They satisfy the equation in Definition \ref{d: inverse of a function}.)

On what subset of the domain of $\tn{Sq}$ are the functions $\tn{Rt}_1, \tn{Rt}_2, \tn{Rt}_3, \tn{Rt}_4$ serve as a fake inverse? (They do not satisfy the equation in Definition \ref{d: inverse of a function}.). 
\end{task}

Finally, let's apply the ideas of partial inverse to a trigonometric function:
 
\begin{task}
Construct at least three candidates for partial inverses (``fake inverses'') for $f(x)=\sin(x)$. 

Rank the candidates you have constructed from best to worst ``inverse'' for the sine function.

On what subset of the domain of sine do each of your candidates serve as a true inverse? On what subset of the domain of sine do your candidates serve as a fake inverse? 

\vspace*{2pt} \hrule \vspace*{2pt}
Construct at least three candidates for partial inverses (``fake inverses'') for $f(x)=\cos(x)$. 

Rank the candidates you have constructed from best to worst ``inverse'' for the cosine function.

On what subset of the domain of cosine do each of your candidates serve as a true inverse? On what subset of the domain of cosine do your candidates serve as a fake inverse? 
\end{task}

\smallnote{The following describes what usually happens with rankings of candidates for inverse of sine.}

You may have concluded that the ``best'' inverse is the candidate whose domain of being a true inverse is $[\frac{\pi}{2}, \frac{\pi}{2}]$. This is in fact how the function that you may have seen before, with the (unfortunate) notation $\sin^{-1}$, is defined. (An alternative notation is $\arcsin$.) In general, when working with functions for which we would like to have an inverse but the functions are not actually invertible, we can define partial inverses (``fake inverses'') for them which we take to be standard. Examples of this include $x\mapsto \sqrt{x}$ for Sq, arcsin for sine, and arccosine for cosine. They usually are the candidate where if you looked at the alternatives, you would agree that there's something about that candidate that seems to work better: they are continuous, the subset of the domain on which they are a ``true'' inverse is close to zero or symmetric around zero, etc.


%%%%%%%%%%%%%%%%%%%%%%%%%%%%%%%%%%%%%%%%
%%%%%%%%% 3 IN-CLASS RESOURCES %%%%%%%%%%%%%%%%%
%%%%%%%%%%%%%%%%%%%%%%%%%%%%%%%%%%%%%%%%
\newpage \subsection{In-Class Resources}  
\handout{Opener: Review of key examples}

Compare your responses to Homework 2 Problem \ref{h: composition correspondence}. How did you use the definition of graph in your reasoning? Use $g\circ f(3)$ as an example to illustrate your use of these definitions. Then illustrate your reasoning using $h\circ g$ and where it sends $2$.

\vspace*{-12pt}
\begin{center}
\begin{tikzpicture}[scale=0.3]
\draw (0,-2.5) -- (0, 3.5) node[above]{$y$};
\draw (-6.25,0) -- (8.25,0) node [right]{$x$};
\foreach \x in {-6,-5,-4,...,8}
	\draw (\x,-0.1) -- (\x, 0.1);
\foreach \x in {-2,-1, ..., 3}
	\draw (-0.1,\x) -- (0.1,\x);
\draw[xshift=1cm, ultra thick, red] (-2,-2) -- (2,2);
\draw[ultra thick, red] (3,2) -- (5,2) node[above]{$f$} -- (8,2);
\draw[ultra thick, red] (-1,-2) -- (-6,-2);
\draw[yshift=1cm, ultra thick, dashed, blue] (-2,2) -- (2,-2);
\draw[ultra thick, dashed, blue] (2,-1) -- (5,-1)node[below]{$g$}-- (7.5,-1);
\draw[ultra thick, dashed, blue] (-2,3) -- (-5.5,3); 
\end{tikzpicture}
%
\begin{tikzpicture}[scale=0.3]
\draw (0,-2.5) -- (0, 3.5) node[above]{$y$};
\draw (-6.25,0) -- (8.25,0) node [right]{$x$};
\foreach \x in {-6,-5,-4,...,8}
	\draw (\x,-0.1) -- (\x, 0.1);
\foreach \x in {-2,-1, ..., 3}
	\draw (-0.1,\x) -- (0.1,\x);
\draw[ultra thick, green] (-6,-2) -- (6,2) node[above]{$h$}  (-6,2) -- (6,-2) ;
\end{tikzpicture}
\end{center}


\vfill 


In the homework, you worked with the graphs of $y=\sin(x)$, $y=\cos(x)$, $y=|x|$, $y=x^2$, and $y=x^3$ and their inverse relations. We also learned last time that a point $(a,b)$ is on the graph of a relation $r$ if and only if $(b,a)$ is on the graph of its inverse relation $r^{-1}$. 

You may have obtained graphs that looked like this:

\begin{tabular}{C{1.1in}C{1.1in}C{1.1in}C{1.1in}C{1.1in}}
$\sin$ and its inverse relation & $\cos$ and its inverse relation & $\tn{Abs}$ and its inverse relation
 & $\tn{Sq}$ and its inverse relation &   $\tn{Cu}$ and its inverse relation \\ 
\begin{tikzpicture} [scale=0.14]
\draw (-10, 0) -- (10,0) (0,-10)--(0,10);
\draw[domain=-9.7:9.7,smooth,variable=\x]    plot (\x,{sin(\x r)});
\draw[domain=-9.7:9.7,smooth,variable=\y] plot ({sin(\y r)},\y);
\end{tikzpicture}
&
\begin{tikzpicture} [scale=0.14]
\draw (-10, 0) -- (10,0) (0,-10)--(0,10);
\draw[domain=-9.7:9.7,smooth,variable=\x]    plot (\x,{cos(\x r)});
\draw[domain=-9.7:9.7,smooth,variable=\y] plot ({cos(\y r)},\y);
\end{tikzpicture}
& 
\begin{tikzpicture} [scale=0.14]
\draw (-10, 0) -- (10,0) (0,-10)--(0,10);
\draw[domain=-9.5:9.5,smooth,variable=\x]    plot (\x, {abs(\x)});
\draw[domain=-9.5:9.5,smooth,variable=\y] plot ({abs(\y)},\y);
\end{tikzpicture}
&
\begin{tikzpicture} [scale=0.14]
\draw (-10, 0) -- (10,0) (0,-10)--(0,10);
\draw[domain=-3.2:3.2,smooth,variable=\x]  plot ({\x},{\x*\x});
\draw[domain=-3.2:3.2,smooth,variable=\y] plot ({\y*\y},{\y});
\end{tikzpicture}
&
\begin{tikzpicture} [scale=0.14]
\draw (-10, 0) -- (10,0) (0,-10)--(0,10);
\draw[domain=-2.2:2.2,smooth,variable=\x]  plot ({\x},{\x*\x*\x});
\draw[domain=-2.2:2.2,smooth,variable=\y] plot ({\y*\y*\y},{\y});
\end{tikzpicture} 
\end{tabular}

What comments or questions do you have about these graphs?

\newpage

Compare your response to problem \ref{h: function definition} from the last homework with someone next to you. What comments or questions do you have about the definition of function?

{\bf Definition.} A {\it function} from $D$ to $R$ is defined as a relation from $D$ to $R$ where each input in $D$ is assigned to no more than one output in $R$.

\begin{center}
\begin{tabular}{C{1in}|C{0.5in}|C{0.5in}|C{0.5in}|C{0.5in}|C{0.5in}|C{0.5in}|}
	& $r$ & $r^{-1}$ & $s$ & $s^{-1}$ & $t$ & $t^{-1}$ \\ \hline
is a function & &&&&& \\ \hline
is NOT a function &&&&&& \\ \hline 
\end{tabular}
\end{center}


\bigskip

\begin{tabular}{C{2in}C{2in}C{2in}}
$r$ & $s$ & $t$ \\ 
%r
\begin{tikzpicture}
% [cloud diagram r - is not a function, inverse is not a function]
	\def\pt{circle (0.05)}
	\foreach \x in {0, 1, 2}
	{\filldraw(0,-0.5*\x)  \pt;
	\draw[->] (0,-0.5*\x) -- (2.94, -0.5-0.03*\x+0.03);
	\draw[->] (0,-0.5*\x) -- (2.94, -1.5-0.03*\x+0.03);
	}
	\foreach \y in {0, 1, 2, 3}
	{\filldraw(3,-0.5*\y) \pt;}
	\draw[xshift=-0.5cm] (0,0) arc (180:0:0.5) (0,-1.5) arc (180:360:0.5) (0,0)-- (0,-1.5) (1,0) -- (1,-1.5); 
	\draw[xshift=2.5cm] (0,0) arc (180:0:0.5) (0,-1.5) arc (180:360:0.5) (0,0)-- (0,-1.5) (1,0) -- (1,-1.5); 
\end{tikzpicture}
&
\begin{tikzpicture}
% [cloud diagram s - is a function and its inverse is a function]
	\def\pt{circle (0.05)}
	\foreach \x in {0, 1, 2}
	{\filldraw(0,-0.5*\x)  \pt;}
	\foreach \y in {0, 1, 2, 3}
	{\filldraw(3,-0.5*\y) \pt;}
	\draw[xshift=-0.5cm] (0,0) arc (180:0:0.5) (0,-1.5) arc (180:360:0.5) (0,0)-- (0,-1.5) (1,0) -- (1,-1.5); 
	\draw[xshift=2.5cm] (0,0) arc (180:0:0.5) (0,-1.5) arc (180:360:0.5) (0,0)-- (0,-1.5) (1,0) -- (1,-1.5); 
	%\draw[->] (0,-0.5*\x) -- (2.94, -0.5*\y)
	\draw[->] (0,-0.5*0) -- (2.94, -0.5*3);
	\draw[->] (0,-0.5*2) -- (2.94, -0.5*1);
\end{tikzpicture}
&
\begin{tikzpicture}
% [cloud diagram t - inverse is a function but it is not itself a function]
	\def\pt{circle (0.05)}
	\foreach \x in {0, 1, 2}
	{\filldraw(0,-0.5*\x)  \pt;}
	\foreach \y in {0, 1, 2, 3}
	{\filldraw(3,-0.5*\y) \pt;}
	\draw[xshift=-0.5cm] (0,0) arc (180:0:0.5) (0,-1.5) arc (180:360:0.5) (0,0)-- (0,-1.5) (1,0) -- (1,-1.5); 
	\draw[xshift=2.5cm] (0,0) arc (180:0:0.5) (0,-1.5) arc (180:360:0.5) (0,0)-- (0,-1.5) (1,0) -- (1,-1.5);
	%\draw[->] (0,-0.5*\x) -- (2.94, -0.5*\y)
	\draw[->] (0,-0.5*0) -- (2.94, -0.5*3);
	\draw[->] (0,-0.5*2) -- (2.94, -0.5*1); 
	\draw[->] (0,-0.5*1) -- (2.94, -0.5*2); 
	\draw[->] (0,-0.5*2) -- (2.94, -0.5*0); 
\end{tikzpicture}
\end{tabular}



\newpage 
%%%%%%%%% 3-4 TEACHING DEFINITIONS: GRAPH OF FUNCTIONS %%%% 

\handout{Teaching definitions: Graphs of functions}

\begin{mdframed}
\begin{center}
{\bf Teaching definitions}
\end{center}
\vspace*{-4pt}
	\begin{itemize*}
	\item Introductory examples and/or non-examples of the definition
	\item Precise statement of the definition
	\item Interpreting the precise statement, especially any new terminology or key rules, in terms of the introductory example and/or non-example
	\item Interpreting the terminology and rules in terms of the introductory examples, often using different \\ representations that students will continue to encounter.
	\end{itemize*}
\end{mdframed}


Review the definition of the graph of a relation. (We can use this definition for functions, too, because all functions are relations.)

Graph the relations $f(x)=\frac{1}{x}$, $g(x)=\sqrt{x}$ and $\tn{Abs}^{-1}$ on separate graphs. For each, how could you use the graph to understand the definition of function?

Suppose your students are seeing graphs of functions for the very first time. How would you explain the definition of a function using a graph, in general terms? (meaning, not relying on any particular example; an explanation whose wording would apply to all possible cases)


\newpage
%%%%%%%%%% 3-4 VLT %%%%%%%%%
\handout{Explaining a mathematical ``test'' of a property: The Vertical Line Test}


What does the ``vertical line test'' do? How do you perform this ``test''? How does a graph of a relation to pass or fail the ``vertical line test''? What does it mean about the relation if its graph passes or fails? Why is this conclusion true?

\vfill 

Based on discussion of the above questions, how would you do the following?

\begin{center}
{\bf Explaining a Mathematical ``Test'' of a Property: The Vertical Line Test}

\begin{tabular}{|L{2.5in}|L{4in}|}
\hline 
Introduce {\it what}:	
	\begin{itemize*}
	\item Name the test.
	\item What is the test supposed to tell us? (Be precise!) 
	\item What are you testing? (Be precise!) 
	\end{itemize*} \vspace*{-12pt}$\;$
	& 

	\\ \hline 
Describe {\it how}:
	\begin{itemize*}
	\item How do you do the test?
	\item How do you tell whether the thing passes or fails the test?
	\end{itemize*} \vspace*{-12pt}$\;$
	& 

	\\ \hline 
Deliver the {\it punchline}: What happens when the thing ``passes'' the test? What happens when the thing ``fails'' the test?
	& 

	 \\ \hline 
Explain {\it why} the test ``works'': 
	& \vspace*{2in}\\ 
	\hline 
\end{tabular}
\end{center}

%%%%%%%%% 3-4 INVERTIBLE 
\newpage \handout{Teaching definitions: Invertible function}

\begin{mdframed}\raggedright
{\bf Definition \ref{d: invertible function}}.
 A function $f:D\to R$ is \emph{invertible function} if its inverse relation $f^{-1}:R\to D$ is a function.

A function $f:D\to R$ is \emph{non-invertible function} if its inverse relation $f^{-1}:R\to D$ is a not a function.
\end{mdframed}

Examples of functions we have seen include sine, cosine, Abs, Sq, Cu, $x\mapsto \frac{1}{x}$, and $x\mapsto \sqrt{x}$. 

How would you use these examples as a way to teach the definitions of {\it invertible function} and {\it non-invertible function}? As you think about this, focus on interpreting the terminology and rules in terms of the introductory examples.

\newpage 
%%%%%%%% 3-4 HLT 
\handout{Explaining a mathematical ``test'' of a property: The Horizontal Line Test}

What is the ``horizontal line test''? Explain this test of a mathematical property.  

\begin{center}
{\bf Explaining  \underline{\hspace*{2in}}}
\end{center}
\begin{tabular}{|L{2.2in}|L{3.7in}|}
\hline 
Introduce {\it what}:	
	\begin{itemize*}
	\item Name the test.
	\item What is the test supposed to tell us? (Be precise!) 
	\item What are you testing? (Be precise!) 
	\end{itemize*} \vspace*{-12pt}$\;$
	& 

	\\ \hline 
Describe {\it how}:
	\begin{itemize*}
	\item How do you do the test?
	\item How do you tell whether the thing passes or fails the test?
	\end{itemize*}
	& 

	\\ \hline 
Deliver the {\it punchline}: What happens when the thing ``passes'' the test? What happens when the thing ``fails'' the test?
	& 

	 \\ \hline 
Explain {\it why} the test ``works'': 
	& \vspace*{2in}
	\\ \hline 
\end{tabular}

%%%%%%%%%%
\newpage
\handout{Inverses and partial inverses}
Suppose $f$ and $g$ are invertible functions. Is $g\circ f$ invertible? If so, what is its inverse?


\vfill
Explain why the following procedure works:

\begin{quote}
To find the formula for the inverse of a function, switch the $y$'s and $x$'s then solve for $y$.
\end{quote}

Your explanation should consist of two parts: specific and general. For the specific version, you might use an example such as $f(x)=x^3+1$. The general version should not rely on any example and instead apply to all functions.

\vfill 

\vspace*{2pt}\hrule\vspace*{2pt}
What if a function is non-invertible but we would like to be able to create some sort of inverse anyway? 
The function $\tn{Sq}:\R\to\R, x\mapsto x^2$ does not have an inverse function. However, suppose you wanted to construct an function that is like an inverse, given that we know we can't have a true inverse.

Rank these from best to worst as candidates for an ``inverse'' for $\tn{Sq}$:

(Remember that $\sqrt{x}$ is defined as the zero or positive root of $x$, when a real root exists.)

\begin{itemize}[label=$\circ$]
\item $\tn{Rt}_1(x)=-\sqrt{x}$ 
\item $\tn{Rt}_2(x)=\sqrt{x}$
\item $\tn{Rt}_3(x)$ is defined to be $-\sqrt{x}$ when $x\in[0,1)$ and $\sqrt{x}$ when $x\in [1,\infty)$.
\item $\tn{Rt}_4(x)$ is defined to be $-\sqrt{-x}$ when $x\leq 0$ and $\sqrt{x}$ when $x\geq 0$.
\end{itemize}

\vfill
\newpage
\subsubsection*{Partial inverses, continued}

Construct at least three candidates for partial inverses (``fake inverses'') for $f(x)=\sin(x)$. 

Rank the candidates you have constructed from best to worst ``inverse'' for the sine function.

On what subset of the domain of sine do each of your candidates serve as a true inverse? On what subset of the domain of sine do your candidates serve as a fake inverse? 

\vspace*{2pt} \hrule \vspace*{2pt}
Construct at least three candidates for partial inverses (``fake inverses'') for $f(x)=\cos(x)$. 

Rank the candidates you have constructed from best to worst ``inverse'' for the cosine function.

On what subset of the domain of cosine do each of your candidates serve as a true inverse? On what subset of the domain of cosine do your candidates serve as a fake inverse? 


%%%%%%%%%%%%%%%%%%%%%%%%%%%%%%%%%
%%%%%%%%%%%%%%%%%%%%%%%%%%%%%%%%%%
%%%%%%% 4 COVARIATIONAL VIEW  %%%%%%%
%%%%%%%%%%%%%%%%%%%%%%%%%%%%%%%%%%
%%%%%%%%%%%%%%%%%%%%%%%%%%%%%%%%%%

\newpage 
\section{Comparing Correspondence and Covariation Views of Functions}
\label{s: function covariation}

\smallnote{Begin the class by reviewing Homework 3 Problems \ref{h: covariation intro morgan} and \ref{h: covariation linear relationship nonlinear variables}.}

\subsection{Introducing covariation}

So far we have been looking at things from a {\it correspondence view}, meaning that we primarily think of functions in terms of individual inputs and the images to which they are assigned. In other words, we focus on how inputs and outputs correspond. This perspective can be very useful for defining things like domain, range, composition, or inverse. However, this perspective can be at a distinct disadvantage when it comes to looking at the behavior of a function or of understanding how changes in the input and output variables influence each other. 

By {\it covariational view}, we mean understanding how changing the value of one variable impacts the value of the other variable, and learning to coordinate changes in one variable with changes in the other. We saw this already in Homework 3 Problems \ref{h: covariation intro morgan} and \ref{h: covariation linear relationship nonlinear variables}. We use the Morgan Minicase\footnote{(c) 2013, Educational Testing Service, used with permission} to dig deeper into the difference between correspondence and covariational views. This minicase also introduces the teaching practice of {\it recognizing and explaining correspondence and covariation views}. 

%Morgan item

\begin{mdframed}
\begin{center} {\bf Ms.~Morgan's class} \end{center}
\begin{minipage}{5in}\raggedright \parskip4pt
During a lesson on writing equations of linear functions represented in tables, Ms. Morgan
asked her students to write the equation of the linear function represented in the table below,
and to explain how they found their answers.

Students found the correct equation, but they gave different explanations of how they found
their answers:
\end{minipage}
\begin{minipage}{1in}
\vspace*{-20pt}
\begin{center}\begin{tabular}{c|c}
$x$ & $y$ \\ \hline
1 & 6  \\ 
2 & 11 \\ 
3 & 16 \\
4 & 21 
\end{tabular}
\end{center}
\end{minipage}

\begin{tabular}{L{0.75in}L{5in}}
\hline
Student A: & Each time the value of $x$ goes up by 1, the value of $y$ goes up by 5, so the
slope is 5. And if $x$ goes down by 1, then $y$ will have to go down by 5, so
the $y$-intercept is 1. That means the equation is $y = 5x + 1$.
\\ \hline
Student B: &  I just looked at the value of $y$ and saw that it kept increasing by 5, so $m =
5$. Then I subtracted that number from the first value
of y in the table, so $b = 1$. You always put m times x and add the b, so the
equation is $y = 5x + 1$.
\\ \hline
Student C: & For this function, I saw that you can always multiply the value of $x$ by 5
and then add 1 to get the value of $y$, so the equation is $y = 5x + 1.$
\\ \hline
\end{tabular}
\end{mdframed}

\begin{task}
Read through the student responses in Ms.~Morgan's class.

\begin{itemize}
\item {\bf Observe}: What is the student thinking? How might they have arrived at each step of their solution?
\item {\bf Interpret}: What are you sure that each student understands? What are you sure that each student
does not understand? What are you unsure that each student understands?  Based on what evidence?

Here are some concepts to consider analyzing for students' understanding:

	\vspace*{-2pt}
	\begin{itemize*}
	\item $y$-intercept
	\item Constant rate of change
	\item Form of a linear equation 
	\item How changes in one variable impact changes in the other variable 
	\item Definition of graph of a function
	\end{itemize*}
	
\item {\bf Interpret, continued:} Are the explanations mathematically complete? Why or why not?
	
\end{itemize}

\vspace*{2pt}
\hrule
\vspace*{2pt}

{\bf Observe}: What may Student A/B/C be thinking?

{\bf Interpret}: 

\begin{tabular}{C{1.75in}|C{1.75in}|C{1.75in}}
I am sure that Student A/B/C understands \dots  & I am sure that Student A/B/C
does NOT understand \dots & I am unsure whether
Student A/B/C understands \dots  \\ \hline
& & \\
\end{tabular}

What is complete or incomplete about Student A/B/C's explanation?
%\vspace*{2pt}
%\hrule
%\vspace*{2pt}
%
%{\bf Observe}: What may Student B be thinking?
%
%{\bf Interpret}: \begin{tabular}{C{1.5in}|C{1.5in}|C{1.5in}}
%I am sure that Student B understands \dots  & I am sure that Student B
%does NOT understand \dots & I am unsure whether
%Student B understands \dots  \\ \hline
%& & \\
%\end{tabular}
%
%What is complete or incomplete about Student B's explanation?
%\vspace*{2pt}
%\hrule
%\vspace*{2pt}
%
%{\bf Observe}: What may Student C be thinking?
%
%{\bf Interpret}: \begin{tabular}{C{1.5in}|C{1.5in}|C{1.5in}}
%I am sure that Student C understands \dots  & I am sure that Student C
%does NOT understand \dots & I am unsure whether
%Student C understands \dots  \\ \hline
%& & \\
%\end{tabular}
%
%What is complete or incomplete about Student C's explanation?
\end{task}

\begin{task}
There are two useful views on functions: covariation and correspondence. Use this space to take notes on what these mean.

Correspondence: 

Covariation: 

Then discuss: How do covariation and correspondence views come up in the students' thinking?
\end{task}


\begin{solution}({\it Partial})

Ways that the covariation and correspondence views arise in the Morgan Minicase include the following:

\vspace*{-8pt}
\begin{itemize*}
\item How changes in one variable impact changes in the other variable: Covariation view; and 
\item Definition of graph of a function: Correspondence view.
\end{itemize*}

\vspace*{-4pt}
Student A (does provide a mathematically complete explanation)

\vspace*{-8pt}
\begin{itemize*}
\item understands concept of constant rate of change, $y$-intercept, form of linear function
\item may not understand correspondence view of function
\end{itemize*}

\vspace*{-4pt}
Student B (does {\it not} provide a mathematically complete explanation)

\vspace*{-8pt}
\begin{itemize*}
\item does know that linear functions have the form $y=mx+b$, where $m, b\in \R$
\item does not understand constant rate of change
\item may not understand $y$-intercept
\item does not necessarily understand correspondence view of function
\end{itemize*}

\vspace*{-4pt}
Student C (does provide a mathematically complete explanation)

\vspace*{-8pt}
\begin{itemize*}
\item does know that linear functions have the form $y=mx+b$, where $m, b\in \R$
\item does understand correspondence view of function
\item may not understand constant rate of change, $y$-intercept.
\end{itemize*} 
\end{solution}

\subsubsection{Noticing student thinking and recognizing correspondence and covariation views}
In any teaching, it is important to attend to student work so that you can notice the student thinking. Some things to keep in mind for this are:

\begin{mdframed}\raggedright
\begin{center} {\bf Noticing student thinking} \end{center}
	\begin{itemize}
	\item First \emph{observe} what the student's thinking is, \emph{without judgment} as to what they understand or do not understand. 
	\item Then, \emph{interpret} what they understand, may not understand, or what you are unsure of whether they understand. Always base this interpretation on the evidence of the student thinking you have, and be sure that you know what evidence you are drawing upon.
		
	Only after this, you might interpret the completeness or correctness of the student's thinking. 
	\item From here, you might \emph{respond} to the student based on what you have observed and interpreted. 
	\end{itemize}
\end{mdframed}

The reason to split up observing without judgement, interpreting, and responding is that it interpretations tend to be more accurate after we have taken a step back to observe what the student may be thinking, without judgment.

In the case of teaching functions, it is helpful to be able to {\it recognize and explain correspondence and covariation viewpoints}. The Morgan Minicase gave us an opportunity to recognize how both can show up in response to the same problem as well as how differently they can appear. In the next part of this lesson, we will practice explanations from a covariation view. Throughout this minicase and in the next section, we keep in mind the following. 

\begin{mdframed}\raggedright
\begin{center} {\bf Recognizing and explaining correspondence and covariation viewpoints} \end{center}

Correspondence and covariation views can be thought of as the following:
\begin{itemize}
	\item (Correspondence) Conceiving of functions and their behavior primarily in terms of maps from individual elements of the domain to individual elements of the range. 
	\item (Covariation) Conceiving of functions and their behavior primarily in terms of coordinating how changes in the value of one variable impact the value of the other variable.
\end{itemize}

When introducing ideas in class, coming up with examples, or giving explanations, it is helpful to think about whether you are working with a correspondence or covariation view, and then to see what an explanation in the other view might look like. 

When noticing student thinking, it can be helpful to interpret whether they are taking a correspondence or covariation view.
\end{mdframed}

\subsubsection{Explaining from correspondence and covariation views: Building functions}


We now revisit our ways of building functions to see how the covariation view can come into play.

Let's begin by looking at compositions of functions that high school students are likely to encounter: linear functions and quadratic functions. 

\begin{bignote}
% Good questions project note
One way to use the following tasks may be the following:
	\begin{enumerate*}
	\item Pose the first question as a quick vote. 
	\item Ask pre-service teachers to talk to a neighbor about the answer.
	\item Go over the answer and the reasoning.
	\item Ask pre-service teachers to discuss the reasoning and ask questions about the reasoning.
	\end{enumerate*}
Then assign the follow-up questions and go over the reasoning as a whole class after giving students individual and/or group time to work on them.

This protocol is based on the Good Questions Project led by Maria Terrell at the Cornell University department of mathematics. She and colleagues observed that instructors who followed this protocol tended to have students who understood pre-calculus and calculus concepts better than instructors who only did 1 and 2, or who did only 3 or 4, or some other strict subset of these steps.  In fact, the students in their observations who fared worst on average were those whose instructors only did 3 and 4. 

One way to make sense of these findings is that students who vote and discuss their reasoning have created a personal stake in the mathematics, so they are more primed to listen to reasoning about the question, as well as to compare and contrast their reasoning with someone else's proposed reasoning. They may also be more likely to take the question seriously if they know that they initially got the wrong answer, because they understand what the challenge of the question is. When students only see the answer and the reasoning, even if it makes sense, they may also not appreciate the concepts that the question is getting at. So they are less likely to remember or take the concept seriously. 
\end{bignote}


\begin{task} 
Suppose that $f$ is a linear function whose constant rate of change is 5 output units per input unit; $g$ is a linear function whose constant rate of change is $-3$ output units per input unit; and $h$ is a linear function whose constant rate of change is $0$ output units per input unit.

What is the rate of change of $g\circ f$? Circle your response.

	$$2 \quad\quad -2 \quad\quad 15 \quad\quad -15 \quad\quad 8 \quad\quad -8\quad\quad \tn{it is not constant} \quad\quad \tn{none of the above}$$

What about for $f\circ g$? What is your reasoning? 

What about for $h\circ f$?  $f\circ h$? What is your reasoning?
\end{task}

{\it Correspondence view.} There are many ways to address this problem; for instance, we might find the answer using algebra, saying that if $f(x)=5x+b$ and $g=-3x+c$ then $g\circ f(x)=-15x+d$, where $d$ is a constant. This kind of explanation would be more from a correspondence viewpoint than a covariation one, because it looks at how inputs are assigned to outputs of each function and performs the composition for an individual element. It is not an explanation from a covariation view because it does not take into account how changes in one variable (input or output) of $g\circ f$ would impact the value of the other variable.  
Through similar reasoning, we can find that $f\circ g$ must have slope $-15$, and that $h\circ f$ and $f\circ h$ will both have slope $0$. (Try this yourself to make sure that you see why.)

{\it Covariation view.} Each time the value of $x$ changes by $+1$ unit, the value of $f(x)$ changes by $+5$ units, and the value of $g(x)$ changes by $-3$ units. Now let us look at what changing the value of $x$ by $+1$ unit does to the output of $g(f(x))$. So if we change $x$ by $+1$, then $f(x)$ changes by $+5$ units, which means $g(f(x))$ changes by $-3\cdot 5$ units. This is true for all $x$, so the rate of change for $g\circ f$ is $-15$ output units per input unit. 
We can use similar reasoning to solve the other cases. (Try this yourself.)

\smallnote{The covariation view may seem messier. Indeed, it often takes more words to say. However, there is good reason to learn a covariation view, one of which is that students seem to understand concepts of algebra, calculus, and statistics more deeply and intuitively when they can see from the covariation view.}

\begin{task}
Suppose that $L$ is a linear function, and $Q$ is a quadratic function.  What is $Q\circ L$? Circle your response.

\vspace*{-12pt}
\begin{center}
linear \quad\quad quadratic \quad\quad cubic \quad\quad something else
\end{center}

\vspace*{-12pt}
What is your reasoning? How would you explain this from a covariation view?

Let $\sin$ be the sine function. What is $\sin\circ L$? Circle your response.

\vspace*{-12pt}
\begin{center}
linear \quad\quad sinusoidal \quad\quad something else
\end{center}
\vspace*{-12pt}

(We can think of sinusoidal as behaving like sine or cosine: it moves ``up'' and ``down'' in a period fashion, always to the same maximum and minimum value.)

What is your reasoning? How would you explain this from a covariation view?

\end{task}

{\it Correspondence view.} The function $Q$ can be expressed as $Q(x)=ax^2+bx+c$, and $L$ can be expressed $L=mx+d$. Then $Q\circ L(x) = a(mx+d)^2+b(mx+d)+c$, which is still quadratic.  

{\it Covariation view.} We can think of a quadratic function as one whose rate of change of rate of change is constant. If we change the input variable to $Q$ linearly, as we would for $Q\circ L$, then the rate of change of the rate of change would multiply by the linear rate of change. It would still be constant. So the composition is still quadratic.

{\it Combining points of view.}
The function $\sin \circ L$ can be expressed as $x\mapsto \sin(mx+d)$. It still has a regular period (though the periods are now $1/m$ of the length they were before since the inputs are moving $m$ times as fast through $\sin$) and the maxima and minima stay the same. So $\sin\circ L$ is still sinusoidal.

We have so far looked at examples of composition. Now we examine inverses.

\begin{task}
Review: What is the high school definition of inverse of an invertible function? (Look this up to check whether you remembered correctly.)
\end{task}

\begin{task}
Suppose that $f$ is a linear function with constant rate of change 5 output units per input unit. 

What is $f^{-1}$? Circle your response. 
	\begin{center}
	a linear function with constant rate of change $-5$   
		
	a linear function with constant rate of change $5$  
	
	a linear function with constant rate of change $\frac{1}{5}$ 
	
	a linear function with constant rate of change $-\frac{1}{5}$ 
		
	none of the above
	\end{center}
What is your reasoning?

How does this make sense in terms of the high school definition of inverse of an invertible function?

How would you explain this from a covariation view? 
\end{task}

{\it Correspondence view.}  If $f(x)=5x+b$, the inverse function of $f$ maps $x$ to $y$ where $x=5y+b$. Solving for $y$, we obtain $y=\frac{x-b}{5}$, which is a linear function with constant rate of change $\frac{1}{5}$.

{\it Covariation view.} If $f$ is changing at a rate of 5 output units to 1 input unit, and we want to compose this with something that will get us back to 1 output unit per input unit (to satisfy the equation $f^{-1}\circ f(x)=x$), then we need $f^{-1}$ to slow the input $f(x)$ down by $\frac{1}{5}$ output units per input unit. So $f^{-1}$ is a linear function with with constant rate of change $\frac{1}{5}$.

\subsection{Bottle Problem}

\smallnote{This is a fun problem to do, time permitting. It builds off the idea of the Pot Problems from Homework 3 and is a nice example of covariational reasoning.}

\begin{task}
\begin{minipage}{5in} \raggedright \parskip4pt
Sketch a graph of the volume of water in the bottle as a function of the height, as the bottle is being filled up with water. 

Then sketch volume as a function of height. 
\end{minipage}
\begin{minipage}{1.1in}
\begin{center}
    \includegraphics[width=0.6in]{4_Bottle}
\end{center}
\end{minipage}
\end{task}

\subsection{Revisiting a key example}

Let us look back at a key example from the beginning of class. 

\begin{center}
\begin{tikzpicture}[scale=0.6]
\draw (0,-2.5) -- (0, 3.5) node[above]{$y$};
\draw (-6.25,0) -- (8.25,0) node [right]{$x$};
\foreach \x in {-6,-5,-4,...,8}
	\draw (\x,-0.1) -- (\x, 0.1);
\foreach \x in {-2,-1, ..., 3}
	\draw (-0.1,\x) -- (0.1,\x);
\draw[xshift=1cm, ultra thick, red] (-2,-2) -- (2,2);
\draw[ultra thick, red] (3,2) -- (5,2) node[above]{$f$} -- (8,2);
\draw[ultra thick, red] (-1,-2) -- (-6,-2);
\draw[yshift=1cm, ultra thick, dashed, blue] (-2,2) -- (2,-2);
\draw[ultra thick, dashed, blue] (2,-1) -- (5,-1)node[below]{$g$}-- (7.5,-1);
\draw[ultra thick, dashed, blue] (-2,3) -- (-5.5,3); 
\end{tikzpicture}
\end{center}

In the below, we ask about ``sections'' of a function. We are using this word informally and here use it to refer to  places where the rate of change is constant.

\begin{task}
What is the rate of change of each section of $f$? What is the domain of each section?
 What is the image of each section?

What is the rate of change of each section of $g$? What is the domain of each section?  What is the image of each section?

How many sections are there of $g\circ f$? What are they? What is the rate of change of each section?

How would you use the above information to graph $g\circ f$?

How did you use correspondence and covariation views in your reasoning?
\end{task}

\begin{solution} ({\it Partial.})
Sections of $f$:  \\
\begin{tabular}{L{1.75in}L{1.75in}L{1.75in}}
section (domain) & rate of change & image \\ \hline
$(-\infty, -1)$ & 0 & $\{-2\}$ \\
$(-1,3)$ & 1 & $(-2,2)$ \\
$(3,\infty)$ & 0 & $\{2\}$
\end{tabular}

Sections of $g$:  \\
\begin{tabular}{L{1.75in}L{1.75in}L{1.75in}}
section (domain) & rate of change & image \\ \hline
$(-\infty, -2)$ & 0 & $\{3\}$ \\
$(-2,2)$ & -1 & $(3,-1)$ \\
$(2,\infty)$ & 0 & $\{-1\}$
\end{tabular}

There are three sections of $g\circ f$, corresponding to the sections of $f$. These sections go nicely into the sections of $g$. The rate of change of each section is $0$, $1\cdot-1$, $0$. 
\end{solution}

\newpage

\subsection{Summary of mathematical/teaching practices}

\subsubsection*{Teaching definitions}
	\begin{itemize}
	\item Introductory examples and/or non-examples of the definition
	\item Precise statement of the definition
	\item Interpreting the precise statement, especially any new terminology or key rules, in terms of the introductory example and/or non-example
	\item Interpreting the terminology and rules in terms of the introductory examples, often using different \\ representations that students will continue to encounter.
	\end{itemize}
	
\subsubsection*{Explaining a Mathematical ``Test'' of a Property}

\begin{itemize}
\item Introduce {\it what}:	
	\begin{itemize*}
	\item Name the test.
	\item What is the test supposed to tell us? (Be precise!) 
	\item What are you testing? (Be precise!) 
	\end{itemize*} \vspace*{-12pt}$\;$

\item Describe {\it how}:
	\begin{itemize*}
	\item How do you do the test?
	\item How do you tell whether the thing passes or fails the test?
	\end{itemize*}

\item Deliver the {\it punchline}: What happens when the thing ``passes'' the test? What happens when the thing ``fails'' the test?
\item Explain {\it why} the test ``works'': 
\end{itemize}

\subsubsection*{Noticing student thinking}
\begin{itemize}
\item {\bf Observe}: What is the student thinking? How might they have arrived at each step of their solution?
\item {\bf Interpret}: What are you sure that each student understands? What are you sure that each student
does not understand? What are you unsure that each student understands?  Based on what evidence?	
\item {\bf Interpret, continued:} Are the explanations mathematically complete? Why or why not?
\item {\bf Respond}: Based on the above, determine how to respond.
\end{itemize}


\subsubsection*{Recognizing and explaining correspondence and covariation views}

Correspondence and covariation views can be thought of as the following:
\begin{itemize}
	\item (Correspondence) Conceiving of functions and their behavior primarily in terms of maps from individual elements of the domain to individual elements of the range. 
	\item (Covariation) Conceiving of functions and their behavior primarily in terms of coordinating how changes in the value of one variable impact the value of the other variable.
\end{itemize}

When introducing ideas in class, coming up with examples, or giving explanations, it is helpful to think about whether you are working with a correspondence or covariation view, and then to see what an explanation in the other view might look like. 

When noticing student thinking, it can be helpful to interpret whether they are taking a correspondence or covariation view.

%%%%%%%%%%%%%%
% \header{Preparation for next lesson}
%%%%%%%%%%%%%%%%%%%%%%%%%%%%%%%%%%%%%%%%
%%%%%%%%% 4 IN-CLASS RESOURCES %%%%%%%%%%%%%%%%%
%%%%%%%%%%%%%%%%%%%%%%%%%%%%%%%%%%%%%%%%
\newpage
\subsection{In-Class Resources}

%%%%%% 3-4 MORGAN MINICASE
\handout{Morgan Minicase}

\begin{mdframed}
\begin{center} {\bf Ms.~Morgan's class} \end{center}
\begin{minipage}{5in}\raggedright \parskip4pt
During a lesson on writing equations of linear functions represented in tables, Ms. Morgan
asked her students to write the equation of the linear function represented in the table below,
and to explain how they found their answers.

Students found the correct equation, but they gave different explanations of how they found
their answers:
\end{minipage}
\begin{minipage}{1in}
\vspace*{-20pt}
\begin{center}\begin{tabular}{c|c}
$x$ & $y$ \\ \hline
1 & 6  \\ 
2 & 11 \\ 
3 & 16 \\
4 & 21 
\end{tabular}
\end{center}
\end{minipage}

\begin{tabular}{L{0.75in}L{5in}}
\hline
Student A: & Each time the value of $x$ goes up by 1, the value of $y$ goes up by 5, so the
slope is 5. And if $x$ goes down by 1, then $y$ will have to go down by 5, so
the $y$-intercept is 1. That means the equation is $y = 5x + 1$.
\\ \hline
Student B: &  I just looked at the value of $y$ and saw that it kept increasing by 5, so $m =
5$. Then I subtracted that number from the first value
of y in the table, so $b = 1$. You always put $m$ times $x$ and add the $b$, so the
equation is $y = 5x + 1$.
\\ \hline
Student C: & For this function, I saw that you can always multiply the value of $x$ by 5
and then add 1 to get the value of $y$, so the equation is $y = 5x + 1.$
\\ \hline
\end{tabular}
\end{mdframed}

Read through the student responses in Ms.~Morgan's class.

\begin{itemize}
\item {\bf Observe}: What is the student thinking? How might they have arrived at each step of their solution?
\item {\bf Interpret}: What are you sure that each student understands? What are you sure that each student
does not understand? What are you unsure that each student understands?  Based on what evidence?

Here are some concepts to consider analyzing for students' understanding:

	\vspace*{-2pt}
	\begin{itemize*}
	\item $y$-intercept
	\item Constant rate of change
	\item Form of a linear equation 
	\item How changes in one variable impact changes in the other variable 
	\item Definition of graph of a function
	\end{itemize*}
	
\item {\bf Interpret, continued:} Are the explanations mathematically complete? Why or why not?
	
\end{itemize}

Record your thinking on the next page.

\newpage

{\bf Observe}: What may Student A be thinking?

{\bf Interpret}: 

\begin{tabular}{C{1.75in}|C{1.75in}|C{1.75in}}
I am sure that Student A understands \dots  & I am sure that Student A
does NOT understand \dots & I am unsure whether
Student A understands \dots  \\ \hline
& & \vspace*{1.75in}\\
\end{tabular}
\vspace*{2pt}
\hrule
\vspace*{2pt}

{\bf Observe}: What may Student B be thinking?

{\bf Interpret}: 

\begin{tabular}{C{1.75in}|C{1.75in}|C{1.75in}}
I am sure that Student B understands \dots  & I am sure that Student B
does NOT understand \dots & I am unsure whether
Student B understands \dots  \\ \hline
& & \vspace*{1.75in}\\
\end{tabular}

\vspace*{2pt}
\hrule
\vspace*{2pt}

{\bf Observe}: What may Student C be thinking?

{\bf Interpret}: 

\begin{tabular}{C{1.75in}|C{1.75in}|C{1.75in}}
I am sure that Student C understands \dots  & I am sure that Student C
does NOT understand \dots & I am unsure whether
Student C understands \dots  \\ \hline
& & \vspace*{1.75in}\\
\end{tabular}

%%%%%%%%%%%% 3-4 PRACTICING CORR AND COVAR VIEWS 
\newpage
\handout{Practicing Correspondence and Covariational Explanations}

There are two useful views on functions: covariation and correspondence. Use this space to take notes on what these mean.

Correspondence: 

Covariation: 

Then discuss: How do covariation and correspondence views come up in the students' thinking in Ms.~Morgan's class?

\newpage

\begin{enumerate}
\item 
Suppose that $f$ is a linear function whose constant rate of change is 5 output units per input unit; $g$ is a linear function whose constant rate of change is $-3$ output units per input unit; and $h$ is a linear function whose constant rate of change is $0$ output units per input unit.

What is the rate of change of $g\circ f$? Circle your response.

	$$2 \quad\quad -2 \quad\quad 15 \quad\quad -15 \quad\quad 8 \quad\quad -8\quad\quad \tn{it is not constant} \quad\quad \tn{none of the above}$$

What about for $f\circ g$? What is your reasoning? 

What about for $h\circ f$?  $f\circ h$? What is your reasoning?

\vfill 

\item Suppose that $L$ is a linear function, and $Q$ is a quadratic function.  What is $Q\circ L$? Circle your response.


\begin{center}
linear \quad\quad quadratic \quad\quad cubic \quad\quad something else
\end{center}

What is your reasoning? How would you explain this from a covariation view?

Let $\sin$ be the sine function. What is $\sin\circ L$? Circle your response.

\begin{center}
linear \quad\quad sinusoidal \quad\quad something else
\end{center}

(We can think of sinusoidal as behaving like sine or cosine: it moves ``up'' and ``down'' in a period fashion, always to the same maximum and minimum value.)

What is your reasoning? How would you explain this from a covariation view?

\vfill 

\item Review: What is the high school definition of inverse of an invertible function? (Look this up to check whether you remembered correctly.)

\vfill 
\item Suppose that $f$ is a linear function with constant rate of change 5 output units per input unit. 

What is $f^{-1}$? Circle your response. 
	\begin{center}
	a linear function with constant rate of change $-5$   
		
	a linear function with constant rate of change $5$  
	
	a linear function with constant rate of change $\frac{1}{5}$ 
	
	a linear function with constant rate of change $-\frac{1}{5}$ 
		
	none of the above
	\end{center}
What is your reasoning?

How does this make sense in terms of the high school definition of inverse of an invertible function?

How would you explain this from a covariation view? 

\end{enumerate}


%%%%%%%%%%%%%% 3-4 BOTTLE PROBLEM
\newpage
\handout{Bottle Problem}

\begin{minipage}{5in} \raggedright \parskip4pt
Sketch a graph of the volume of water in the bottle as a function of the height, as the bottle is being filled up with water. 

Then sketch volume as a function of height. 
\end{minipage}
\begin{minipage}{1.1in}
\begin{center}
    \includegraphics[width=0.6in]{4_Bottle}
\end{center}
\end{minipage}

%%%%%%%%%%%%%% 3-4 REVISIT KEY EXAMPLE
\newpage
\handout{Revisiting a key example}


Let us look back at a key example from the beginning of class. 

\begin{center}
\begin{tikzpicture}[scale=0.6]
\draw (0,-2.5) -- (0, 3.5) node[above]{$y$};
\draw (-6.25,0) -- (8.25,0) node [right]{$x$};
\foreach \x in {-6,-5,-4,...,8}
	\draw (\x,-0.1) -- (\x, 0.1);
\foreach \x in {-2,-1, ..., 3}
	\draw (-0.1,\x) -- (0.1,\x);
\draw[xshift=1cm, ultra thick, red] (-2,-2) -- (2,2);
\draw[ultra thick, red] (3,2) -- (5,2) node[above]{$f$} -- (8,2);
\draw[ultra thick, red] (-1,-2) -- (-6,-2);
\draw[yshift=1cm, ultra thick, dashed, blue] (-2,2) -- (2,-2);
\draw[ultra thick, dashed, blue] (2,-1) -- (5,-1)node[below]{$g$}-- (7.5,-1);
\draw[ultra thick, dashed, blue] (-2,3) -- (-5.5,3); 
\end{tikzpicture}
\end{center}

\begin{enumerate}
\item What is the rate of change of each section of $f$? What is the domain of each section?
 What is the image of each section?

\item What is the rate of change of each section of $g$? What is the domain of each section?  What is the image of each section?

\item How many sections are there of $g\circ f$? What are they? What is the rate of change of each section?

\item How would you use the above information to graph $g\circ f$?

\item How did you use correspondence and covariation views in your reasoning?
\end{enumerate}

%%%%%%%%% 3-4 HOMEWORK %%%%%%%%%%%%
\newpage \subsection{Homework for Chapters \ref{s: function correspondence}-\ref{s: function covariation}}


\begin{bignote}
Chapter 3 Homework Problems \ref{h: covariation intro morgan} and \ref{h: covariation linear relationship nonlinear variables} are used to set up Chapter 4 material. If Problem \ref{h: covariation linear relationship nonlinear variables} seems too long, it would work to assign only (a) and (c).

Chapter 4 Homework Problems \ref{h: horizontal shift intro} and \ref{h: horizontal scale intro} are used to set up Chapter 5 material.

From remaining problems of these chapters, we recommend assigning one or two parts of each of a subset of the questions, depending on your perception of how your students would most benefit. If you assign part of question 0, you might use google forms or another online submission so that you can read over responses prior to the next time you see your class. 

When grading the proof questions, we have often awarded full credit only to proofs that are both correct and that follow the proof communication guidelines established in Chapters 0 and 1. 

The purpose of these problems are as follows:
	\begin{enumerate}
	\item  Invertible functions; inverse of an invertible functions and non-invertible functions; correspondence view.
	\item Invertible functions; correspondence view; exploratory. If this question is assigned, it may be worth saying to students that you only expect as complete of a response as their best effort.
	\item Invertible function; correspondence view.
	\item Partial inverse of a function and how to construct them; correspondence view
	\item Explaining a mathematical ``test'' of a property; note that the last three may go somewhat beyond the scope of your course, depending on your students' background
	\item (Sets up Chapter 4) Introducing covariation; Morgan Minicase embedded problem
	\item (Sets up Chapter 4) Introducing covariation; linear relationship between two nonlinear relationships. I
	\item Invertible function; recognizing and explaining correspondence and covariation views
	\item Correspondence and covariation views for composition; follow-up to Chapter 2 Problem \ref{h: composition correspondence} (p.~\pageref{h: composition correspondence})
	\item (Sets up Chapter 5) Introducing horizontal shifts; covariation view.
	\item (Sets up Chapter 5) Introducing horizontal scaling; covariation view.
	\end{enumerate}
	
Additionally, we include two Simulations of Practice (SoP), whose titles and purposes are:
	\begin{itemize}
	\item (Written) Concept of Inverse: Introducing a definition; definition of invertible function and its inverse; using definition of graph of an equation; covariation view.
	\item (Video) Concepts of Linear Equations and Graphs: Noticing student thinking; Comparing covariation and correspondence views.
	\end{itemize}
	
Throughout the \MODULES materials, written SoPs engage prospective teachers in planning to lead a whole-class discussion on a given task; and video SoPs engage prospective teachers in noticing student thinking.
\end{bignote}


\begin{enumerate}
\setcounter{enumi}{-1}
\item In these chapters, we learned about:
 	\begin{itemize*}
	\item Functions and invertible functions
	\item Partial inverse of a function and how to construct them
	\item Correspondence and covariation views for inverse and composition
	\item The mathematical/teaching practices of:
		\begin{itemize*}
		\item Introducing a definition
		\item Explaining a mathematical ``test'' of a property
		\item Noticing student thinking
		\item Recognizing and explaining correspondence and covariation views.
		\end{itemize*}
	\end{itemize*}
	For each of these ideas: 
	\begin{enumerate}
	\item Where in the text are these ideas located?
	\item Review this section of the text. What definitions and results were important? How do examples use these definitions and results?
	\item What questions or comments do you have about the ideas in this section?
	\end{enumerate}
\end{enumerate}

\subsubsection*{Chapter \ref{s: function correspondence}}

\begin{enumerate}[resume]	
\item 
% Invertible functions; inverse of an invertible functions and non-invertible functions; correspondence view.
Explain why the following procedure works: 

\begin{quote}To find the formula for the inverse of an invertible function, switch the $y$'s and $x$'s then solve for $y$. \end{quote}

Explain why this procedure works in terms of the example:
	\begin{enumerate}
	\item $f(x) = 5x$
	\item $f(x) = x^3 -1$
	\item $f(x) = \frac{1}{x-3}$
	\end{enumerate}
Then:
	\begin{enumerate}[resume] 
	\item Explain why this procedure in general terms.
	\item Explain why this procedure is equivalent to reflecting the graph of the function about the line $y=x$.
	\item Explain what would work and what would not work if you were to use this procedure on non-invertible functions.
	\end{enumerate}
	
\item %  Invertible functions; correspondence view; exploratory. If you assign this question, you may want to warn your students that you do not expect a complete response, but you do expect them to try their best. 
(This problem comes from Mason, Burton, and Stacey (2010, p.~203)\footnote{Mason, J., Burton, L., Stacey, K. (2010). {\it Thinking Mathematically}. Essex, England: Pearson Education Limited.}) Under what conditions can you rotate the graph of a function about the origin, and still have the resulting graph being the graph of a function? If the graph of a function cannot be rotated about the origin without ceasing to be the graph of a function, might there be other points which could act as center of rotation and preserve the property of being the graph of a function?

\item % Invertible function; correspondence view.
	\begin{enumerate}
	\item Let $f$ and $g$ be two invertible functions.
	Explain why $g\circ f$ is invertible in terms of the middle school and university versions of the definition of relation.
	\item Explain why the inverse of $g\circ f$ should be $(f^{-1})\circ (g^{-1})$ in terms of the middle school version of the definition of composition.
	\item Explain why the inverse of $g\circ f$ should be $(f^{-1})\circ (g^{-1})$ in terms of the university version of the definition of invertible function.
	\end{enumerate}
	
\item % Partial inverse of a function and how to construct them; correspondence view
	\begin{enumerate}
	\item Construct three candidates for partial inverses for the sine function.
	\item On what subset of the domain of sine do each of your candidates serve as a true inverse? On what subset of the domain of sine do your candidates serve as only a partial inverse?
	\item Construct three candidates for partial inverses for the cosine function?
	\item On what subset of the domain of cosine do each of your candidates serve as a true inverse? On what subset of the domain of cosine do your candidates serve as only a partial inverse?
	\end{enumerate}
	
% Explaining a mathematical ``test'' of a property
\item  Explain the following tests of a mathematical property using the structure 
discussed in Section \ref{s: mathematical test of a property} (beginning p.~\pageref{s: mathematical test of a property}). In communicating your explanation, place the different sections of the structure separately from each other and label them.
	\begin{enumerate}
	\item Vertical line test.
	\item Horizontal line test.
	\end{enumerate}
	For the following tests of mathematical properties, in the section on {\it why} it works, provide an explanation in two parts, first in terms of a specific example, and second as a general explanation.
	\begin{enumerate}[resume]
	\item Testing points on a graph of a linear inequality to see which side to shade (e.g., \url{http://www.wtamu.edu/academic/anns/mps/math/mathlab/beg_algebra/beg_alg_tut24_ineq.htm}).
	
	{\it Note:} The definition of graph of an inequality in $x$ and $y$ is similar to that of graph of an equation: It is the set of points $(a, b)$ that if you evaluate the inequality at $x=a$ and $y=b$, you get a true statement. The property being tested here is of a half-plane, and whether the points in that half-plane satisfy a given inequality.
	
	\item Direct comparison test (for convergence of a series).
	\item Ratio test (for convergence of a series).
	\end{enumerate}

\item \label{h: covariation intro morgan}
Examine the table to the right. Based on the given information, discuss how change in $x$ appears to impact change in $y$. 

\begin{minipage}{5in}\raggedright
	\begin{enumerate}
	\item If $x$ changes by $\pm 1$ units, how does $y$ appear to change?
	\item If $x$ changes by $\pm h$ units, how does $y$ appear to change, in terms of $h$?
	\item Based on how change in $x$ impacts change in $y$, and using the given data, describe how you would find a plausible value of $y$ when $x$ is $0$. 
	\end{enumerate}
\end{minipage}
\begin{minipage}{1in}
\vspace*{-12pt}
\begin{center}
\begin{tabular}{c|c}
$x$ & $y$ \\ \hline
3 & 16 \\
5 & 26 \\
6 & 31 \\
9 & 46 \\
\end{tabular}
\end{center}
\end{minipage}

\item \label{h: covariation linear relationship nonlinear variables}
Pot A and Pot B have the radial cross section shown below. (This means that to get the shapes of Pot A and Pot B, you can rotate this cross section around a central axis.) The sides of Pot A are vertical. Both pots have a 1 gallon capacity.

\begin{enumerate}
\item Water is being poured into Pot A at an unsteady pace. Draw a graph that represents the relationship between volume of water and height of the water, with volume as input variable, height as output variable.
\item Draw a graph that represents the same relationship for Pot A, but this time with height as an input variable and volume as output variable.
\item Water is being poured into Pot B at an unsteady pace. Draw a graph that represents the relationship between volume of water and height of the water, with volume as input variable, height as output variable.
\item Draw a graph that represents the same relationship for Pot B, but this time with height as an input variable and volume as output variable.
\end{enumerate}
\end{enumerate}

\begin{center}
\begin{tikzpicture}[thick, scale=0.5]
\draw (-0.2, 5) -- (0,5) -- (0,0) node[below right]{Pot A}-- (3,0)-- (3, 5) -- (3.2, 5);  
\draw[xshift=10cm] (-1-0.2, 5) -- (-1,5) -- (0,0) node[below right]{Pot B}-- (2,0)-- (3, 5) -- (3.2, 5);  
\end{tikzpicture}
\end{center}

	
\newpage

\subsubsection*{Chapter \ref{s: function covariation}}
\begin{enumerate}[resume]
\item % Invertible function; recognizing and explaining correspondence and covariation views.
Suppose you are teaching the idea of inverse of a composition of invertible functions in a high school class. You want to explain why the inverse of $g\circ f$ should be $(f^{-1})\circ (g^{-1})$ with the example
of $f(x)=x+5$ and $g(x)=3x$.
	\begin{enumerate}
	\item Give an explanation from a correspondence view.
	\item Give an explanation from a covariation view.
	\end{enumerate}
	
\item 
% Correspondence and covariation views for inverse and composition; follow up from Homework 2 \ref{h: function composition} (p.~\pageref{h: composition correspondence}).

Below are graphs of the relations $f$ and $g$. The pieces of these graphs are lines and line segments, and their  turning points are integer coordinate points. Consecutive tick marks on the axes are distance 1 from each other.
    \begin{enumerate}
    \item What is the rate of change of each section of $f$? What is the domain of each section?
     What is the image of each section?
    
    \item What is the rate of change of each section of $g$? What is the domain of each section?  What is the image of each section?
    
    \item How many sections are there of $g\circ f$? What are they? What is the rate of change of each section?
    
    \item How would you use the above information to graph $g\circ f$? Cite specifically where you use each piece of information from (a), (b), and (c).
    
    \item How did you use correspondence and covariation views in your reasoning?
    \end{enumerate}


\begin{center}
\begin{tikzpicture}[scale=0.4]
\draw (0,-2.5) -- (0, 3.5) node[above]{$y$};
\draw (-6.25,0) -- (8.25,0) node [right]{$x$};
\foreach \x in {-6,-5,-4,...,8}
	\draw (\x,-0.1) -- (\x, 0.1);
\foreach \x in {-2,-1, ..., 3}
	\draw (-0.1,\x) -- (0.1,\x);
\draw[xshift=1cm, ultra thick, red] (-2,-2) -- (2,2);
\draw[ultra thick, red] (3,2) -- (5,2) node[above]{$f$} -- (8,2);
\draw[ultra thick, red] (-1,-2) -- (-6,-2);
\draw[yshift=1cm, ultra thick, dashed, blue] (-2,2) -- (2,-2);
\draw[ultra thick, dashed, blue] (2,-1) -- (5,-1)node[below]{$g$}-- (7.5,-1);
\draw[ultra thick, dashed, blue] (-2,3) -- (-5.5,3); 
\end{tikzpicture}
\end{center}
\end{enumerate}

Problems \ref{h: horizontal shift intro} and  \ref{h: horizontal scale intro} use the diagram below. Pot B has the radial cross section shown and has a 1 gallon capacity.


\begin{center}
\begin{tikzpicture}[thick, scale=0.5]
\draw[xshift=0cm] (-1-0.2, 5) -- (-1,5) -- (0,0) node[below right]{Pot B}-- (2,0)-- (3, 5) -- (3.2, 5);  
% \draw[xshift=10cm,xscale=2] (-1-0.1, 5) -- (-1,5) -- (0,0) node[below right]{Pot D}-- (2,0)-- (3, 5) -- (3.1, 5);  
\end{tikzpicture}
\end{center}

\begin{enumerate}[resume]
\item \label{h: horizontal shift intro}
Let $B$ be the function that maps volume of water in Pot B to height of water, when Pot B starts empty.

Let $C$ be the function that maps volume of water in Pot B to height of water, when Pot B starts with 4 cups of water already in it. (Look up how many cups are in a gallon.)

Let $v$ represent volume and $h$ represent height.
	\begin{enumerate}
		\item Graph $h=B(v)$ and $h=C(v)$ on the same set of $v$-$h$ axes.
		\item How do the graphs of $B$ and $C$ relate? Why does this make sense?
	\end{enumerate}

\item \label{h: horizontal scale intro}
Pot D fills up at twice the rate as Pot B, meaning that the rate of change of height with respect to volume for Pot D is twice that of Pot B. 
	\begin{enumerate}
	\item If Pot D and Pot B are equally tall, how much volume does Pot D hold?
	\item Let $D$ be the function that maps volume of water in Pot D to height of water, when Pot D starts empty. Graph $h=B(v)$ and $h=D(v)$ on the same set of $v$-$h$ axes.
	\item Which of the following best captures the relationship between $B(v)$ and $D(v)$?
	
		
		$$B(v)=2D(v) \quad\quad \quad
		B(v)=D(2v) \quad\quad \quad
		B(v)=\frac{1}{2}D(v) \quad\quad\quad 
		B(v)=D(\frac{1}{2}v) $$
		
	\end{enumerate}
\end{enumerate}



%%%%%%%%% PART II WRITTEN SIMULATION OF PRACTICE  %%%%%%%%%%%%
\newpage \subsection{Simulation of Practice: Concept of Inverse}

% SoP goals: Noticing student thinking; Using definition of invertible function and inverse of an invertible function

% Written SoPs might have the structure:
%Providing an example of a task or context that high school students would encounter (for geometry and algebra or statistics, this might be a task that could be found in curricular materials; for modeling and also statistics, this might be a real world context that high school students would work with)
%Ask pre-service teachers to solve the task or work with the context with multiple strategies and/or representations.
%Possibly providing a high school student response to the task or context.
%Ask pre-service teachers to describe how they would conduct a whole class discussion to elicit student thinking, including a list of discussion questions and how students might respond.
%Ask pre-service teachers to write a summary sentence or two they would use to conclude the discussion to get at the mathematical point of the discussion.
%Ask pre-service teachers to write a follow-up task or context that they might assign after this discussion concludes.

Suppose that you are teaching high school pre-calculus and you are introducing the concept of inverse of an invertible function. The students have already learned the definition of invertible function.  

Here is a task that you plan to use:

	\begin{mdframed}
	 A pot has straight vertical sides, stands 6 inches tall, and has 2 gallon capacity.  Water is being poured into this pot. 
		\begin{enumerate}[label=(\alph*)]
		\item Draw a picture of this pot.
		\item Find the relationship between $v$, for a volume of water poured into the pot, and $h$, the height of water in the pot at that volume. Graph this relationship.
		\item If there is $\frac{3}{4}$ gallon of water in the pot, how high is the water level?
		\item If the water level in the pot is $3$ inches high, how much water is in the pot?
		\item If the water level is $3.5$ inches high, how much water is in the pot?
		\end{enumerate}
	\end{mdframed}
	
You would like your class to understand both of the following definitions, and how these definitions can be interpreted using different representations, especially algebraic, graphical, and verbal.

	\begin{mdframed}
	{\bf Definition 1.}  Given an invertible function $f$, the inverse of $f$ is the function that maps $y\mapsto x$ whenever $x \mapsto y$ is an assignment of $f$. The inverse function is denoted $f^{-1}$.
	
	{\bf Definition 2.} Given an invertible function $f$, the inverse of $f$ is the function such that for all $x$ in the domain of $f$, we have $f^{-1}\circ f(x)=x$.
	\end{mdframed}

Break down your plan into the following sections. In communicating this plan, place these sections separately from each other and label them. The page limit for this plan is 2 pages (meaning 1 page double-sided or 2 pages single-sided.)

	\begin{itemize}
	\item {\bf Goals for the lesson}. Write this to be consistent with the scenario described above as well as the sections below. You may need to revise/reword this part as you work through the rest of the planning. (This is a very normal thing to happen when planning a lesson.)
	\item {\bf Solution to the task}. Include at least least three different solutions to (e). 
	\item {\bf Motivating example.}  Write down what you would say after the class has completed the activity to introduce the concept of inverse of a function. In this description, work in the notation you would use to refer the function being inverted, and how you would define this function and its inverse.
	\item {\bf Key terminology.} State the order that you would bring in each of Definitions 1 and 2. Then list the phrases or terminology in these definitions that you think students would benefit from discussion to understand.
	\item {\bf Illustrating concept with multiple representations.} Describe how you would use the problem context, algebraic notation, and graph to help students make sense of these phrases or terminology, and then the definition.
		\begin{itemize}
		\item What are 2-3 questions you would ask to lead a discussion on this?
		\item For each question, what would you anticipate students to respond?
		\item For each question, what would an ideal response be?
		\end{itemize}
	\item {\bf Mathematical equivalence of definitions.} Suppose you ask the class the question: ``Why are these two definitions saying the same thing?'' 
		\begin{itemize}
		\item What would you anticipate students to respond to this question?
		\item Describe two ideal responses to this question, one using the example, and one that is a general explanation.
		\end{itemize}
	\item {\bf Summary.} Write down a summary sentence or two that you would use to conclude the discussion. This sentences should be short enough to be easy to say, and drive home the main mathematical point of the lesson.
	\item {\bf Follow up.} Write a task you would assign for homework to launch a discussion on how to understand the following statement: if $f$ is an invertible function and $y=f(x)$, then $y=f^{-1}(x)$.
	\end{itemize}
	
\newpage

\subsubsection*{Feedback chart}

\begin{center}
\begin{tabular}{|L{2in}|L{2in}|L{2in}|}
\hline
{\bf Descriptor} & {\bf Meets Expectations} & {\bf Does Not Meet Expectations} \\ \hline
Are anticipated solutions realistic?	
& Both correct and incorrect solutions are generated using a variety of methods.	
& Solutions are not reasonable and/or do not include a variety of method and/or no correct solution is provided. \\ \hline
Is the whole class discussion plan reasonable? &	Discussion questions are included that follow a logical path and will move student thinking forward.&	Discussion questions are not logically sequenced and/or not appropriate to move thinking forward. 
\\ \hline
Are anticipated student responses for the whole class discussion reasonable?	& Anticipated student responses are reasonable. &	Anticipated student responses are unreasonable or not included.
\\ \hline
Will the task posed move thinking toward understanding the statement, {\it if $f$ is an invertible function and $y=f(x)$, then $y=f^{-1}(x)$}? &	Task provided will move students toward understanding statement.	
& Task provided does not logically follow from previous and/or will not move toward understanding the statement. 
\\ \hline
\end{tabular}
\end{center}

\subsubsection*{Reflection Prompt (to be completed after receiving feedback)}
\begin{enumerate}
\item What are some take-aways for you about using student thinking in moving toward a learning goal?
\item When you teach this concept in the future, what will you change?  What will you keep?  For what reasons? 
\end{enumerate}


%%%%%%%%% PART II VIDEO SIMULATION OF PRACTICE  %%%%%%%%%%%%
\newpage \subsection{Simulation of Practice: Concepts of Linear Equations and Graphs}

% SoP goals: Noticing student thinking; Comparing covariation and correspondence views

%Video SoPs might have the structure:
%Providing an example of a task or context that high school students would encounter (for geometry and algebra or statistics, this might be a task that could be found in curricular materials; for modeling and also statistics, this might be a real world context that high school students would work with)
%Providing a high school student response to the task or context.
%Ask pre-service teachers to respond to the student and
%Summarize the student?s thinking
%Say what is worthwhile about student thinking
%Help the student complete their thinking (if there are gaps in the thinking), prompts the student to investigate an error, or helps the student move forward in their thinking.
%Give feedback to student that they would be able to use in their future work in mathematics
%

{\it (This simulation of practice is an adaptation of the Allen Minicase, part of the Content Knowledge for Teaching Minicases project of the Educational Testing Service.)}

Suppose that you are working with your students to review for an end of year Algebra I exam. You include the following problem on a worksheet of practice problems that they completed for homework.

\begin{center}
\begin{minipage}{0.85\textwidth}
While visiting New York City, John kept track of the amount of money he spent on transportation by recording the distance he traveled by taxi and the cost of the ride.

\begin{center}
\begin{tabular}{C{1.2in}|C{1in}}
Distance, $d$, in miles & Cost, $C$, in dollars \\ 
\hline
3 & 8.25 \\
5 & 12.75 \\
11 & 26.25 \\
\end{tabular}
\end{center}

Show that the data can be represented by the linear function $C=2.25d+1.5$.
\end{minipage}
\end{center}

Your students used different methods to solve the problem; two solutions that you would like to go over with the class are that of Jing's and Matt's. You notice that one of these solutions is correct, and the other one has a mathematical error.
 
 \begin{tabular}{C{0.45\textwidth}|C{0.4\textwidth}}
 Jing & Matt  \\ 
 \hline 
$$  \frac{12.75 - 8.25 }{5-3}=\frac{4.50}{2}=2.25  $$
 \begin{eqnarray*} 
 C-26.25&=&2.25 (d-11) \\ 
C - 26.25 &=& 2.25d - 24.75\\
C &=& 2.25d + 1.5
 \end{eqnarray*}
  & 
   \begin{eqnarray*} 
  2.25 (3) + 1.5 &=& 8.25\\
2.25 (5) + 1.5& = &12.75 \\
  2.25 (11) + 1.5 &=& 26.25
   \end{eqnarray*}
 \end{tabular}

You are planning what you might say to the class in response to these methods. 

{\bf Record a video of yourself providing a response to students with these solution methods}

\begin{enumerate}
\item Summarize what Jing and Matt may be thinking.
\item Say what is worthwhile about Jing's thinking and Matt's thinking.
\item Help the students complete their thinking (if there are gaps in the thinking), prompt the students to investigate an error, or help the students move forward in their thinking.
\item Give specific feedback to students that they would be able to use in their future work in mathematics.
\end{enumerate}


\newpage

\subsubsection*{Feedback chart}

\begin{center}
\begin{tabular}{|L{2in}|L{2in}|L{2in}|}
\hline
{\bf Descriptor} & {\bf Meets Expectations} & {\bf Does Not Meet Expectations} \\ \hline
Is the summary of student
thinking reasonable?
& Summary points to reasonable
explanation of student
responses.
& Summary does not attend to
what students might have been
thinking. \\ \hline
Is the response to student 1
reasonable?
& Response to student 1
appropriately helps the student
complete their thinking,
prompts the student to
investigate an error, or helps
the student move forward in
their thinking.
& Response to student 1 does
not accurately assess student
understanding and move the
student in a reasonable
direction. \\ \hline
Is the response to student 2
reasonable?
& Response to student 2
appropriately helps the student
complete their thinking,
prompts the student to
investigate an error, or helps
the student move forward in
their thinking.
& Response to student 2 does
not accurately assess student
understanding and move the
student in a reasonable
direction. \\ \hline
Is the mathematical language
used appropriate?
& Oral description of
mathematical ideas uses
accurate mathematical
language.
& Oral description of
mathematical ideas does not
use accurate mathematical
language.\\ \hline
\end{tabular}
\end{center}


\subsubsection*{Reflection Prompt (to be completed after receiving feedback)}
\begin{enumerate}
\item What are some take-aways for you about using student thinking in moving toward a learning
goal?
\item In light of this experience, what are the mathematical points you will make sure to highlight
when you teach this in the future? What will you deemphasize? For what reasons?
\end{enumerate}


%%%%%%%%%%%%%%%%%%%%%%%%%%%%%%%% 
%%%%%%%%%%%%%%%%%%%%%%%%%%%%%%%% 	
%%%%%% PART III %%%%%%%%%%%%%%%%%%%%%	
%%%%%%%%%%%%%%%%%%%%%%%%%%%%%%%% 
%%%%%%%%%%%%%%%%%%%%%%%%%%%%%%%% 	
\newpage 
\part{Transformations of Functions} 

%%%%%%%%%%%%%%%%%%%%%%%%%%%%%%%% 	
%%%%%% LESSON 5-6 %%%%%%%%%%%%%%%%%%%%	
%%%%%%%%%%%%%%%%%%%%%%%%%%%%%%%% 
 \section{Defining transformations (Weeks 5-6) (Length: \about 6 hours)}  % Use Title Case for Title
%%%%%%%%% 5-6 OVERVIEW  %%%%%%%%%%%%%%%%%%%%
\subsection{Overview}

\begin{tabular}{L{6.5in}} 
{\bf Content} \\ \hline \parskip4pt
\emph{Thing 1}, defined as \dots 

\emph{Thing 2}, defined as \dots 
\end{tabular} 

\begin{tabular}{L{3.2in}|L{3.2in}}
{\bf Proof Structures} & {\bf Mathematical/Teaching Practices} \\ 
\hline \parskip4pt
% Proof structures
\emph{To show \dots} means \dots 
&
% Mathematical/Teaching Practices
\emph{Practice name}, meaning \dots
\end{tabular}
%%%%%%%%%%%%%%
\header{Summary}
% {\it Acknowledgements.}   <-- fill this in as appropriate.
%%%%%%%%%%%%%%
\begin{bignote}[Materials]
\begin{itemize*}
\item Handouts from In-Class Resources (can be printed double-sided)
         % \item other things as necessary, such as colored chalk or markers; other handouts; other props
\end{itemize*}
\end{bignote}
\todo{Ch 5: Write overview}

Prep for next lesson:
Make sure to assign mini lesson for homework between 5 and 6. 

This set of lessons is shorter so that you can have time to conduct a review session and/or have a buffer for the lessons 1-4 taking longer than expected.

%%%%%%%% 5-6 CONTENT %%%%%%
\newpage
\subsection{Many ways to compose to the same function}

\begin{verbatim} 
Jeremy:
> On p. 81, the scenario is unclear. What is the input quantity, x? Does (6,600) mean that after 6 hours, 
> the factory was making 600 candies per hour? Or... Do you mean that C is a function that outputs the 
> number of candies made per hour, or is C the name of a function that outputs the total number of candies 
> made after making candies for x hours?
I haven't looked at this yet but will put it into the notes for that Chapter to finalize later.
\end{verbatim}

\todo{Ch 5: respond to Jeremy's comment (shown here).}

\todo{Ch 5: Go over the problems that were assigned in Homework 4}

\todo{Ch 5: Read over and add narrative}
\begin{task}
Define:
$$A:\R\to \R, x\to 3x$$
$$B:\R\to \R, x\to x+4$$
$$E:\R\to\R, x\to x+12.$$
Write $x\mapsto 3x+12$ as a composition of $A, B, E$ in two different ways. (You do not have to use all of them in either of the ways.)
\end{task} 

\begin{solution}
$3x+12$ can be obtained by multiplying by $3$ first, then adding $12$; or adding $4$ first, then multiplying by $3$. 
\end{solution}

Observations based on problem:

\begin{itemize}
\item To make more complicated functions, we can sometimes compose simpler functions.
\item One of the more common patterns is decomposing expressions like $3x+12$ into multiplication/division then addition/subtraction, or addition/subtraction then multiplication/division.


	$$x \stackrel{\times 3}{\rightsquigarrow\!\!\rightsquigarrow} 3x \stackrel{+12}{\rightsquigarrow\!\!\rightsquigarrow}  (3x)+12 \tn{\quad or\quad} x \stackrel{+4}{\rightsquigarrow\!\!\rightsquigarrow}  x+4 \stackrel{\times 3}{\rightsquigarrow\!\!\rightsquigarrow} 3(x+4)$$  
\end{itemize}

\vspace*{-3pt}
\begin{task}

\begin{enumerate}
\item Find four different ways to start from $x$ and obtain the expression $2\sin(\frac{1}{3}x+12)-10$ through compositions.
\item Let
	$A:\R\to\R, x\mapsto 3x$, \quad $B:\R\to \R, x\mapsto x+4$, \quad $C:\R\to \R,x\mapsto 2x$, \quad$D:\R\to \R,x\mapsto x+5$. Write the map $f:\R\to\R, x\mapsto 2\sin(3x+12)+10$ in terms of $A, B, C, D$ and $\sin$. 
\end{enumerate}
\end{task}


\begin{solution}
\begin{enumerate}
\item Use different types of multiplication/division and addition/subtraction combinations on the inside and outside of the sin function.
\item $f(x)= C(D(\sin(A(B(x)))))$
\end{enumerate} \vspace*{-8pt}$\;$
\vspace*{-10pt}
\end{solution}


Observations based on problem:
\begin{itemize}
\item We can express the above multiplication/division and addition/subtraction combinations as function compositions. 
\item This leads into the idea of input transformations.
\end{itemize}

%%%%%%%%%%%%%%%%%%%%%
\subsection{Input transformations}

\begin{definition}\label{d: input transformation}
Given a function $f$, a function that can be expressed in the form $$f\circ T$$ is said to be an \emph{input transformation} of $f$. (The assignment rule is $f(T(x))$ instead of $f(x)$).
\end{definition}

\begin{task}
On January 1, 2101, Wall-E the robot wakes up at 9am, and at noon, he begins shelving away a large pile of metal toy blocks, one by one. He decides to put away 1 block every 6 minutes in the the first hour as he is figuring out his system for these blocks. Then, in the second hour, he puts away 1 block every 4 minutes. From there on, he puts away 1 block every minute. When will he have put away 144 blocks?
\end{task}

\begin{task}
Now let's represent some different scenarios with graphs. In each of the graphs:
	\vspace*{-4pt}\begin{itemize}
	\item Put a green dot at the point in the graph representing when Wall-E begins his work. 
	\item Put a red dot at the point in the graph representing when Wall-E has put away 144 blocks. 
 	\end{itemize}
 Let $W$ be the function of Wall-E's total blocks put away versus time. 
\begin{enumerate}
\item On the same set of axes, graph each of the following functions of Wall-E's total blocks put away versus time. 
	\begin{itemize}
	\item $W$, the function of Wall-E's total blocks put away versus time. 
	\item The function if Wall-E had started at 11:30am instead of noon. 
	\item The function if Wall-E had started at 10:30am instead of noon.
	\item  The function if Wall-E had started at 2pm instead of noon.
	\end{itemize}
\item On the same set of axes, graph each of the following functions of Wall-E's total blocks put away versus time.
	\begin{itemize}
	\item The function in the original situation.
	\item The function if Wall-E's internal clock is running 2 times slower (so what he thinks is an hour is actually 2 hours).
	\item The function is Wall-E's internal clock is running 1.5 times fast (so what he thinks is an hour is actually 40 minutes.)
	\end{itemize}
\end{enumerate}
\end{task}


{\it Example.} Let $W$ be the Wall-E function in the opening problem. What is $T$ for:
	\begin{enumerate}[label=(\alph*)]
	\item The scenario where Wall-E begins at 11:30am? 10:30am? 2pm?
	\item The scenario where Wall-E moves 2 times slower? 1.5 times faster?
	\item What if Wall-E began at 10:30am and moves 1.5 times faster?
	\item What would be a scenario that is represented by $W(3x+1)$? How about $W(3(x+1))$? (Here assume the unit of time are in hours).
	\end{enumerate}
	

{\bf Principles:} 
	\begin{itemize}
	\item If you transform the input to move faster per time, the graph of the function will appear to move faster per time. The graph will look as though you took the original graph and you squished the input coordinates of the coordinate plane, because you are moving faster through time. 
	\item If you transform the input to have a head start, the graph will look as though you pushed the origin forward, because you are effectively starting before the zero mark.
	\item If you transform the input to move slower per time, \dots [homework].
	\item If you transform the input to start later, \dots [homework].
	\end{itemize}

%%%%%%%%%%%%%%%
\subsection{Illustrating principles through word problems}
\label{section: transformation principles}

Common types of word problem contexts in high school and middle include:
	\begin{itemize}
	\item Races (on foot, by car, by bike, etc.). Input is typically time, output is typically distance. 
	\item Candle burning. Input is typically time, output is typically height of the candle. 
	\item Cell phone plan costs (solo plan, family plan, other incentives). Input is typically total duration of calls, output is typically cost.
	\item Mileage (of a motorcycle, car, truck, etc.). Input is typically distance, output is typically cost of gas or volume of gas used.
	\item Ferris wheel. Input is typically time, output is typically height from the ground.
	\end{itemize}

Each of these word problems can be used to illustrate the principles of input transformations. The strategy is typically to compare and contrast a function (which represents an situation) and a transformation of the function (which represents a competitor or alternate possibility).  

\bigskip

\begin{mdframed}
\begin{center} {\bf Characteristics of using a word problem to represent a transformation principle}\end{center}
	\begin{itemize}
	\item Use realistic input and output units.
	\item Use realistic input and output quantities.
	\item Clear problem statement.
	\item Ideal solution to the problem, in which you define the original function and its transformation with different variables.
	\item Commentary about the solution where you graph the function and its transformation and then discuss why the differences in the graph make sense in terms of the problem context and its solution. This part should lead sensibly to a statement of the principle(s) you are illustrating.
	\item State the principle(s) you are illustrating.
	\end{itemize}
\end{mdframed}

%%%%%%%%%%%%%%%%%%%%%%%%%%%%%%
\newpage \section{Output transformations and explaining definitions}

Today our agenda is:

\begin{itemize}
\item Mini lesson
\item Output transformations
\item Explanations with definition of graph
\end{itemize}


\subsection*{Mini lesson}

As you view the demo mini lesson, give feedback on:


\hspace*{-0.25in}\begin{tabular}{L{2.3in}|L{4.5in}}
\hline 
Are units and quantities realistic? 
	& 
	Input units and quantities realistic/not realistic
	
	Output units and quantities realistic/not realistic
\\ 
\hline
How clear is the problem statement?
	& 
	Clear / sort-of clear / unclear 
\\ 
\hline
How accurate is the solution to the problem? How clear is the solution to the problem?
	& 
	Accurate / sort-of accurate / not accurate  
	
	Clear / sort-of clear / unclear 
\\ 
\hline
Is the explanation of the graph strictly procedural (e.g., ``you move the graph by 1 unit to the right'') or does it combine the procedure with explanations both in terms of the definition of graph (in terms of input and output) and the problem context? (``Combine'' means ``connection is not implied, it is explicit'')
	&
	Strictly procedural / not strictly procedural 
	
	Connects explicitly with definition of graph / does not do so 
	
	Connects explicitly with problem context  / does not do so \\
\hline
How accurate are the statement of the principles? How clear are they? 
	& Accurate / sort-of accurate / not accurate  
	
	Clear / sort-of clear / unclear  \\ 
\hline
\end{tabular}

\subsection{Output transformations}

\begin{definition}
Given a function $f$, a function that can be expressed in the form $T\circ f$ is said to be an \emph{output transformation} of $f$. ($T$ takes as input the outputs of $f$.)
\end{definition}

{\it Example.} Wall-E encounters another stack of metal toy blocks to put away. This time, the toy boxes are glued in pairs, so in each move, he is actually moving 2 blocks rather than one block.  He otherwise makes the same moves as he did before. 
\begin{enumerate}[label=(\alph*)]
\item When does he put away 144 blocks?
\item What does the graph of blocks put away versus time looks like?
\item Express this function of blocks versus time in this scenario as a transformation of $W$.
\end{enumerate}

{\it Example.} Wall-E has already put away 144 blocks, and he is on to his next pile of metal blocks. .  He makes the same moves as he did before. 
\begin{enumerate}[label=(\alph*)]
\item What does the graph of total blocks put away versus time look like, including the 144 blocks in the count?
\item Express this function of blocks versus time in this scenario as a transformation of $W$.
\end{enumerate}

{\bf Principles:} 
	\begin{itemize}
	\item If you transform the output by a factor, the graph will be stretched by that factor. 
	\item If you transform the output by a constant up or down, graph will be moved up and down by that factor.
	\end{itemize}

%%%%%%%%%%%%%%%%%%%%%%%%%
\subsection{Explaining transformation principles in terms of the definition of graph}

Recall that graph of an equation $y=f(x)$ contains the point $(a,b)$ if and only if when you evaluate the equation $y=f(x)$ at $(a, b)$, you obtain a true statement. I.e., The point $(a,b)$ is on the graph if and only if ``$b=f(a)$'' is a true statement. 

\begin{task}
Suppose the point $(x,y)$ is on the graph $y=C(x)$, where $C$ represents candies made per hour at a candy factory. 

\begin{enumerate}[label=(\alph*)]
\item Let $T(x)=2x$. Does $C(T(x))$ represent more candies made or fewer candies made, per hour? 
\item If $(6,600)$ is on the graph of $y=C(x)$, what is a coordinate on the graph of $y=C(T(x))$?
\item Now let $T(x)=\frac{1}{3}x$. Write two scenarios, one described by $T(C(x))$, and the other described by $C(T(x))$.
\item If $(6,600)$ is on the graph of $y=C(x)$, what is a coordinate on the graph of $y=T(C(x))$? What is a coordinate on the graph of $y=C(T(x))$?
\end{enumerate}
\end{task}

\begin{task}
Use this task to explain input and output principles, incorporating the feedback that we discussed in the mini lesson. We will have some of you present so we can practice giving and receiving further feedback.
\end{task}


%%%%%%%%%%%%%%%%%%%%%%%%%%%%%%%%%%%%%%%%
%%%%%%%%% 5-6 IN-CLASS RESOURCES %%%%%%%%%%%%%%%%%
%%%%%%%%%%%%%%%%%%%%%%%%%%%%%%%%%%%%%%%%
\newpage \subsection{In-Class Resources}  
\handout{Many ways to compose to the same function}


Define:
$$A:\R\to \R, x\to 3x$$
$$B:\R\to \R, x\to x+4$$
$$C:\R\to \R,x\mapsto 2x$$
$$D:\R\to \R,x\mapsto x+5$$
$$E:\R\to\R, x\to x+12.$$



\begin{enumerate}
\item Write $x\mapsto 3x+12$ as a composition of $A, B, E$ in two different ways. (You do not have to use all of them in either of the ways.)
\end{enumerate}


\vfill

\begin{enumerate}[resume] 
\item \begin{enumerate}
\item Find four different ways to start from $x$ and obtain the expression $2\sin(\frac{1}{3}x+12)-10$ through compositions.
\item Write the map $f:\R\to\R, x\mapsto 2\sin(3x+12)+10$ in terms of $A, B, C, D$ and $\sin$. 
\end{enumerate}
\end{enumerate}

\vfill 

Takeaways:
\vfill


Solve the following problem:

\begin{mdframed}
On January 1, 2101, Wall-E the robot wakes up at 9am. At noon, Wall-E begins shelving away a large pile of metal toy blocks, one by one. He decides to put away 1 block every 6 minutes in the the first hour as he is figuring out his system for these blocks. Then, in the second hour, he puts away 1 block every 4 minutes. From there on, he puts away 1 block every minute. When will he have put away 144 blocks?
\end{mdframed}

\vfill

%%%%%%%%%% 5-6 INPUT 
\newpage \handout{Input Transformations}

Now let's represent some different scenarios with graphs. In each of the graphs:
	\begin{itemize}
	\item Put a green dot at the point in the graph representing when Wall-E begins his work. 
	\item Put a red dot at the point in the graph representing when Wall-E has put away 144 blocks. 
 	\end{itemize}
 Let $W$ be the function of Wall-E's total blocks put away versus time. 

\begin{enumerate}
\item On one pair of axes, graph each of the following functions of Wall-E's total blocks put away versus time. 
	\begin{itemize}
	\item $W$, the function of Wall-E's total blocks put away versus time.
	\item The function if Wall-E had started at 11:30am instead of noon. 
	\item The function if Wall-E had started at 10:30am instead of noon.
	\item  The function if Wall-E had started at 2pm instead of noon.
	\end{itemize}
	
\vfill 

\item On one pair of axes, graph each of following functions of Wall-E's total blocks put away versus time.
	\begin{itemize}
	\item The function in the original situation.
	\item The function if Wall-E's internal clock is running 2 times slower (so what he thinks is an hour is actually 2 hours).
	\item The function is Wall-E's internal clock is running 1.5 times fast (so what he thinks is an hour is actually 40 minutes.)
	\end{itemize}
\end{enumerate}


\vfill 
\newpage

\begin{enumerate}[resume]
\item Let $W$ be the function of Wall-E's total blocks put away versus time, as defined previously.
	\begin{enumerate}
	\item Describe a modification to the scenario in which the function of total blocks put away versus time would be $W(3x+1)$. 
	\item  Describe a modification to the scenario in which the function of total blocks put away versus time would be $W(3(x+1))$. 
	\item In each of the above scenarios you wrote, highlight in pink where the "3" comes in, and highlight in green where the "+1" comes in. 
	\item On the same pair of axes, graph $y=W(3x+1)$ and $y=W(3(x+1))$. Explain the differences you see in terms of the scenario.
	\end{enumerate} 
\end{enumerate}



%%%%%%%%%%%%%%%%%%%%%%%%%%%%%%%

\newpage
\handout{Mini lesson feedback}

Your name: 

Presenters' names:

Feedback:
\begin{tabular}{L{2.3in}|L{4.5in}}
\hline 
Are units and quantities realistic? \vspace*{0.5in}
	& 
	Input units and quantities realistic/not realistic
	
	Output units and quantities realistic/not realistic
\\ 
\hline
How clear is the problem statement?\vspace*{0.5in}
	& 
	Clear / sort-of clear / unclear 
\\ 
\hline
How accurate is the solution to the problem? How clear is the solution to the problem? \vspace*{0.5in}
	& 
	Accurate / sort-of accurate / not accurate  
	
	Clear / sort-of clear / unclear 
\\ 
\hline
Is the explanation of the graph strictly procedural (e.g., ``you move the graph by 1 unit to the right'') or does it combine the procedure with explanations both in terms of the definition of graph (in terms of input and output) and the problem context? (``Combine'' means ``connection is not implied, it is explicit'') \vspace*{1in}
	&
	Strictly procedural / not strictly procedural
	
	Connects explicitly with definition of graph /  does not 
	
	Connects explicitly with problem context / does not \\
\hline
How accurate are the statement of the principles? How clear are they?  \vspace*{1in}
	& Accurate / sort-of accurate / not accurate  
	
	Clear / sort-of clear / unclear  \\ 
\hline
\end{tabular}

Other comments:


\newpage \handout{Output transformations}

{\bf Defintion.}
Given a function $f$, a function that can be expressed in the form $T\circ f$ is said to be an \emph{output transformation} of $f$. ($T$ takes as input the outputs of $f$.)

\begin{enumerate} \item Wall-E encounters another stack of metal toy blocks to put away. This time, the toy boxes are glued in pairs, so in each move, he is actually moving 2 blocks rather than one block.  He otherwise makes the same moves as he did before. 
\begin{enumerate} 
\item When does he put away 144 blocks?
\item What does the graph of blocks put away versus time looks like?
\item Express this function of blocks versus time in this scenario as a transformation of $W$.
\end{enumerate}
\end{enumerate}

\vfill

\begin{enumerate}[resume]
 \item  Wall-E has already put away 144 blocks, and he is one to his next pile of metal blocks.  He makes the same moves as he did before. 
\begin{enumerate}
\item What does the graph of total blocks put away versus time look like, including the 144 blocks in the count?
\item Express this function of blocks versus time in this scenario as a transformation of $W$.
\end{enumerate}
\end{enumerate}

\vfill
%%%%%%%%%%%%%%%%%%%%%%%%%%%%%%
\newpage
\handout{Explaining transformation principles in terms of the definition of graph}

Recall that graph of an equation $y=f(x)$ contains the point $(x_0,y_0)$ if and only if when you evaluate the equation $y=f(x)$ at $(x_0, y_0)$, you obtain a true statement. I.e., the point $(x_0,y_0)$ is on the graph if and only if ``$y_0=f(x_0)$'' is a true statement. 


\begin{enumerate}
\item Suppose the point $(x,y)$ is on the graph $y=C(x)$, where $C$ represents candies made per hour at a candy factory. 

\begin{enumerate}
\item Let $T(x)=2x$. Does $C(T(x))$ represent more candies made or fewer candies made, per hour? 
\item If $(6,600)$ is on the graph of $y=C(x)$, what is a coordinate on the graph of $y=C(T(x))$?
\item Now let $T(x)=\frac{1}{3}x$. Write two scenarios:

	\centerline{one described by $T(C(x))$} 
	
	\centerline{and the other described by $C(T(x))$.}

\item If $(6,600)$ is on the graph of $y=C(x)$, what is a coordinate on the graph of $y=T(C(x))$? What is a coordinate on the graph of $y=C(T(x))$?
\end{enumerate}
\end{enumerate}

\vfill

\begin{enumerate}[resume]
\item Given a function $f$.
	\begin{enumerate}
	\item State {\it how} the graphs of $y=f(ax)$, and $y=f(-ax)$ can be obtained as transformations of the graph of $y=f(x)$
	\item Prove {\it why} your answer to (a) works. 	Your proof should use the definition of graph. 		
	\end{enumerate}
\end{enumerate}
\vfill


%%%%%%%%%%%%%%%%%%%%%%%%%%%%%%%

\newpage
\handout{Explaining with definition of graph: Feedback}

Your name: 

Presenters' names:

Feedback:
\begin{tabular}{L{2.3in}|L{4.5in}}
\hline 
Are units and quantities realistic? \vspace*{0.5in}
	& 
	Input units and quantities realistic/not realistic
	
	Output units and quantities realistic/not realistic
\\ 
\hline
How clear is the problem statement?\vspace*{0.5in}
	& 
	Clear / sort-of clear / unclear 
\\ 
\hline
How accurate is the solution to the problem? How clear is the solution to the problem? \vspace*{0.5in}
	& 
	Accurate / sort-of accurate / not accurate  
	
	Clear / sort-of clear / unclear 
\\ 
\hline
Is the explanation of the graph strictly procedural (e.g., ``you move the graph by 1 unit to the right'') or does it combine the procedure with explanations both in terms of the definition of graph (in terms of input and output) and the problem context? (``Combine'' means ``connection is not implied, it is explicit'') \vspace*{1in}
	&
	Strictly procedural / not strictly procedural
	
	Connects explicitly with definition of graph /  does not 
	
	Connects explicitly with problem context / does not \\
\hline
How accurate are the statement of the principles? How clear are they?  \vspace*{1in}
	& Accurate / sort-of accurate / not accurate  
	
	Clear / sort-of clear / unclear  \\ 
\hline
\end{tabular}

Other comments:


%%%%%%%%% 5-6 HOMEWORK %%%%%%%%%%%%
\newpage \subsection{Homework}

\todo{Ch 5-6: write homework}

\begin{enumerate}
\item Let $W$ be the function of Wall-E's total blocks put away versus time, as defined in the Lesson 5 Explorations. {\it Hint: It may be helpful for this homework problem to put the origin at 9am instead of at noon.}
	\begin{enumerate}
	\item Describe a modification to the scenario in which the function of total blocks put away versus time would be $W(3x+1)$. 
	\item  Describe a modification to the scenario in which the function of total blocks put away versus time would be $W(3(x+1))$. 
	\item In each of the above scenarios you wrote, highlight in pink where the "3" comes in, and highlight in green where the "+1" comes in. 
	\item On the same pair of axes, graph $y=W(3x+1)$ and $y=W(3(x+1))$. Explain the differences you see in terms of the scenario.
	\end{enumerate} 
\end{enumerate}

\begin{enumerate}[resume]
\item Complete the following:
	\begin{mdframed}
{\bf Principles of graph transformation} 
	\begin{itemize}
	\item If you transform the input to move faster per time, the graph of the function will appear to move faster per time. The graph will look as though you took the original graph and you squished the input coordinates of the coordinate plane, because you are moving faster through time. 
	\item If you transform the input to have a head start, the graph will look as though you pushed the origin forward, because you are effectively starting before the zero mark.
	\item If you transform the input to move slower per time:
	
	\bigskip
	
	\item If you transform the input to start later:
	
	\bigskip\bigskip
	
	\end{itemize}
	\end{mdframed}
	
\item Read over Section \ref{section: transformation principles} on illustrating principles of transformations. 

\begin{itemize} 
\item Prepare a 7-minute lesson in which you explain the solution to a problem that combines a faster/slower per time transformation with a head start/delayed start transformation. 

\item Assume that you had assigned the problem for homework, and most students have done the problem correctly. You are now reviewing the problem so as to reinforce the principles.

\item The problem should use one of the common contexts listed in Section \ref{section: transformation principles}.

\item Your explanation should adhere to the characteristics of a good explanation listed in that section. 

\item Your lesson should include some comprehension questions about the graphs that tie directly into the statement of the principles.

\end{itemize}
\end{enumerate} 

Here is the feedback chart that will be used for the lesson:

\hspace*{-0.25in}\begin{tabular}{L{2.3in}|L{4.5in}}
\hline 
Are units and quantities realistic? 
	& 
	Input units and quantities realistic/not realistic
	
	Output units and quantities realistic/not realistic
\\ 
\hline
How clear is the problem statement?
	& 
	Clear / sort-of clear / unclear 
\\ 
\hline
How accurate is the solution to the problem? How clear is the solution to the problem?
	& 
	Accurate / sort-of accurate / not accurate  
	
	Clear / sort-of clear / unclear 
\\ 
\hline
Is the explanation of the graph strictly procedural (e.g., ``you move the graph by 1 unit to the right'') or does it combine the procedure with explanations both in terms of the definition of graph (in terms of input and output) and the problem context? (``Combine'' means ``connection is not implied, it is explicit'')
	&
	Strictly procedural / not strictly procedural 
	
	Connects explicitly with definition of graph / does not do so 
	
	Connects explicitly with problem context  / does not do so \\
\hline
How accurate are the statement of the principles? How clear are they? 
	& Accurate / sort-of accurate / not accurate  
	
	Clear / sort-of clear / unclear  \\ 
\hline
\end{tabular}

\end{document} 

Cut and paste the below for a new part and/or lesson:
%%%%%%%%%%%%%%%%%%%%%%%%%%%%%%%% 
%%%%%%%%%%%%%%%%%%%%%%%%%%%%%%%% 	
%%%%%% PART # %%%%%%%%%%%%%%%%%%%%%	
%%%%%%%%%%%%%%%%%%%%%%%%%%%%%%%% 
%%%%%%%%%%%%%%%%%%%%%%%%%%%%%%%% 	
\part{Part Title} 
%%%%%%%%%%%%%%%%%%%%%%%%%%%%%%%% 	
%%%%%% LESSON # %%%%%%%%%%%%%%%%%%%%	
%%%%%%%%%%%%%%%%%%%%%%%%%%%%%%%% 
\newpage \section{Lesson Title (Length: \about ## minutes)}  % Use Title Case for Title
%%%%%%%%% # OVERVIEW  %%%%%%%%%%%%%%%%%%%%
\subsection{Overview}
\subsection{Overview}

\begin{tabular}{L{6.5in}} 
{\bf Content} \\ \hline \parskip4pt
\emph{Thing 1}, defined as \dots 

\emph{Thing 2}, defined as \dots 
\end{tabular} 

\begin{tabular}{L{3.2in}|L{3.2in}}
{\bf Proof Structures} & {\bf Mathematical/Teaching Practices} \\ 
\hline \parskip4pt
% Proof structures
\emph{To show \dots} means \dots 
&
% Mathematical/Teaching Practices
\emph{Practice name}, meaning \dots
\end{tabular}
%%%%%%%%%%%%%%
\header{Summary}
% {\it Acknowledgements.}   <-- fill this in as appropriate.
%%%%%%%%%%%%%%
\begin{bignote}[Materials]
\begin{itemize*}
\item Handouts from In-Class Resources (can be printed double-sided)
         % \item other things as necessary, such as colored chalk or markers; other handouts; other props
\end{itemize*}
\end{bignote}
%%%%%%%% # CONTENT %%%%%%
\subsection{Content}
%%%%%%%%%%%%%%
\header{Preparation for next lesson}
%%%%%%%%%%%%%%%%%%%%%%%%%%%%%%%%%%%%%%%%
%%%%%%%%% # IN-CLASS RESOURCES %%%%%%%%%%%%%%%%%
%%%%%%%%%%%%%%%%%%%%%%%%%%%%%%%%%%%%%%%%
\newpage \subsection{In-Class Resources}  
\handout{Handout Title}
%%%%%%%%% # HOMEWORK %%%%%%%%%%%%
\newpage \subsection{Homework}
%%%%%%%%% # WRITTEN SIMULATION OF PRACTICE  %%%%%%%%%%%%
\newpage \subsection{Simulation of Practice: Title of Simulation 1}
%%%%%%%%% # VIDEO SIMULATION OF PRACTICE  %%%%%%%%%%%%
\newpage \subsection{Simulation of Practice: Title of Simulation 2}
%%%%%%%%% # ADDITIONAL NOTES %%%%%%%%%%%%%%%%%%%%
\newpage \subsection{Title of an Additional Note} 

